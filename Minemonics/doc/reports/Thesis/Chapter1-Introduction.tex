\documentclass[main]{subfiles}

\begin{document}
% Chapter Template
\setcounter{chapter}{0}

%  Locomotion
%  A more natural approach
%   Evolutionary Algorithms
%   Chaotic systems as a source of high variability
%   Simple limiter control
\chapter{Introduction} % Main chapter title

\label{Chapter\thechapter} % Change X to a consecutive number; for referencing this chapter elsewhere, use \ref{ChapterX}

\lhead{Chapter \thechapter. \emph{Introduction}} % Change X to a consecutive number; this is for the header on each page - perhaps a shortened title

\section{Relating adaptability and locomotion}

Motion is a central aspect of life for organisms on earth. In order to survive and be competitive in the endless development of evolution, one has to be highly adaptive on all levels from the microscopic level of the DNA to the macroscopic level of the morphology of an organism. The adaptability of an organism permits it to find a niche, live in it and adapt to it in order to simplify the exploitation the environment of the niche. Primary producers have found a way to be adaptable completely without self-generated motion, usually growing and living in one specific spot without the possibility of relocation, mainly relying on environmental forces such as wind or water to distribute seeds. Combined with an r-selection strategy living in an unstable environment, this might be a valid approach to quickly spread within and dominate a certain niche. For organisms of higher complexity, however, it might not suffice to rely on environmental forces to relocate. The ability to locomote enables an organism to move to another environment if the attractivity of its current environment is reduced due to environmental changes or due a strong competitor. // If a population of organisms finds a .Locomotion apparently 



\todo[inline]{Write an introduction on locomotion.}

\section{The engineering approach on locomotion}

\lipsum[1]

\todo[inline]{Describe some engineering approaches on how to built robots that locomote.}

\section{A more natural approach}

Locomotion in animals seems solve the problem of getting from point A to point B extremely easily and elegantly. Even under constraints of rough terrain and obstacles interfering with the goal, a solution is nearly always found and does generally not include a long planning time. On the other hand, engineering solutions usually rely on a longer planning times and still do not reach competitive solutions in term of leg placement and adaptability to the environment. Conventionally, the need for adaption has to be detected and then a new gait or leg position has to be calculated. Using a system that adaptively produces a variety of periodic movements, the calculation intensive controller could be replaced by the periodic controller. The interaction of the body and the controller with the environment automatically constrain the motion space of the legs in order to make the creature move forward. An example of periodicity generating controller is a chaotic, dynamical system that gets controlled using simple limiter control. The sensors from the joints as well as the weight, shape and inertia of the limbs act as limiters. The goal of the thesis is to use evolutionary algorithms and an example of genotype to phenotype transcription to let creatures evolve in a rigid-body engine to show examples of evolved creatures that use the above described system to move on different kinds of ground and adapt its movement depending on the ground.

\todo[inline]{Give some drawbacks on the engineering approach and how nature finds new solutions in order to make locomotion possible.}

\subsection{Evolutionary Algorithms}

Evolutionary algorithms mimic the general way of how evolution as a self-optimizing process finds competitive individuals able to survive on earth. The general approach is based on a population of individuals, each based on a genotype, an element defining the characteristics of the individual. The genotype is the blueprint of the animal and subject to variation through mutation and crossover. The fitness functions, which model the environmental constraints that define what has to be achieved in order to be called fit, are applied to all individuals of the population to measure their fitness. If an individual is competitive with respect to the fitness functions, it has a higher chance to reproduce than a non-competitive individual. This means that the genotypes of individuals with high fitness values then are crossed, which forms a new genotype which is a mixture of the parent's genotypes. All individuals are subject to one or multiple types of mutations, which lead to slight changes in the genotype, thereby also leading to changes in the fitness of that individual. In order to keep the size of the population constant, some of the non-competitive individuals are culled. Using this process, nearly optimal solutions can be found to very complicated problems can be found such as the travelling salesman problem [?look at Ruedi's notes]. This is due to the features of the variation operators, crossover leading to mixtures of different solutions which possibly are a better solution than the crossed solutions, and mutation helping to avoid getting stuck in local minima. It must be noted that the process per se does not aim to generate better individuals and that variating does not necessarily lead to a better fitness of an individual, but the population under an evolutionary process will adapt to the fitness landscape and very fit individuals will ...

\todo[inline]{Describe the general approach of evolutionary algorithms.}

\subsection{Chaotic systems as a source of high variability}

\lipsum[1]


\todo[inline]{Write a short section on chaotic systems and their applicability as a source of periodical and chaotic movement.}

\subsection{Simple limiter control}

\lipsum[1]

\todo[inline]{Describe the natural approach on limiter control.}

\end{document}