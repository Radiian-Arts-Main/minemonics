%% See the TeXed file for more explanations

%% [OPT] Multi-rowed cells in tabulars
\usepackage{multirow}

%% [REC] Intelligent cross reference package. This allows for nice
%% combined references that include the reference and a hint to where
%% to look for it.
\usepackage{varioref}

%% [OPT] Easily changeable quotes with \enquote{Text}
%\usepackage[german=swiss]{csquotes}

%% [REC] Format dates and time depending on locale
\usepackage{datetime}

%% [OPT] Provides a \cancel{} command to stroke through mathematics.
\usepackage{cancel}

%% [NEED] This allows for additional typesetting tools in mathmode.
%% See its excellent documentation.
\usepackage{mathtools}

%% [ADV] Conditional commands
%\usepackage{ifthen}

%% [OPT] Manual large braces or other delimiters.
%\usepackage{bigdelim, bigstrut}

%% [REC] Alternate vector arrows. Use the command \vv{} to get scaled
%% vector arrows. (package texlive-fonts-extra)
\usepackage[h]{esvect}

%% [NEED] Some extensions to tabulars and array environments.
\usepackage{array}

%% [OPT] Postscript support via pstricks graphics package. Very
%% diverse applications.
%\usepackage{pstricks,pst-all}

%% [?] This seems to allow us to define some additional counters.
%\usepackage{etex}

%% [ADV] XY-Pic to typeset some matrix-style graphics
%\usepackage[all]{xy}

%% [OPT] This is needed to generate an index at the end of the
%% document.
%\usepackage{makeidx}

%% [OPT] Fancy package for source code listings.  The template text
%% needs it for some LaTeX snippets; remove/adapt the \lstset when you
%% remove the template content.
\usepackage{listings}
\lstset{language=TeX,basicstyle={\normalfont\ttfamily}}
\usepackage{color}

\definecolor{mygreen}{rgb}{0,0.6,0}
\definecolor{mygray}{rgb}{0.5,0.5,0.5}
\definecolor{mymauve}{rgb}{0.58,0,0.82}
\definecolor{darkblue}{rgb}{0,0,.75}

\lstloadlanguages{Matlab} %use listings with Matlab for Pseudocode
\lstnewenvironment{PseudoCode}[1][]
{
\lstset{ %
  language=Matlab,
  backgroundcolor=\color{white},   % choose the background color; you must add \usepackage{color} or \usepackage{xcolor}
  basicstyle=\scriptsize,        % the size of the fonts that are used for the code
  breakatwhitespace=false,         % sets if automatic breaks should only happen at whitespace
  breaklines=true,                 % sets automatic line breaking
  captionpos=b,                    % sets the caption-position to bottom
  commentstyle=\color{mygreen},    % comment style
  deletekeywords={...},            % if you want to delete keywords from the given language
  escapeinside={\%*}{*)},          % if you want to add LaTeX within your code
  extendedchars=true,              % lets you use non-ASCII characters; for 8-bits encodings only, does not work with UTF-8
  frame=single,	                   % adds a frame around the code
  keepspaces=true,                 % keeps spaces in text, useful for keeping indentation of code (possibly needs columns=flexible)
  keywordstyle=\color{darkblue},       % keyword style
  language=Octave,                 % the language of the code
  otherkeywords={*,...},           % if you want to add more keywords to the set
  numbers=left,                    % where to put the line-numbers; possible values are (none, left, right)
  numbersep=5pt,                   % how far the line-numbers are from the code
  numberstyle=\tiny\color{mygray}, % the style that is used for the line-numbers
  rulecolor=\color{black},         % if not set, the frame-color may be changed on line-breaks within not-black text (e.g. comments (green here))
  showspaces=false,                % show spaces everywhere adding particular underscores; it overrides 'showstringspaces'
  showstringspaces=false,          % underline spaces within strings only
  showtabs=false,                  % show tabs within strings adding particular underscores
  stepnumber=1,                    % the step between two line-numbers. If it's 1, each line will be numbered
  stringstyle=\color{mymauve},     % string literal style
  tabsize=2,	                   % sets default tabsize to 2 spaces
  title=\lstname,                   % show the filename of files included with \lstinputlisting; also try caption instead of title
  #1
}
}
{}

%% [REC] Fancy character protrusion.  Must be loaded after all fonts.
\usepackage[activate]{pdfcprot}

%% [REC] Nicer tables.  Read the excellent documentation.
\usepackage{booktabs}

%% International System measurement units (package texlive-science)
\usepackage{siunitx}

%% Subfigures
%\let\subcaption\undefined
%\let\subfloat\undefined
%\usepackage{subcaption}

%section customisation
\usepackage{titlesec}

%% Advanced figures
\usepackage{tikz}
\usetikzlibrary{backgrounds,fit,decorations.pathreplacing,positioning,fadings}
\usetikzlibrary{arrows,decorations.markings}
\usepackage{pgfplots,pgfplotstable}

%% Electronics circuits
\usepackage[arrowmos]{circuitikz}

%%Image position
\usepackage{float}
\usepackage{subcaption}

%%Long tables
\usepackage{longtable}
\usepackage{tabu}

%% LaTeX' own graphics handling
\usepackage{graphicx}

%% The AMS-LaTeX extensions for mathematical typesetting.  Do not
%% remove.
\usepackage{amsmath,amssymb,amsfonts,mathrsfs,amscd,xspace}

%% NTheorem is a reimplementation of the AMS Theorem package. This
%% will allow us to typeset theorems like examples, proofs and
%% similar.  Do not remove.
%% NOTE: Must be loaded AFTER amsmath, or the \qed placement will
%% break
\usepackage[amsmath,thmmarks]{ntheorem}

%different enumerations
\usepackage{enumitem}

%% Make document internal hyperlinks wherever possible. (TOC, references)
%% This MUST be loaded after varioref, which is loaded in 'extrapackages'
%% above.  We just load it last to be safe.
\usepackage[linkcolor=black,colorlinks=true,urlcolor=black,citecolor=black,filecolor=black]{hyperref}


\input{glyphtounicode}
  \pdfgentounicode=1
\usepackage{cmap}

\usepackage{accsupp}
\usepackage{calc}
\usepackage{layouts}
\usepackage{layout}

 
\mathtoolsset{showonlyrefs}

\usepackage{xpatch}
\xpretocmd{\eqref}{Equation ~}{}{}

% Lorem ipsum
%\usepackage[]{blindtext}
\usepackage{lipsum}% dummy text
% include pdfs into the latex document
\usepackage{pdfpages}
%for landscape cheatsheet
\usepackage{pdflscape}


% Units
\usepackage{units}

% tables
\usepackage{array}
\usepackage{rotating}
\usepackage{longtable}

%layout
\usepackage{multicol}
\setlength{\columnseprule}{0.4pt}
\usepackage{chngpage}
