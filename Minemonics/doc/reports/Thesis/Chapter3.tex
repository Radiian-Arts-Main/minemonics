% Chapter 3
\documentclass[main]{subfiles}
\setcounter{chapter}{2}

\begin{document}
\chapter{Prosthetic Constraints \& Requirements Definition} % Main chapter title

\label{Chapter3} % For referencing the chapter elsewhere, use \ref{Chapter1} 

\lhead{Chapter 3. \emph{Prosthetic Constraints \& Requirements Definition}} % This is for the header on each page - perhaps a shortened title

The design of a prosthesis has come a long way so far. Based on several studies on prosthesis abandonment, the constraints for a new prosthesis are listed and discussed one by one and what they mean in particular for a pneumatic prosthetic control system. Further listed are the requirements of a prosthetic device and what must be considered in a pneumatically actuated device.

%----------------------------------------------------------------------------------------
%Prosthetic Requirements & Constraints Definition
% Dexterity of the hand system
% Representation of different muscles in the prosthesis
% Analysis of the constraints(weight, size,noise)

\section{Analysis of the constraints}
Physics and the current state of technology constrain our life similar to two layers of possibilities. While physical laws are unchangeable and give us boundaries to what is possible and what is not, the current state of technology is a temporary boundary that just tells us that a certain action is not possible with the current state of the art. For example, it is physically possible to fly in very delicate ways in between houses as birds do or to live in the deep sea like the anglerfish, but the current state of the art does not permit a human to do either of those actions. It is very difficult to find the exact boundaries of the physically possible since this can only be approached by the current state of technology. At least we can list what is possible today and which are the constraints that we must keep in mind when it comes to designing a prosthetic control system.

\subsection{Size}

A constraint basically defined by nature is size. Every animal has its defined optimal size and its limbs are set into proportions to the overall size of the body. Robotic hands on the one hand, which are copying the human hand in dexterity and in some cases even best the natural model in speed and precision, do not necessarily meet the size constraint. Prostheses on the other hand have to meet this constraint to be able to even get into the market. The best highly dexterous hand being able to perform prehensile and forceful actions could never be worn if it was any bigger than the human counterpart it is replacing. The size affects nearly every component of the prosthetic system. Several components already come in very small sizes such as micro-controllers and processors or rather simple sensory interfaces, others such as the valves to control the airflow can not be miniaturized so easily. The small valves will feature lower air flow compared to their bigger counterparts as it will be shown section \ref{valve}. The size constraint further affects the type of control of the valve so that the smaller ones mostly only provide on-off instead of proportional flow control. 

For the hand itself, we can use anthropometric data as it has been gained by the U.S. military (see Figure \ref{hand-anthropometric}). This constraint on the hand is crucial to the prosthesis wearer because it is inconvenient to have a pair of hands of asymmetric size. For a tendon-driven system, this constraint can be easily fulfilled by separating actuation and the hand from each other and thereby making the hand smaller and more human-like.

The really challenging part with this constraint is the air pressure supply for the device. The size constraint reduces the size of the tank, which reduces the amount of air supplied to the system as the air pressure decreases linearly as air is used. So to provide a constant force output over many grasps with the hand, it is important to supply an approximately constant pressure, the tank has to be refilled frequently or it needs to provide a high pressure that is regulated down by a pressure regulator. The option of a very large tank would not meet the size constraint since it would make the system immobile. Therefore the constraint for the prototype can be described as using only as much space as needed to be mounted a human arm (device radius of approximately the human arm). The valves should be integrated as close to the hand as possible or in the worst case be fitted into a backpack together with the tank and a compressor if any. The actuators should further be fitted onto the prosthesis itself as they do not need too much space.

\subsection{Weight}

The weight of a prosthetic device is one of its major constraints. Defining how many components can be mounted onto the prosthesis, it is very important to the wearer of a prosthesis that the system is not extremely heavy even if this constrains the dexterity of the hand. The weight especially constrains the number of actuators of a traditional DC-motor driven system, since the motors are heavy compared to other components.  A pneumatic system's actuators are not considered heavy compared to the other components, because pneumatic actuators in general have a high force to weight ratio e.g. they can be lightweight and still meet the exerted force requirement. The weight of the pneumatic device constraints the number of valves and sensors a lot more than the actuators, also meaning that the lightweight valves will be constrained on their maximum pressure and flow. The air flow in particular determines the speed of the prosthesis' movement, the pressure defines its strength. The reduced strength should not constitute a problem since the actuators exert high forces already, but the speed of the prosthetic will have to be improved if it falls below a certain threshold. Moreover, a pneumatic device needs a pneumatic power source, being either a compressor or a tank to provide compressed air. The weight, combined with the previously presented size constraint, strongly constrains maximum pressure and flow of the compressor as well as the maximum pressure of the tank. As described in detail in section \ref{pressure-tank} about tanks, the weight of a tank increases drastically with the maximum pressure and therefore the maximum pressure of the system is also constrained by the weight.\\ 
The prosthestics community prescribes just a vague maximum weight for a prosthesis. Ultimately the weight of the device depends on the required size and dexterity of the hand. The lowest maximum weight is defined by Kay and Rakic \cite{Kay1972}, which set a requirement that the entire hand with its cosmetic glove should remain under 370g. Pons et al. \cite{Pons2004} defined the weight of an adult-sized prosthetic device to be less than 400g. Further groups such \cite{Light2000} and \cite{Vinet1995} defined a 500g weight limit to be appropriate. Therefore a new measure can be defined as $m_{min} < m_{device} < m_{max}$ where $m_{min}$ equals about 370g ( = Sensorhand (350g),\cite{Kay1972,Belter2011,Ottobock-Sensor1,Ottobock-Sensor2}) and $m_{max}$ equals about 500g ( = iLimb prosthesis (450g),\cite{Light2000,Vinet1995,Belter2011,RSLSteeperBebionic}). This measure is appropriate as the human hand without the extrinsic muscles weights about 400g \cite{Belter2011}. The minimum weight was defined here because the hand should not be too light as this can also cause irritations to the amputee. Luckily, in case of a prosthesis that is too lightweight, it is still possible to increase the weight by improving the dexterity of the system or by simply adding heavier materials to it that improve the overall stability of the device. The weight constraint currently excludes the tank and/or compressor since they would not be fitted onto the prosthesis. The defined constraint should be considered as being sufficient to reduce the risk of building an artificial hand that is massively heavier compared to the natural human hand.

\subsection{Noise}

The third constraint especially difficult to handle in pneumatic systems is the noise constraint. In servomotor driven prostheses, a problem with noise mainly occurs as soon as the system uses a higher amount of motors and thereby generates a lot of monotone noise. As shown in section \ref{abandonment}, the noise can be very displeasing to the user and can therefore lead to prosthesis abandonment. In a pneumatic system, most of the noise is generated by the compressor, but that can easily be solved by adding a tank as a reservoir to be able to run the prosthesis for a period of time with the need to repressurize the device by an integrated or external compressor. The other noise-generating component is the air valve. Switching solenoids are particularly noisy in on-off valves. On-Off valves are mostly driven by a pulse width modulation (PWM) signal, which is used to approximate a proportionally controlled air flow by opening the valve for a precalculated percentage of a defined time interval to approximate the area under the curve of a desired proportional signal in the same defined interval. Depending on the precision of the modulation, this results in a fast on-off switching of the solenoid to control the air flow and to reach the desired air pressure inside of the actuator. If the precision of the finger movement is too high, the approximation by PWM can get very noisy because very little amounts of air are provided to the muscle to continuously fill it. The noise can be reduced by several strategies when using PWM such as opening and closing the valves at lower frequencies and raising the frequency when near to the desired position or force. This strategy should reduce the switching noise by trying to reach the desired pressure with only a few switches. The disadvantage of the latter strategy could be lower precision of the position and force control of the muscle, which will show as soon as the strategy can be tested on a prototype. In case of yet unknown drawbacks, an optimum will have to be found for the corresponding application.\\
According to a study by Altinsoy et al \cite{Altinsoy2010} the noise does not seem to disturb the users, as long as it is unobtrusive. Several people also use the sound as a feedback of successful operation, velocity or force output. The study also presents options on how to design the sound to be more pleasing to the user instead of trying to avoid it completely. But as a good measure of sound pressure we should define a threshold of 40 dB that should not be exceeded. 40 dB is approximately the volume of a normal conversation, therefore this is a good limit to abide.

\subsection{Aesthetic properties}
 
In some cases, aesthetic properties can also be a constraint for the creation of a human hand prosthesis. For example, cosmetic constraints make most people not want to wear a prosthetic tentacle instead of a human hand (despite its ability to grasp several shapes due to an adaptive joint mechanism over eight joints (eight degrees of freedom with one degree of actuation \cite{TentacleProsthesis}). It is a problem that prostheses mostly still do not look like real human hands.


\begin{figure}[H]
\subcaptionbox{\label{fig:close-up-tentacle}}{\includegraphics[width=0.4\linewidth]{Misc/Tentacle-prosthesis1}}\hspace{0.15\textwidth}
\subcaptionbox{\label{fig:underactuated-mechanism}}{\includegraphics[width=0.4\linewidth]{Misc/Tentacle-prosthesis3}}\\
\subcaptionbox{\label{fig:internal-control}}{\includegraphics[width=0.4\linewidth]{Misc/Tentacle-prosthesis4}}\hspace{0.15\textwidth}
\subcaptionbox{\label{fig:grasping-patterns}}{\includegraphics[width=0.4\linewidth]{Misc/Tentacle-prosthesis5}}
\caption[Tentacle prosthesis]{Tentacle prosthesis which a highly underactuated grasping mechanism but a morphology that allows easy adaption to several objects.}
\label{tentacle-prosthesis}
\end{figure}

There are several patients that choose a cosmetic non-functional prosthesis for one occasion (such as swimming or public events) and a myo-electric for other occasions where a functional hand is needed \cite{Crandall2002,Leow2001,Trost1983,Datta1989,Mendez1985}. The constraint of aesthetics leads to similar effects as the environment of usage does, resulting in abandonment of the prosthesis because it does not give the user a natural look when it is needed. The here defined soft-constraint should in fact state that the prosthesis should look as similar as possible to the other hand of an unilateral amputee or in general as human-like as possible. This can be considered separately by the designer as soon as the prosthetic hand is finished and the hand can be covered by a realistically looking silicone glove or similar. The constraint mainly states that the hand should stay in the boundaries of thickness of the fingers, palm and wrist and should feature five fingers including an opposable thumb.

\subsection{Environment of usage}

Another constraint mostly forgotten by the designers and therefore a major cause of prosthesis abandonment is the environment in which the prosthesis can be used \cite{abandonment1,abandonment2}. If the wearer likes to climb, wants to go to the Arctic or loves to swim, the prosthetic device should be able to accompany him. Some wearers prefer to have several prostheses, for instance a myo-electric device for normal daily living and for occasional tasks that do not allow electronics, they have a cosmetic version of their hand. However, if the prosthesis has only a very limited environment in which it is fully functioning or comfortable to wear, the wearer might abandon it or never wear it at all. This could be for instance the case with myo-electric devices that do not support utilisation in water or prostheses that have a metal surface and therefore heat up in the hot sun or cool down quickly in cold environments. This constraint might be the most complicated to fulfil, in some cases it can not be considered at all or just to a limited extent because some components only allow application in certain environments. This constraint also comes in many hidden ways and can possibly only be completely uncovered by conducting field studies that test the prostheses with several wearers.

\subsection{In practice}
In practice, all of the constraints have to be combined because they have to be met at the same time. Combining the weight, size and noise constraints leads to a much more complicated problem that needs to be optimized for a special application than just one constraint. A pneumatic actuator is very lightweight and can be easily miniaturized (good for the size constraint), but to use smaller pneumatic actuators, the pressure has to be increased. This is difficult because of the size and weight constraint because it would require a bigger tank or one with a higher pressure. Therefore it is better to calculate enough space for the pneumatic actuators and make them with the biggest diameter possible to reach the highest output force. Still the bigger diameters will use more air for a contraction which will again affect the size of the tank. It is now apparent why the creation of a pneumatic control system for prosthetic hands is a very difficult task.

\section{Analysis of Requirements}
The requirements, different from the constraints but not unrelated, define what we expect from the prosthetic actuation system. Based on the user's expectation and nature's human hand, we want to mimic the performances of the human hand as well as possible to reenable the prosthesis wearer to manipulate its environment as hand-alike as possible. Listed here are the requirements that are particularly important to the actuation system and each requirement is discussed in detail.

\subsection{Speed}

The speed of the human hand can be remarkable depending on the action. One important prerequisite to achieve higher speeds while having a low impact on the object manipulated is that the fingers have a very low weight and added compliance similar to the human skin. The latter can be achieved by the use of a soft glove covering the fingers. Pneumatic actuator systems speed of action is basically dependent on the air flow ($\frac{L}{min}$) and the actuator's volume needed to move. The valve is often the flow limiting factor, all the other components such as tubes and actuators can be replaced more easily because they are less air flow sensitive components. Most hand prostheses reach a maximum speed of about 1 Hz for a full finger flexion with the only exception of the Bebionic v.2 (1.1 Hz) which is slightly faster. Robotic hands can reach flexion speeds up to 10Hz and higher (such as the M.I.T. Dextrous Hand which is also a pneumatic system), but this speed can not be compared to prostheses, since the Dextrous Hand does not meet several constraints that are applied to prostheses (weight, size etc.). Moreover, it is important to note that the high finger speed is useless if it can not be accurately controlled by the prosthesis user.



\begin{table}[H]
\begin{adjustwidth}{-3em}{-3em}
\scriptsize 
\begin{tabular}{p{8cm}|p{5.5cm}|l}
\toprule
Name of the paper & Original Finger speed as found in paper & Finger Speed in Hz\\ 
\midrule
	Goebl et al. (2009), Finger motion in piano performance Touch \& Tempo \cite{Goebl2009} &  0.6 - 0.7 $\frac{m}{s}$ / 24 $\frac{tones}{s}$ & 12 Hz\\
	Aoki et al. (2008),Differences in the abilities of individual fingers during the performance of fast repetitive tapping movements \cite{Aoki2008} & Asc v ($\frac{cm}{s}$): Index:32 Middle: 30 Ring: 18 Little 21 \newline Desc v($\frac{cm}{s}$: Index:36 Middle: 34 Ring: 19 Little 25 & $\leq$ 6Hz \\
	Bridge et al. (1913),Development of speed in repetitive \& successive finger-movements in normal children \cite{BridgeDenckla1913} & Children age 7: $\frac{40cm}{6s}$ = 6.6 $\frac{cm}{s}$ & 6.6 Hz \\
	Wessberg et al. (1996),Pulsatile motor output in human finger movement is not dependent on the stretch reflex \cite{Wessberg1996} & 180 $\frac{deg}{s}$ with movement of 20 deg & 9.5 Hz\\
	Yokoe et al. (2008), Opening velocity, a novel parameter for finger tapping test in patients with parkinson disease \cite{Yokoe2008} & 0.6-0.8 $\frac{m}{s}$ with ampl 4cm & 10Hz \\
	Rodriguez et al. (2009),Rapid slowing of maximal finger movement rate: fatigue of central motor control? \cite{Rodrigues2009} & Asc v: 890 $\frac{deg}{s}$ \newline Desc v:1258 $\frac{deg}{s}$ with ampl. 55 deg & 5.5 Hz  \\
	Jobbagy et al. (2005), Analysis of finger tapping movement \cite{Jobbagy2005} & index: 41.3 $\frac{cm}{s}$ ampl 7.75cm \newline Middle/Ring: 32 $\frac{cm}{s}$ with ampl 6cm & 5.3Hz \\
\bottomrule
\end{tabular}
\caption[Finger tapping speed]{Table showing the results of different studies describing the finger tapping speed performance of humans.}
\label{finger-tapping}
\normalsize
\end{adjustwidth}
\end{table}
A comparison of this speed to the finger speed measured in humans shows that this is far above the average speed and  outpowers nearly all findings of the maximum speed of human fingers. Referring to several studies, the finger tapping speed varies from 5.3 Hz \cite{Jobbagy2005} up to 10 Hz \cite{Yokoe2008}. For the exact numbers, please refer to the table \ref{finger-tapping}.

Based on these numbers, a threshold of 1 Hz should be set to see whether the prosthesis can compete with the other devices. A further threshold could be set at 10 Hz which would be a secondary goal to see whether it could compete with several robotic hands.

\subsection{Strength}

Hand strength is another very important requirement for the prosthesis. Without strength, a prosthesis is entirely useless and degrades to a very unpleasant hand replacement. The precision and speed can not be useful without being able to lift anything and the hand should be able to interact with objects of at least ''handsize''. To exert force, there are several possibilities: either the prosthesis has strong actuators that continuously exert force or the actuators are less strong but there exists a locking mechanism to lock the current position of the hand and only release it on unlock. In myoelectric prostheses, several locking mechanisms were tried \cite{locking1}, but their disadvantages are mostly caused by unlocking too easily or not unlocking at all, both cases being a security issue for the amputee. The first approach can be used in pneumatic actuators. The whole actuation mechanism of pneumatic actuators is based on emitting air by opening the valves, thus changing the pressure inside the actuator and thereby its position. By closing valves air is locked inside resulting in a different kind of locking mechanism while providing some air compressibility based compliance. This mechanism does not need any energy while keeping a position, which is a very desirable property. Pneumatic actuators further have a high force to weight ratio of up to 1800:1 as will be shown in the experiments, giving them excessive force to actuate a prosthesis. The only drawback of pneumatic actuators is that their position and force is rather difficult to control compared to servo motors, therefore a proper tuning of the control has to be done to avoid positional oscillations or rather slow positioning of the fingers.

To properly define a measure for the strength of the prosthetic hand, again a literature review was done on the strength of the human hand. The maximum strength of a single hand grip has been found to be $760.65 N$. Further the strength of an index finger hook at the tip is found to be about 117 N. Assuming that all fingers would have the same output strength (even though the ring and little finger are not as strong as the index and middle finger), a finger hook with all fingers would result to the output force of $5 \cdot 117 N = 585 N$.  According to the listing of prosthetic and robotic hands the Bebionic v.2 hand prosthesis has an output force of about $900 N$.

\scriptsize
\setlength{\LTleft}{-30pt}%
\setlength{\LTright}{\LTleft}
\begin{longtable}{p{5.2cm}|p{4.2cm}|p{3.7cm}|p{1.7cm}}
\caption[Listing of hand strength studies]{Table describing the finger strength performance of humans. $\mu = Mean$.}\\
\toprule
Name of the paper & Original hand strength as found in paper & Hand strength(N) & Type of strength\\ 
\midrule
\label{hand-strength}
\endfirsthead

\multicolumn{4}{p{8cm}}
{{\bfseries \tablename\ \thetable{} -- continued from previous page}}\\ 
\hline
Name of the paper & Original hand strength as found in paper & Hand strength(N) & Type of strength\\ 
\midrule
\endhead

\hline \multicolumn{4}{r}{{Continued on next page}} \\ \hline 
\endfoot
\endlastfoot

Mathiowetz et al. (1985), Grip and Pinch Strength: Normative Data for Adults \cite{Mathiowetz1985} & Age: 25-29 [Min:91/$\mu$:121/Max:167] lb  \newline
Age: 30-34 [Min:70/$\mu$:121.8/Max:170] lb  & 742.85-760.65 N & Grip strength\\

Mathiowetz et al. (1985), Grip and Pinch Strength: Normative Data for Adults \cite{Mathiowetz1985} & Age: 20-24 [Min:12/Mean:17.5/Max:36] lb & 77-160 N & Pinch tip strength\\

Wells et al. (2001), Characterizing human hand prehensile strength by force and moment wrench \cite{Wells2001} & Index finger hook: 117 N & 117 N & Index finger hook\\
	
Swanson et al. (1970), The strength of the hand \cite{Swanson1970} & Average Grip: 47.6 kg \newline
Age 30-40 [Min:44.5/Max:49.2]kg & 466.96-482.65 N & Hand strength\\

Didomenico et al. (2003),Measurement and prediction of single and multi-digit finger strength \cite{Didomenico2003} & 
	Poke: $\mu:45.95 N /Max: \mu + 3 \cdot 17.80 N = 99.35 N$\newline 
Press: $\mu:43.05 N /Max: \mu + 3 \cdot 18.43 N = 98.34 N$ \newline
Pull: $\mu:60.09 N /Max: \mu + 3 \cdot 25.24 N = 135.81 N$\newline
Lateral: $\mu:80.93 N /Max: \mu + 3 \cdot 28.15 N = 165.38 N$ \newline
Chuck: $\mu:79.75 N /Max: \mu + 3 \cdot 28.96 N = 166.63 N$ \newline
Palmar: $\mu:54.16 N /Max: \mu + 3 \cdot 18.84 N = 110.68 N$ \newline
Grip: $\mu:370.67 N /Max: \mu + 3 \cdot 117.73 N = 723.86 N$ \newline& 
Poke: 99.35 N \newline
Press: 98.34 N \newline
Pull: 135.81 N \newline
Lateral: 165.38 N \newline
Chuck: 166.63 N \newline
Palmar: 110.68 N \newline
Grip: 723.86 N \newline & Various\\

Li et al. (2000),Characteristics of finger force production during one- and two-hand tasks \cite{Li2000} & Index :45N / Middle: 37.6N \newline Ring: 25.7N / Little: 25.5N & Index :45N / Middle: 37.6N \newline Ring: 25.7N / Little: 25.5N & Individual finger strength\\

Radwin et al (1992),External finger forces in submaximal five-finger static pinch prehension \cite{Radwin1992} & Index :61N / Middle: 58N \newline Ring: 36N / Little: 28N & Index :61N / Middle: 58N \newline Ring: 36N / Little: 28N & Individual finger strength\\

Dickson et al (1972), A device for measuring the force of the digits of the hand \cite{Dickson1972} & Index :45N / Middle: 43N \newline Ring: 31N / Little: 27N & Index :45N / Middle: 43N \newline Ring: 31N / Little: 27N & Individual finger strength\\

MacDermid et al (2004), Individual finger strength: Are the ulnar digits ''powerful''? \cite{MaxDermid2004} & Index: 25-26N \newline Middle: 35-37N \newline Ring: 24-27N \newline Small:11-15N & Index: 25-26N \newline Middle: 35-37N \newline Ring: 24-27N \newline Small:11-15N & Individual finger strength\\

Zong-Ming (2002),The influence of wrist position on individual finger forces during forceful grip \cite{Zong2002} & Index: 29.6-42.6N \newline Middle: 24.3N-31.4N \newline Ring: 19.7-28.4N \newline Little: 10.3-15.7N & Index: 29.6-42.6N \newline Middle: 24.3N-31.4N \newline Ring: 19.7-28.4N \newline Little: 10.3-15.7N & Individual finger forces\\
\bottomrule
\end{longtable}
\normalsize

\pagebreak
The prototype should therefore at least have an output force of $585 N$, the next higher output force would be $760 N$ and to try to compete with the Bebionic v.2 hand it should have an output force of $900 N$. A further possibility would also be to work on a more bio-inspired version that maximizes the contact area between the hand and the object to reduce the needed force to manipulate objects \cite{Kargov2004}. The the unequal force distribution between current prosthesis and the human hand is visualized in the figure \ref{force-requirement}.

Further can be seen in table \ref{force-distribution} below, that the grasp force in the human hand is much lower than the grasp force in the prosthetic limbs to achieve the same stability of grasp (Sum of forces). Only the adaptive prosthesis features similar results as the human hand.

\begin{figure}[H]
\centering
\includegraphics[width=0.5\textwidth]{Requirements/prosthesis-force}
\caption[Force distributions in different hands]{Force distributions in different hands:\\
Top-left: Human hand,
Top-right: Adaptive prosthesis\\
Bottom-left: Sensor-hand,
Bottom-right: System-electro-hand\\
Image from \cite{Kargov2004}.}
\label{force-requirement}
\end{figure}


\begin{table}[H]
\scriptsize
\begin{adjustwidth}{-1.5em}{-1.5em}
\begin{tabular}{lcccc}
\toprule
Name & Average force [N] & Maximum force [N] & Sum of forces [N] & Force at fingertips [N]\\
\midrule

Human hand & 0.8 (0.7) & 3.8 & 16.7 & 6.3 \\
Adaptive prosthesis & 1.3 (0.4) & 4.7 & 21.3 & 9.9\\
System-electro-hand$^{TM}$ & 2.6 (2.7) & 13.8 & 28.5 & 17.3\\
Sensor-hand$^{TM}$ & 3.9 (4.6) & 24.7 & 47.4 & 24.9\\
\bottomrule
\end{tabular}
\caption[Force characteristics of different hand types]{Force characteristics of different hand types [and standard deviation]. Data from \cite{Kargov2004}}
\label{force-distribution}
\normalsize
\end{adjustwidth}
\end{table}

The another guideline would therefore be that we need to aim for an adaptive design in our hand to achieve a good force distribution.

\subsection{Reliability}
Even though the reliability does not have to be as high in a prototype as it should be in a finished prosthesis, it should after a certain time reach a point at which it performs its task in a reliable manner. This means that initially it would be acceptable for the prototype to show the desired action just in certain cases and later is improved to perform it in a reliable manner. For a prosthesis later on, it is very important that all components work reliably and stably to ensure that the device properly replaces the missing limb. Otherwise it might be abandoned by the user because it complicates manipulation instead of increasing comfort for the amputee. Special attention should be paid to the durability of the pneumatic actuators and to their precise control, otherwise the prosthesis will not work for long or its performance could easily be mistaken for a malfunction in the case of improper control. The latter should be avoided at all costs by rather reducing the speed or force of the prosthesis and controlling it reliably than having more of either one and not being able to control it securely.

The reliability in terms of the pneumatic actuation system means that its control works stably for an extended time so that the amputee can depend on it as much as a healthy person can rely on its healthy hand. In a study of F.J. Trost (1983) it has been found that the conventional prosthesis needs to be repaired 1.9 times per year on average \cite{Trost1983}. To compete with a conventional prosthesis, it will therefore have to successfully work for at least a third of a year before anything should have to be replaced. 
\subsection{Security}
A very important requirement mostly ignored is security. It is mostly not mentioned because there are less security issues for the user with electronics than with pneumatics. Pressure generating and consuming devices can be manufactured to be working securely and safely, but one has to pay attention to the security issues that could arise and their consequences. Air especially, being a medium that can be highly compressed compared for instance to water which is nearly incompressible, can cause severe injuries because of its emitted expansion force when released. In case of breakage of tanks, tubes, or valves, the air expands through the leakage which can cause damage similar to an explosion. This issue can be dealt with by carefully checking that the pneumatic components are matching the pressure applied and the user is protected by security measures i.e. pressure cut off in case a leakage is detected and so on. Also a double wall at several pneumatic components can help to prevent pieces of a broken component from harmingthe user.

\subsection{Fast maintenance by modularity}
Most myo-electric prostheses of today still suffer from higher maintenance costs and a difficult replacement of components \cite{Scotland1983,Trost1983,Datta1989,Ballance1989,Glynn1986,Weaver1988}. Only a few myo-electric prostheses such as the Vincent hand \cite{Schulz2011}, the Smart hand \cite{TheSmartHand2011} or the Bebionic hand \cite{Medynski2011} try to reduce the time for maintenance by making the prosthesis modular so that a defective module can easily be replaced. The prosthesis user is then allowed to leave with his prosthesis after it has been assured that the new module works properly and the clinician can repair the module by himself or send it to the manufacturer for repair.

For a tendon-driven pneumatic prosthesis, the modularity should not be a very difficult task. If a tendon breaks, it is easy to replace, the same counts for the actuator and the valve. For the prototype, high modularity is not yet important, since the repair is relatively easy to do.

\subsection{Power efficiency}
Several studies have shown that myo-electrically driven prostheses are mostly worn for longer than 8 hours a day \cite{Fraser1998,Crandall2002,Silcox1993,Northmore1980,Hubbard1997,Kyberd1993,Millstein1986,Datta1989,Weaver1988,Keijlaa1993,Scotland1983}. Considering that and the limited energy provided by the prosthesis battery, the prosthesis has to be energy efficient using low-power components on the one hand and limitations of high-power components to short periods. Also the use of compressed air should be reduced as much as possible to make efficient use of it. It should be noted that a lot of energy can be saved by making the hand able to keep its current state without using energy. In a pneumatic system with no leakage, closing the valves keeps the actuators at their current force output and stiffness which allows to keep a strong grip for an extended time without using electrical energy or compressed air. Only the modification of the position of the hand uses electrical energy while only either closing or opening the hand uses compressed air depending on the configuration. This is a clear advantage over motor-driven systems that can not save energy while keeping a defined position. 

\subsection{Cost}
Another cause of prosthesis abandonment is simply the price of a myoelectric prosthesis. Already in 1988 when Weaver et al. conducted a study to compare the myoelectric prosthesis with the body powered ones, they experienced a price gap of 6500 USD vs. 1500 USD \cite{Weaver1988}. Today, there is still not much difference.
The Bebionic v.2 for instance costs about 11'000 USD \cite{wwwrslsteeper} and the iLimb pulse device costs about 17'000 USD \cite{wwwtouchbionics}. Some amputees can not afford a myoelectric prosthesis because of its high price, in other cases it can be the maintenance costs that make the device consume a high amount of money. Furthermore a low price can improve the probability of the prototype of the prosthesis to become a research platform, which in the end could lead to a better prosthesis for the consumer. This is why it is important to make the system as modular as possible and build it from already highly available components instead of manufacturing several highly specialized components that then never enter the market. Therefore it must be a requirement for the design to keep costs low under all circumstances or to clearly reason why the component is necessary.\\

The Requirements have been written based on the requirement and constraint analyses of \cite{Jacobsen1985} and \cite{Misra2010}.

\subsection{In practice}
While some of the requirements listed here are considered easier to fulfill for pneumatic systems such as strength and speed, requirements such as power efficiency and security need more attention. Since the budget of most research projects are fairly limited, the initial costs of the prototype will also come out relatively low, therefore being a self-fulfilling requirement. The reliability and the maintenance by modularity will be considered at a later time as soon as the prototype shows the other desired features and the optimization for daily use can begin. This thesis will just reach a prototype that provides a proof of concept for a pneumatic actuation system, which is actually the main task of this project.
%----------------------------------------------------------------------------------------

\section{Definition of grasp patterns}
\label{sec:grasp-patterns}
Although the hand with its 21 degrees of freedom has a seemingly infinite space of movements and positions, the usage of a hand for actions of daily living shows several more common ways of movement to do a certain task. Nobody holds a pen in between its abducted middle and ring finger even though it would be possible to do so. Most people use a pen in a writing position with the pen between the index and middle finger and the thumb, but especially little children sometimes hold a pen by wrapping all the fingers around it. This shows there seem to exist several useful so called ''grasp patterns'' as well as a lot of non-useful patterns. Some of the grasp patterns are more useful for tasks that need more force and less precision and some are more useful if more precision than force is needed.

To reduce the complexity of control of a prosthetic hand and thereby attacking the enigma of the natural hand, several approaches were made to categorize the grasp patterns and describe their basic functionality and their most useful applications. One of the first works that contained an approach to create a taxonomy to classify hands was in Schlesinger et al. ''Der mechanische Aufbau der kuenstlichen Glieder''\cite{Schlesinger}. Schlesinger classified the grasping of the hand according to and objects shape and size into cylindrical grasp, precision grasp, hook prehension, tip grasp, spherical grasp, and lateral hip.

\begin{figure}[htp]
\centering
\includegraphics[width=0.6\textwidth]{Patterns/Schlesinger-primitives}
\caption[Basic grasp primitives]{Basic grasp primitives according to Schlesinger: cylindrical grasp, precision grasp, hook prehension, tip grasp, spherical grasp and lateral hip. Image from \cite{Schlesinger}.}
\label{schlesinger-primitives}
\end{figure}

This type of classification was later stated to be able to cover 90\% of all common dexterous actions with just these six basic grasping patterns \cite{Schlesinger}. The drawback of this taxonomy is that some of its names associate the grasping pattern with a certain object even though for instance a cylinder can also be held with the so defined precision grasp. Another approach of creating a taxonomy was then published by Napier in 1956 \cite{Napier1956}. This taxonomy introduced a functional classification of grasp separating the patterns by the general action it is associated with. The terms precision grasp and power grasp mean that in this grasp, the power resp. precision component is more important than the other. Iberall et al. in ''Human Prehension and Dexterous Robot Hands'' then further chose a classification that reflected the hand's posture which corresponds to the hook grasp and the cylindrical grasp \cite{Iberall1997}. Still the remaining problem is the versatility of the thumb. Compared to other primates' hands, the thumb of the human hand has the highest ability to directly oppose the other fingers. The thumb plays an important role in reshaping the grasp so that in opposition to the other fingers, precision and power grasps are available. The non-opposing thumb then again enables the lateral and hook grasp \cite{Kyberd2003}. From a study conducted to classify all the actions of daily living, it has been found a person uses about 50\% power grasps, 30\% precision grasp and 20\% lateral grasps \cite{Sollerman1995}.

Even if the classification is not yet perfect, today's modern commercial prostheses implement such grasping patterns that are accessible to the user. The grasp patterns are chosen by several mechanisms such as a switching signal caused by co-contraction of the muscles of the amputee that are linked to the EMG sensors or a pulse train of multiple EMG impulses in a short time \cite{TouchBionics2013}. The best solution is found in the rehabilitation training with the individual user.

\subsection{Power grasps}

This section describes a selection of the most popular power grasps described in literature. They are mainly used to carry heavy objects and hold things not firmly but stably in one's hands to prevent slipping or dropping. Optimizing maximum force output and high contact area is the grasp's main task.

\paragraph{Cylindrical grasp}
The cylindrical power grasp is especially useful to apply high amounts of force onto shapes similar to cylinders. The thumb opposed to the other fingers, it is suitable for objects of several diameters.

\begin{figure}[H]
\subcaptionbox{\label{fig:small-power}}{\includegraphics[width=0.2\linewidth]{Patterns/Small-power}}\hspace{0.15\textwidth}
\subcaptionbox{\label{fig:medium-power}}{\includegraphics[width=0.2\linewidth]{Patterns/Medium-power}}\hspace{0.15\textwidth}
\subcaptionbox{\label{fig:large-power}}{\includegraphics[width=0.2\linewidth]{Patterns/Large-power}}\\
\label{cylindrical grasp}
\end{figure}

\paragraph{Disk grasp}
The disk power grasp enables to grab disks to hold it tightly or apply rotary force to it. Together with the cylindrical power grasp of the other hand, it enables to open caps of jars for instance.

\begin{figure}[H]
\subcaptionbox{\label{fig:power-disk}}{\includegraphics[width=0.2\linewidth]{Patterns/Power-disk}}
\label{Power-disk-grasp}
\end{figure}

\paragraph{Sphere grasp}
Similarly looking as the disk power grasp, the sphere power grasps in a circular manner but describes a spherical shape based on the metacarpophalangeal (MCP) as well as the interphalangeal (DIP;PIP) joints. The grasp works from small diameters up to larger ones where it finally ends up as an open palm.

\begin{figure}[H]
\subcaptionbox{\label{fig:power-sphere}}{\includegraphics[width=0.2\linewidth]{Patterns/Power-sphere}}
\label{Power-sphere-grasp}
\end{figure}

\paragraph{Hook grasp}
A particularly strong grasp is the hook grasp. It comes in various strong variances, which all basically form a hook with all the fingers simultaneously to surround a handle or hang something into it. The grasp can be strengthened by closing the hook with the thumb and therefore increasing the frictional force on the handle being grabbed.

\begin{figure}[H]
\subcaptionbox{\label{fig:open-hook}}{\includegraphics[width=0.2\linewidth]{Patterns/Open-hook}}
\label{Power-hook-grasp}
\end{figure}

\subsection{Precision grasps}

\paragraph{Prismatic grasp}
The prismatic grasp comes in 3 variations either using two, three or four fingers and the thumb. Providing a precision grasp of increasing stability, the force is entirely exerted from the finger pads.

\begin{figure}[H]
\subcaptionbox{\label{fig:prismatic-2finger}}{\includegraphics[width=0.2\linewidth]{Patterns/Prismatic-2finger}}\hspace{0.15\textwidth}
\subcaptionbox{\label{fig:prismatic-3finger}}{\includegraphics[width=0.2\linewidth]{Patterns/Prismatic-3finger}}\hspace{0.15\textwidth}
\subcaptionbox{\label{fig:prismatic-4finger}}{\includegraphics[width=0.2\linewidth]{Patterns/Prismatic-4finger}}\\
\label{Prismatic-grasp}
\end{figure}

\paragraph{Palmar Pinch}
The palmar pinch, using only the index finger and the thumb, enables to grasp very small or thin objects that require only little force to manipulate.

\begin{figure}[H]
\subcaptionbox{\label{fig:palmar-pinch}}{\includegraphics[width=0.2\linewidth]{Patterns/Palmar-pinch}}
\label{Palmar-pinch}
\end{figure}

\paragraph{Disk grasp}
The disk precision grasp is the prehensile version of the two disk grasps. It does not apply force from the palm but holds the disk only with the pads of the finger, thus enabling to lift and rotate firmly.

\begin{figure}[H]
\subcaptionbox{\label{fig:precision-disk}}{\includegraphics[width=0.2\linewidth]{Patterns/Precision-disk}}
\label{Precision-disk-grasp}
\end{figure}

\paragraph{Sphere grasp}
The sphere precision grasp handles spherical objects only with the fingers instead of the whole palm, as the sphere power grasp does. It enables to manipulate spheres in the abductors/adductors of the fingers only.

\begin{figure}[H]
\subcaptionbox{\label{fig:precision-sphere}}{\includegraphics[width=0.2\linewidth]{Patterns/Precision-sphere}}
\label{Precision-sphere-grasp}
\end{figure}

\paragraph{Tripod}
The tripod grasp, being the most important of the precision grasps, come in many variations because of its wide range of applications. From grabbing spherical objects to writing, it holds objects firmly between the index and middle finger and the thumb, thereby forming a tripod.

\begin{figure}[H]
\subcaptionbox{\label{fig:tripod}}{\includegraphics[width=0.2\linewidth]{Patterns/Tripod}}\hspace{0.15\textwidth}
\subcaptionbox{\label{fig:sticks-tripod}}{\includegraphics[width=0.2\linewidth]{Patterns/Sticks-tripod}}\hspace{0.15\textwidth}
\subcaptionbox{\label{fig:writing-tripod}}{\includegraphics[width=0.2\linewidth]{Patterns/Writing-tripod}}\\
\label{Tripod-grasp}
\end{figure}

\paragraph{Quadpod}
The quadpod is the basic tripod grasp extended to hold with the ring finger as well. Extending it further to using the little finger would result in a sphere precision grasp.

\begin{figure}[H]
\subcaptionbox{\label{fig:quadpod}}{\includegraphics[width=0.2\linewidth]{Patterns/Quadpod}}
\label{Quadpod-grasp}
\end{figure}

\subsection{Lateral grasp}
The lateral grasp, being exceptional by using the side of the index finger as a force exerting surface opposing the thumb, enables to hold spoons and forks, further handling of flat objects or wallets, the latter giving it also the name wallet pinch.

\begin{figure}[H]
\subcaptionbox{\label{fig:lateral-pinch}}{\includegraphics[width=0.2\linewidth]{Patterns/lateral-pinch}}\hspace{0.15\textwidth}
\subcaptionbox{\label{fig:lateral-pinch-newspaper}}{\includegraphics[width=0.2\linewidth]{Patterns/lateral-pinch-newspaper}}\hspace{0.15\textwidth}
\subcaptionbox{\label{fig:lateral-pinch-tray}}{\includegraphics[width=0.2\linewidth]{Patterns/lateral-pinch-tray}}\\
\label{Lateral-grasp}
\end{figure}

\paragraph{Lateral tripod}
The lateral tripod grasps the objects between the lateral side of the middle finger and the tips of the index finger and the thumb. Using the middle finger to stabilize and strengthen the palmar pinch, this grasp forms an intermediate grasp between precision and power.

\begin{figure}[H]
\subcaptionbox{\label{fig:lateral-tripod}}{\includegraphics[width=0.2\linewidth]{Patterns/Lateral-tripod}}
\label{Lateral-tripod-grasp}
\end{figure}

\subsection{Various}

\paragraph{Finger abduction/adduction}
Finger abduction and adduction is not a special grasp, even though it is possible to hold thin objects like paper with it. It basically helps in adapting the grasp to the object by increasing the contact surface between the hand and the object.

\begin{figure}[H]
\subcaptionbox{\label{fig:adduction}}{\includegraphics[width=0.2\linewidth]{Patterns/Adduction}}\hspace{0.15\textwidth}
\subcaptionbox{\label{fig:finger-adduction}}{\includegraphics[width=0.2\linewidth]{Patterns/finger-adduction}}\hspace{0.15\textwidth}
\subcaptionbox{\label{fig:finger-adduction-cards}}{\includegraphics[width=0.2\linewidth]{Patterns/finger-adduction-cards}}\\
\label{Adduction-grasp}
\end{figure}

\paragraph{Open palm}
The open palm can be used for various actions, even though it is technically not a grasp. Using the flat palm, the hand can be used to push objects, to collect small objects inside or to carry larger objects with a flat surface on it. This is why several myo-electric prosthetic hands implement this as an available gesture as well.

\begin{figure}[H]
\subcaptionbox{\label{fig:open-hand-plate}}{\includegraphics[width=0.2\linewidth]{Patterns/Open-hand-plate}}\hspace{0.15\textwidth}
\subcaptionbox{\label{fig:open-hand-ball}}{\includegraphics[width=0.2\linewidth]{Patterns/Open-hand-ball}}\hspace{0.15\textwidth}
\subcaptionbox{\label{fig:Open-hand-press-button}}{\includegraphics[width=0.2\linewidth]{Patterns/Open-hand-press-button}}\\
\label{Open-palm}
\end{figure}

\paragraph{Index point}
The index point is not a grasp either, but is very useful for pointing at things in the surrounding of the person. Further it can be used to press buttons and use switches.

%----------------------------------------------------------------------------------------
\end{document}