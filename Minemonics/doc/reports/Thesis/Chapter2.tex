% Chapter 2
\documentclass[main]{subfiles}
\setcounter{chapter}{1}

\begin{document}
\chapter{Prosthetics - An ongoing history} % Main chapter title

\label{Chapter2} % For referencing the chapter elsewhere, use \ref{Chapter1} 

\lhead{Chapter 2. \emph{Prosthetics - An ongoing history}} % This is for the header on each page - perhaps a shortened title 

%----------------------------------------------------------------------------------------
% Discussion of prosthesis abandonment
% Presentation of different types of prostheses
% Listing of prostheses and important properties
%Prosthetics - A history
%----------------------------------------------------------------------------------------

\section{Types of prostheses}

People with prostheses should have the possibility to do everything a person without limb-loss can do. With this vision in mind, the evolution of prosthetics has come a long way from its primitive beginnings to its sophisticated present. Since prosthetic users ask for a life as normal as possible, several different types of prostheses have arised on the market, nevertheless they represent different stages of complexity. Historically, some of them were the only ones available, today they can be seen as an individual choice depending on what suits the user the most or what he or she wants to do with it. Some of them are suitable for hiking and others more fitting for a gala. Or as it has been described in \cite{Pfeifer2006}, every device has its ecological niche, where it is most effective and usable, the same way animals have ecological niches. Up to date, all of them can be put into three categories as described in the following passages.

\subsection{Passive prostheses}
The category of passive prostheses was the first category that arose in human history. From evidence of saws built from stone and bone and what appears to be amputated bone stumps in skeleton, it is believed that amputations were performed in the Neolithic times. The first recorded instance of amputations and prosthetic replacement appears in the book of the Vedas, written in Sanskrit in India. Rig-Veda, the oldest of the Vedas, is a book which is believed to have been written between 3500 and 1800 B.C. It records that the leg of Queen Vishpla was amputated in battle. After the wound had been healed to a high extent, an iron leg was fitted to enable Queen Vishpla to walk and to return to the battlefield \cite{history-of-amputations}.

It has been discovered on a mummified corpse of an egyptian woman that already the old Egyptians replaced limbs with realistically looking counterparts to make the loss less obvious as shown in Fig. \ref{first-prosthesis} \cite{first-prosthesis,amputee-coalition}.

\begin{figure}[htp]
\centering
\includegraphics[width=0.5\textwidth]{History/egypt-toe}
\caption[Egyptian toe prosthesis]{An egyptian toe prosthesis to reenable walking and for aesthetic purposes. Image from \cite{first-prosthesis,amputee-coalition}.}
\label{first-prosthesis}
\end{figure}

The passive prostheses of today are made of rubber or skin-mimetic plastic and have no functionality at all but are a visually appealing replacement for the missing limb. They have a very large ecological niche, because they are very robust and light weight, and are very usable as long as the user does not need a functional hand.

\subsection{Body-powered prostheses}

One of the first arm prostheses was documented by the Roman scholar Pliny the Elder (23--79 A.D.) who wrote of a Roman General in the Second Punic War (218--210 B.C.) whose right arm had been amputated. His iron hand was capable of holding a shield and he could return to battle\cite{amputee-coalition}.

Other prostheses were developed in the dark ages (476--1000 A.D.) but the systems did not advance much. The biggest advancement was the hand hook, which was the first functional upper-limb body-powered limb. But usually only knights were fitted with such devices to hold a shield with little attention to other functionality. Only wealthy people could afford to be fitted with a hand hook for daily use. Later in time, tradesmen, especially armorers and watchmakers started to design and construct artificial limbs. The limbs started to contain gears and springs to implement intricate internal functions.

\begin{figure}[htp]
\subcaptionbox[Multifunctional hand]{A creative multifunctional hand replacement providing many switchable tools to the user. Image from \cite{amputee-coalition}.\label{switch-hand}}{\includegraphics[width=0.4\linewidth]{History/switch-hand}}
\hspace{0.15\textwidth}
\subcaptionbox[Iron hand]{An iron hand as it has been used in the dark ages. Its fingers can be closed by a mechanism that is controlled with the other hand. Image from \cite{first-prosthesis,amputee-coalition}.\label{iron-hand}}{\includegraphics[width=0.4\textwidth]{History/iron-hand}}
\end{figure}

\begin{figure}[htp]
\subcaptionbox[Ancient passive hand]{An ancient passive hand to hide the loss of the arm. Image from \cite{amputee-coalition}.\label{historic-hand}}{\includegraphics[width=0.4\textwidth]{History/historic-hand}}
\hspace{0.15\textwidth}
\subcaptionbox[Goetz iron hand]{A functional drawing of the famous Goetz iron hand. It could be controlled by the other hand and a mechanism with multiple springs and straps. Image from \cite{first-prosthesis,amputee-coalition}.\label{Goetz-eiserne-hand}}{\includegraphics[width=0.4\linewidth]{History/Goetz-eiserne-hand}}
\end{figure}

In the renaissance (1400--1800 A.D.), several new fields like art, philosophy, science and medicine arose. Several prostheses were built from copper, steel, iron, and wood based on the medical discoveries of the Greeks and Romans. Götz von Berlichingen, a German mercenary from around 1500, must be mentioned for his technologically advanced iron hands created to replace his right arm after he had lost it in the Battle of Landshut. It could be controlled by setting it with the other hand to the correct position with the aid of springs and leather straps. The hand that he has built is shown in Fig. \ref{Goetz-eiserne-hand}.

Around 1530, the French army surgeon Ambroise Par\'{e} invented modern procedures for amputation and the creation of prostheses. Therefore he is considered by many as being the father of modern amputation surgery and prosthetic design.\\ Between then and the U.S. Civil War, there are not many recordings about improvements of arm prostheses, only leg prostheses were improved a lot. During the U.S. Civil War on the other hand, the amount of upper-limb amputations grew enormously, so that the field of prosthetics started to be worth another investment. Still the hook remained one of the most popular choices of amputees. In the 1890's, D. W. Dorrance developed the split hook which survives up today. 

\begin{figure}[H]
\subcaptionbox{A prosthetic hook from the civil war in the U.S. The system can be fitted onto the stump of the amputee and is enormously robust. Image from \cite{amputee-coalition}.\label{fig:civil-war-hook}}{\includegraphics[width=0.4\linewidth]{History/civil-war-hook}}\hspace{0.15\textwidth}
\subcaptionbox{The split hook 5x by the Hosmer company. It is a robust gripper actuated by the amputee's body and can be worn in basically any environment. Image from \cite{amputee-coalition}.\label{fig:split-hook}}{\includegraphics[width=0.3\linewidth]{History/split-hook}}\\
\end{figure}

Today's systems have proven to be an effective and reliable device to perform activities of daily living. Generally they work all in the same way, yet there are some variations on the market. Most are made from two hook shaped metal prongs which are fixed at the rear side, similar to the pincers of prawns. They are held together by a tension band that defines the gripping force if the device is closed. The prongs are opened by a cable that is connected to a shoulder halfter. By flexing the shoulders, the prongs can be opened and closed.

\begin{figure}[H]
\centering
\subcaptionbox{\label{fig:body-powered-system2}}{\includegraphics[width=0.6\linewidth]{History/Bowden-cable-system}}\\
\subcaptionbox{\label{fig:body-powered-system}}{\includegraphics[width=0.4\linewidth]{History/body-powered-system}}\hspace{0.15\textwidth}
\subcaptionbox{\label{fig:body-powered-system2}}{\includegraphics[width=0.4\linewidth]{History/body-powered-system2}}\\
\caption[Bowden cable system]{The upper images show the famous Bowden cable system, which is still used in devices of today. The prongs are actuated by the  ''gross'' flexion of the user's shoulders. Images from \cite{Billock1986,amputee-coalition}}
\label{body-powered system}
\end{figure}

\subsection{Active prostheses}
Active prostheses are the last category of prostheses. They are a very young type of device compared to their predecessors, mostly are electrically driven and use actuators to move its components. The actuators can be directly electrically driven (servo-motors, solenoids, shape alloys) or use actuation systems such as fluidal or gas pressure systems. Most common commercial systems are servo-motor driven and have one or two degrees of freedom. The motorized hands or hooks are activated by antagonist residual muscle contractions. Traditionally two electromyographic (EMG) sensors are placed on the stump of the amputee to capture a signal that encodes a certain command to the prosthesis or just defines the force output or position of the fingers. Prosthetic hands, being the youngest section of prosthetic electrically driven devices are available since the early 1970's and are produced by different manufacturers such as Otto Bock(Austria), RSL Steeper(UK), Motion Control(Utah), LTI(Massachusetts), Bebionic(UK) or Touch Bionics(Scotland). Weighing up to about 600g and having having forces to carry up to 90kg, they can be considered as the state of the art of prosthetic manufacture. 


\begin{figure}[htp]
\centering
\subcaptionbox{The fluidhand, which is driven by fluidal bellows and provides a low noise alternative to the servo driven prostheses. \label{fig:fluidhand}}{\includegraphics[width=0.3\textwidth]{Hands/fluidhand}}\hspace{0.1\textwidth}
\subcaptionbox{The iLimb Pulse hand prosthesis, which weights 465g and has a highly modular way of construction for faster repair. \label{fig:iLimb}}{\includegraphics[width=0.3\textwidth]{Hands/i_limb_hand2} }\\
\subcaptionbox{The bebionic v.2 hand, which is able to exert a force of 900N, closes its hand in 0.9s and makes 10 of 14 grip patterns accessible to the amputee at once.\label{fig:bebionic}}{\includegraphics[width=0.3\textwidth]{Hands/Bebionic_Hand}}\hspace{0.1\textwidth}
\subcaptionbox{The MPL hand which provides a prosthesis to different amputation levels(shoulder, humeral and elbow), because it is self contained, 20 degrees of freedom and actuation and low level position control.\label{fig:MPL}}{\includegraphics[width=0.3\textwidth]{Hands/MPL_hand2}}\\
\label{active-prostheses}
\end{figure}

Still they can not compete in reliability and durability with the non-actuated counterparts as several studies show. Drawbacks such as low functionality, cosmesis and controllability make such devices less attractive to users \cite{Carrozza2002} and surveys and interviews with numerous amputees revealed that 30 to 50\% of upper limb amputees do not use them regularly \cite{Atkins1996}. Compared to the sensory transmission of a body-powered prosthesis that transmits vibration and grasping force through the harness to the stump with intact sensory feedback, the myo-prostheses get rejected for the lack thereof \cite{Lundborg2001,Atkins1996}.

In 2007, a Scotland based company called Touch Bionics created a new product called the i-Limb hand. Overcoming several of the drawbacks, it has five individually powered fingers and can adduct and abduct the thumb. This number of degrees of freedom seen only in a few other hands after the i-Limb enables the hand to achieve so called grasping patterns. Those patterns can be chosen by the user depending on the type of object to be grasped in order to get the best grip on it. Nevertheless the system can barely transmit sensory feedback to the user and its control is limited to be done with the traditional two EMG sensors. The development of a more functional and naturally controlled prosthetic hand has been an active research field for decades and is still one of the big research challenges in rehabilitation, for which a close collaboration between engineers, neuroscientists, medical doctors and patients is required \cite{TheSmartHand2011}.

For a list of active prostheses and robotic hands see Appendix \ref{AppendixB}.

\begin{figure}[H]
\subcaptionbox{Electrically powered, switch controlled prosthesis, driven by the signal of the ''fine'' flexions of the amputee's shoulders.\label{fig:switch-controlled-system}}{\includegraphics[width=0.4\linewidth]{History/switch-control-system}}
\hspace{0.15\textwidth}
\subcaptionbox{Electrically powered, myoelectrically controlled hand prosthesis driven by the EMG potentials of the amputee's residual limb.\label{fig:myo-electric-system}}{\includegraphics[width=0.4\linewidth]{History/myo-electric-system}}\\
\caption[Electrically powered prosthesis]{Images from \cite{Billock1986}}
\label{electrically-powered-systems}
\end{figure}

\section{Pneumatically actuated prostheses}

Pneumatic power is historically known for being a source of high externalizable power. In the 1970s, many pneumatically actuated hands and arm designs were employed. In 1971, Cool and Van Hooreweder built a hand prosthesis that was powered by pneumatics with an adaptive mechanism and internally powered fingers \cite{Cool1971}. The hand achieved a good grasping with a very low force because it could adapt to the shape of the grasped object. Simpson et al. developed the Edinburgh Arm in 1973 which was a pneumatically powered arm, which unfortunately was only used in a limited manner \cite{Simpson1973}. Its main disadvantage was its mechanical complexity which was very prone to failure. Another development based on pneumatics was the $C0_2$ powered hand by Kenworthy et al. in 1974 \cite{Kenworthy1974}, which could be used together with the Edinburgh arm. Several of those developments led the company Otto Bock Healthcare to selling $CO_2$-powered systems until they were taken from the market in the mid 1970s. The known disadvantages associated with pneumatic are that a gas tank has to be integrated either into the prosthesis or has to be carried by the user. Further there must be a supply for gas tanks readily available. A further disadvantage is that the tubing from the gas supply to the pneumatic prosthesis makes the self-containment of the prosthesis very difficult.

The market currently features no pneumatically powered commercially available prostheses. Still there are multiple groups working on pneumatic actuation systems for prosthetic applications. The Wilmer group at the Technical University of Delft is trying to build a hand actuation mechanism that runs on small disposable $CO_2$ cartridges called ''Sparklets'' as its gas supply which are normally used in soda machines \cite{Plettenburg2002}. Another group in Seattle in the BioRobotics Lab tries to build McKibben pneumatic muscles that mimic natural muscles.

%----------------------------------------------------------------------------------------
\section{User requirements and prosthesis abandonment}
\label{abandonment}

During the last 40 years, most prosthetic systems and their control have barely changed. Even though many different prostheses have been developed, most of them lack dexterity and sensor integration. All of them still have to deal with the fact that a large bandwidth human-machine interface is non-existent, which makes it very difficult to control a highly dexterous prosthesis. The dexterity itself is also limited by several constraints for prostheses such as its weight, size or noise. Another limitation is the current technological state of the art which is not yet able to supply components with response time, weight and strength as well as energy consumption rates that are appropriate for the perfect limb replacement.

The above mentioned facts add up to the main reasons why approximately 78\% of people with amputations in the United States prefer hooks over actuated hands \cite{LeBlanc1998}. In general, users prefer hooks and passive prostheses over myo-electric, active systems \cite{LeBlanc1998}. Compared to hooks, prosthetics hands generally offer less function and durability at greater weight and cost \cite{abandonment1}. Nonetheless, some individuals still choose hands over hooks primarily for cosmetic reasons \cite{Newman2008}. Since there are so many different groups of people with their individual needs, wishes and problems that require a prosthetic solution, it seems particularly difficult to define common requirements and reasons for the abandonment of a device. However, several point keep reccuring throughout the surveys.

Numerous studies tried to find the reasons for prosthesis abandonment. The aforementioned study called ''Upper Limb Prosthetics- Critical factors in device abandonment'' by Elaine A. Biddis et al. mentions several reasons such as that users are just as or more functional without it, they feel more comfortable without it, that it is too difficult or tiring to use, that it is too heavy, the high costs of such a device, that they have more sensory feedback without it. Further mentioned are a dissatisfaction with the prosthetic technology or its appearance, medical factors such as skin irritation or that it needed to be removed for several activities such as sleeping or swimming \cite{abandonment2}.

One of the earlier surveys were conducted by Atkins, Heard \& Donovan in 1996 \cite{Atkins1996}. They had more than 1500 prosthesis wearers with amputations of the upper-limb in their dataset. The systems used by the majority were body-powered prostheses(1020), the second biggest group were externally powered devices(438). The following results were extracted from both of these groups: Sensory-feedback was mentioned as being a real improvement, but also rotation of the wrist or the coordinated motion of several joints together were desired. Many patients desired to be able to use doors or tools or to eat with cutlery. The most desired motions mentioned required the full dexterity of the thumb, individual finger motions. Especially for electrical systems the extremely high price was mentioned and that the device requires a lot of maintenance such as cable repairs and battery replacements.

An survey conducted Datta, Selvarajah \& Davey from 2004 shows that among 62 patients in their survey, about 34\% of them abandoned their device \cite{Datta2004}. Many who continue to wear them do not find them useful in active daily living (ADL) and employment, 25\% of patients found the prosthesis beneficial for driving and a small part used the prosthesis for employment and recreational activities, but the vast majority used the prosthesis primarily for cosmesis. 

In a survey by Dudkewicz et al. conducted in 2004 \cite{Dudkiewicz2004}, 70\% of the respondents reported difficulties with their prosthetic systems. Especially mentioned were the weight of the prosthesis, discomfort with the fitting, limited functionality and cosmetic issues. That is why most of the patients in this study abandoned their other devices and used cosmetic hands only.

The requirements that the MANUS hand prosthesis originating a survey by Pons et al. (2005) \cite{Pons2005}. It was conducted among 150 clinicians that are specialized in prosthetic fitting and rehabilitation training as well as 200 prosthetic users. Reasons for prosthetic abandonment mentioned in the results were fitting discomfort as well as skin problems, weight and acoustic noise and aesthetic displeasure. From the clinicians the main problems were maintenance, price and autonomy. More grasping patterns as well as tactile feedback were mentioned as desired properties. 

According to interview results among the amputee community \cite{Pylatiuk2005}, and to the approximate percentage of utilization of the main grips in activities of daily living (ADLs) \cite{Light2002}, a minimum set of functionalities (grasps and gestures) that new hands should consider allowing, have been traced as follows:
\begin{itemize}
\item Power grasps (used in 35\% ADLs)
\item Precision grasps (30\% ADLs)
\item Lateral grasp (20\% ADLs)
\item Index pointing (useful for typing on a keyboard as well as for pressing a button, etc.)
\item Basic gestures (counting etc.)
\end{itemize}

A survey from 2007 created by Pylatiuk, Schulz \& Doderlein that distinguishes between different levels of amputation mentions the weight as the main problem to solve. Further notable reasons for abandonment were the lack of control of the grasp speed as well as the noise. Most patients would like to have force as well as temperature feedback. The survey also found out on what actions the prosthetic users mostly depend, which should also be taken into account when designing hand patterns. Among the activities mentioned by 60\% or more of the users were the operation of domestic devices, dressing and undressing, eating with knife, fork and spoon and personal hygiene. Using doors (60\%), writing (40\%), using individual fingers, especially the index finger or the supination and pronation of the wrist are of importance as well \cite{Pylatiuk2007}.

Kyberd et al. (2011) \cite{Kyberd2011} conducted a survey with 180 participating upper-extremity prosthesis users from Sweden, the UK and Canada. Most participants noted that they were wearing their prostheses for 8 or more hours each day but only 40\% of them were using active prostheses. The most unsatisfying facts about the prostheses reported were the lack of sensory-feedback, the missing precision of control and adaptation to objects followed by problems with lifting heavy objects and releasing objects after grasping them. It has to be mentioned that no prosthesis abandoners participated in this survey, every participant was using a device that he or she was satisfied with. Still they could mention several areas of improvement, such as the reliability of the system, the weight and the acoustic noise, as well as the interference caused by electrical noise. Moreover, the range of grip patterns accessible to the user should be improved.

\paragraph{Passive prostheses} are often only the subject of qualitative studies opposed to active prostheses. Users have attributed to them with cosmetic as well as with significant functional value \cite{vanLunteren1983,Fraser1998,Pillet2001}. Passive prostheses have a rejection rate of about 38.8\% (weighted average from \cite{vanLunteren1983,Millstein1986,Silcox1993,Keijlaa1993,Leow2001,Dudkiewicz2004})
The prostheses have had only little decline caused by the myo-electric prostheses that have functional as well as cosmetic appeal \cite{vanLunteren1983,Millstein1986,Silcox1993,Keijlaa1993,Leow2001,Dudkiewicz2004}.
Recent findings show that both infant and adult cosmetic prostheses users even have a higher rate use them permanently \cite{Dudkiewicz2004,Crandall2002}. The development of life-like silicone gloving which increases the cosmesis of a prosthesis\cite{Huang2001} may also have been more satisfying to the consumer.
Although the system lack's active grasping capabilities, passive prostheses tend do elicit the fewest concerns to the user. The most significant concerns are wearing temperature, glove problems or excessive weight, wear on clothes, and strap irritation \cite{Keijlaa1993}.

\paragraph{Body-powered prostheses} have the highest use of all three categories. Despite the advancement in technology of electrically driven devices, body-powered systems remain very popular among upper limb amputees, especially the U.S. users; in 1998 78\% preferred a hook over an electrically driven hand\cite{LeBlanc1998}. The advantage over myo-electric prostheses is that they do not need batteries and do not contains electronics that can be damaged, therefore users do not have fear to do strenous tasks or physical labor. Furthermore the modularity of such a device and good visibility of objects being handled are reasons for users to choose this device\cite{Millstein1986}. Additionally, an entire hook with strap and harness usually costs less than 10'000 USD which is a lot less expensive than the myo-electric prostheses and lasts several years and with little maintenance time. The highest costs come from the custom molded socket that is individually made from carbon fiber. The highest rated reasons for rejection for this type of prosthesis are the low speed of the device, the awkward use, difficulty in cleaning and maintenance, excessive weight, insufficient grip strength and the high operation force that has to be provided by the user \cite{Millstein1986,Keijlaa1993}. Most consumers also report the excessive wearing temperatures, wire or harness failure and abrasion of clothes \cite{vanLunteren1983,Keijlaa1993,Bhaskaranand2003,Dudkiewicz2004}.
All of these reasons contribute to a rejection rate of 32.8\% (weighted average of \cite{Scotland1983,Kruger1993,Crandall2002,vanLunteren1983,Millstein1986,Silcox1993,Keijlaa1993,Pinzur1994,Bhaskaranand2003,Dudkiewicz2004}).
Several studies identified longterm goals for improvement of such systems. In general, the comfortability of the harness and higher durability of the cables and gloving as well as increased wrist movement, improvement control mechanisms requiring less supervision or the actuation of multiple joints are mentioned \cite{Atkins1996}. 

\paragraph{Active prostheses} represent systems with the highest complexity in the range of prostheses, but they face only an insignificantly lower rejection rate than the other two categories. Their rate is at 29.4\% (weighted average of \cite{Trost1983,Mendez1985,Glynn1986,Menkveld1987,Datta1989,Ballance1989,Berke1991,Kruger1993,Hubbard1997,Routhier2001,Crandall2002,Hermanson2005,Northmore1980,Herberts1980,Stein1983,vanLunteren1983,Heger1985,Millstein1986,Datta2004,Dalsey1989,Silcox1993,Keijlaa1993}).
Generally spoken, active prostheses are accepted to offer advantages in appearance, increased force output, ease of operation and lack of harness, but problems faced are increased maintenance in addition to higher cost and weight \cite{Scotland1983,Trost1983,Datta1989,Ballance1989,Glynn1986,Weaver1988}.
Further missed are features such as sensory feedback, overall function and comfort. Even though a decrease of rejection rate was expected over the years due to advances in engineering, social funding, prosthesis availability, training services, no trends seem to reflect that hypothesis.\\
Together with body-powered prostheses active prostheses tend to be worn extensively for social activities and at work for more than 8 h per workday \cite{Fraser1998,Crandall2002,Silcox1993,Northmore1980,Hubbard1997,Kyberd1993,Millstein1986,Datta1989,Weaver1988,Keijlaa1993,Scotland1983}.
In general, active prosthesis users tend to have multiple prostheses \cite{Crandall2002,Leow2001,Trost1983,Datta1989,Mendez1985}.

\begin{comment}
Comparative Studies
One prospective controlled study compared preferences for body-powered and myoelectric
hands in children. Juvenile amputees (toddlers to teenagers, n=120) were fitted in a randomized
order with one of the two types of prostheses; after a three-month period, the terminal devices
were switched, and the children selected one of the prostheses to use. After two years, some
(n=11) of the original study sites agreed to reevaluate the children, and 78 (74% follow-up from
the 11 sites) appeared for interview and examination. At the time of follow-up, 34 (44%) were
wearing the myoelectric prosthesis, 26 (34%) were wearing a body-powered prosthesis (13 used
hands and 13 used hooks), and 18 (22%) were not using a prosthesis. There was no difference
in the children’s ratings of the myoelectric and body-powered devices (3.8 on a 5-point scale).
Of the 60 children who wore prosthesis, 19 were considered to be “passive” users, i.e., they did
not use the prosthesis to pick up or hold objects (prehensile function). A multicenter within-
subject randomized study, published in 1993, compared function with myoelectric and body-
powered hands (identical size, shape, and color) in 67 children with congenital limb deficiency
and nine children with traumatic amputation. Each type of hand was worn for three months
before functional testing. Some specific tasks were performed slightly faster with the
myoelectric hand; others were performed better with the body-powered hand. Overall, no
clinically important differences were found in performance. Interpretation of these results is
limited by changes in technology since this study was published.
Silcox et al conducted a within-subject comparison of preference for body-powered or
myoelectric prostheses in adults. Of 44 patients who had been fitted with a myoelectric
prosthesis, 40 (91%) also owned a body-powered prosthesis and nine (20%) owned a passive
prosthesis. Twenty-two (50%) patients had rejected the myoelectric prosthesis, 13 (32%) had
rejected the body-powered prosthesis, and five (55%) had rejected the passive prosthesis. Use
of a body-powered prosthesis was unaffected by the type of work; good to excellent use was
reported in 35% of patients with heavy work demands and in 39% of patients with light work
demands. In contrast, the proportion of patients using a myoelectric prosthesis was higher in the
group with light work demands (44%) in comparison with those with heavy work demands
(26%). There was also a trend toward higher use of the myoelectric prosthesis (n=16) in
comparison with a body-powered prosthesis (n=10) in social situations. Appearance was cited
more frequently (19 patients) as a reason for using a myoelectric prosthesis than any other
factor. Weight (16 patients) and speed (ten patients) were more frequently cited than any other
factor as reasons for non-use of the myoelectric prosthesis.
McFarland et al conducted a cross-sectional survey of upper limb loss in veterans and service
members from Vietnam (n=47) and Iraq (n=50) who were recruited through a national survey of
veterans and service members who experienced combat-related major limb loss. In the first year
of limb loss, the Vietnam group received a mean of 1.2 devices (usually body-powered), while
the Iraq group received a mean of 3.0 devices (typically one myoelectric/hybrid, one body-
powered, and one cosmetic). At the time of the survey, upper-limb prosthetic devices were used
by 70% of the Vietnam group and 76% of the Iraq group. Body-powered devices were favored
by the Vietnam group (78%), while a combination of myoelectric/hybrid (46%) and body-powered (38%) devices were favored by the Iraq group. Replacement of myoelectric/hybrid
devices was three years or longer in the Vietnam group while 89% of the Iraq group replaced
myoelectric/hybrid devices in under two years. All types of upper limb prostheses were
abandoned in 30% of the Vietnam group and 22% of the Iraq group; the most common reasons
for rejection included short residual limbs, pain, poor comfort (e.g., weight of the device), and
lack of functionality.
A cross-sectional study from ten Shriners Hospitals assessed the benefit of prosthesis (type not
described) on function and health-related quality of life in 489 children 2 to 20 years of age
with a congenital below-the-elbow deficiency (specific type of hand malformation). Outcomes
consisted of parent- and child-reported quality of life and musculoskeletal health questionnaires
and subjective and objective functional testing of children with and without prosthesis. Age-
stratified results were compared for 321 children who wore prosthesis and 168 who did not,
along with normative values for each age group. The study found no clinically relevant benefit
for prosthesis wearers compared with non-wearers, or for when the wearers were using their
prosthesis. Non-wearers performed better than wearers on a number of tasks. For example, in
the 13- to 20-year-old group, non-wearers scored higher than wearers for zipping a jacket,
putting on gloves, peeling back the plastic cover of a snack pack, raking leaves, and throwing a
basketball. Although prostheses have been assumed to improve function, no benefit was
identified for young or adolescent children with this type of congenital hand malformation.
Non-Comparative Studies
A 2007 systematic review of 40 articles published over the previous 25 years assessed upper
limb prosthesis acceptance and abandonment. For pediatric patients the mean rejection rate was
38% for passive prostheses (one study), 45% for body-powered prostheses (three studies), and
32% for myoelectric prostheses (12 studies). For adults, there was considerable variation
between studies, with mean rejection rates of 39% for passive (six studies), 26% for body-
powered (eight studies), and 23% for myoelectric (ten studies) prostheses. The study authors
found no evidence that the acceptability of passive prostheses had declined over the period from
1983 to 2004, “despite the advent of myoelectric devices with functional as well as cosmetic
appeal.” Body-powered prostheses were also found to have remained a popular choice, with the
type of hand-attachment being the major factor in acceptance. Body-powered hooks were
considered acceptable by many users, but body-powered hands were frequently rejected (80–
87% rejection rates) due to slowness in movement, awkward use, maintenance issues, excessive
weight, insufficient grip strength, and the energy needed to operate. Rejection rates of
myoelectric prostheses tended to increase with longer follow-up. There was no evidence of a
change in rejection rates over the 25 years of study, but the results are limited by sampling bias
from isolated populations and the generally poor quality of the studies included.
Biddiss and Chau published results from an online or mailed survey of 242 upper limb
amputees from the United States, Canada, and Europe in 2007. Of the survey respondents, 14%
had never worn a prosthesis and 28% had rejected regular prosthetic use; 64% were either full-
time or consistent part-time wearers. Factors in device use and abandonment were the level of
limb absence, gender, and perceived need (e.g., working, vs. unemployed). Prosthesis rejectors
were found to discontinue use due to a lack of functional need, discomfort (excessive weight
and heat), and impediment to sensory feedback. Dissatisfaction with available prosthesis
technology was a major factor in abandoning prosthesis use. No differences between users and
non-users were found for experience with a particular type of prosthesis (passive, body-
powered, or myoelectric) or terminal device (hand or hook).
In another online survey, the majority of the 43 responding adults used a myoelectric prosthetic
arm and/or hand for 8 or more hours at work/school (approximately 86%) or for recreation
(67%), while the majority of the 11 child respondents used their prosthesis for 4 hours or less at
school (72%) or for recreation (88%). Satisfaction was greatest (more than 50% of adults and
100% of children) for the appearance of the myoelectric prosthesis and least (more than 75% of
adults and 50% of children) for the grasping speed, which was considered too slow. Of 33
respondents with a transradial amputation, 55% considered the weight “a little too heavy” and
24% considered the weight to be “much too high.” The types of activities that the majority of
adults (between 50% and 80%) desired to perform with the myoelectric prosthesis were
handicrafts, operation of electronic and domestic devices, using cutlery, personal hygiene,
dressing and undressing, and to a lesser extent, writing. The majority (80%) of children indicated
that they wanted to use their prosthesis for dressing and undressing, personal hygiene, using
cutlery, and handicrafts.
A 2009 study evaluated the acceptance of a myoelectric prosthesis in 41 children 2 to 5 years of
age. To be fitted with a myoelectric prosthesis, the children had to communicate well and
follow instructions from strangers, have interest in an artificial limb, have bimanual handling
(use of both limbs in handling objects), and have a supportive family setting. A one- to two-
week interdisciplinary training program (in-patient or out-patient) was provided for the child
and parents. At a mean two years’ follow-up (range 0.7–5.1 years), a questionnaire was
distributed to evaluate acceptance and use during daily life (100% return rate). Successful use,
defined as a mean daily wearing time of more than two hours, was achieved in 76% of the study
group. The average daily use was 5.8 hours per day (range 0–14 hours). The level of amputation
significantly influenced the daily wearing time, with above elbow amputees wearing the
prosthesis for longer periods than children with below elbow amputations. Three of 5 children
(60%) with amputations at or below the wrist refused use of any prosthetic device. There were
trends (i.e., did not achieve statistical significance in this sample) for increased use in younger
children, in those who had in-patient occupational training, and in those children who had a
previous passive (vs. body-powered) prosthesis. During the follow-up period, maintenance
averaged 1.9 times per year (range of 0–8 repairs); this was correlated with the daily wearing
time. The authors discussed that a more important selection criteria than age was the activity
and temperament of the child; for example, a myoelectric prosthesis would more likely be used
in a calm child interested in quiet bimanual play, whereas a body-powered prosthesis would be
more durable for outdoor sports, and in sand or water. Due to the poor durability of the
myoelectric hand, this group provides a variety of prosthetic options to use depending on the
situation. The impact of multiple prostheses types (e.g., providing both a myoelectric and body-
powered prosthesis) on supply costs, including maintenance frequency, are unknown at this
time.
An evaluation of a rating scale called the Assessment of Capacity for Myoelectric Control
(ACMC) was described by Lindner et al in 2009. For this evaluation of the ACMC, a rater
identified 30 types of hand movements in a total of 96 patients (age range 2–57 years) who
performed a self-chosen bimanual task, such as preparation of a meal, making the bed, doing
crafts, or playing with different toys; each of the 30 types of movements was rated on a 4-point
scale (not capable or not performed, sometimes capable, capable on request, and spontaneously
capable). The types of hand movements were variations of four main functional categories
(gripping, releasing, holding, and coordinating), and the evaluations took approximately 30
minutes. Statistical analysis indicated that the ACMC is a valid assessment for measuring
differing ability among users of upper limb prostheses, although the assessment was limited by
having the task difficulty determined by the patient (e.g., a person with low ability might have
chosen a very easy and familiar task). Lindner et al recommended that further research with
standard tasks is needed and that additional tests of reliability are required to examine the
consistency of the ACMC over time.
Although the availability of a myoelectric hand with individual control of digits has been
widely reported in lay technology reports, video clips and basic science reports, no peer-
reviewed publications were found to evaluate functional outcomes of individual digit control in
amputees.
Summary
The goals of upper limb prostheses relate to restoration of both appearance and function while
maintaining sufficient comfort for continued use. The identified literature focuses primarily on
patient acceptance and reasons for disuse; detailed data on function and functional status, and
direct comparisons of body-powered and newer model myoelectric prostheses are
limited/lacking. The limited evidence available suggests that in comparison with body-powered
prostheses, myoelectric components may improve range of motion to some extent, have similar
capability for light work but may have reduced performance under heavy working conditions.
The literature also indicates that the percentage of amputees who accept use of a myoelectric
prosthesis is approximately the same as those who prefer to use a body-powered prosthesis and
that self-selected use depends at least in part on the individual’s activities of daily living.
Appearance is most frequently cited as an advantage of myoelectric prostheses, and for patients
who desire a restorative appearance; the myoelectric prosthesis can provide greater function
than a passive prosthesis, with equivalent function to a body-powered prosthesis for light work.
Nonuse of any prosthesis is associated with lack of functional need, discomfort (excessive
weight and heat), and impediment to sensory feedback. Because of the differing advantages and
disadvantages of the currently available prostheses, myoelectric components for individuals
with an amputation at the wrist or above may be considered when passive or body-powered
prostheses cannot be used or are insufficient to meet the functional needs of the patient in
activities of daily living. Evidence is insufficient to evaluate full or partial hand prostheses with
individually powered digits; these are considered investigational.

[Bluecross Blueshield Alabama: Myoelectric Prosthetic Components for the Upper Limb]
\end{comment}

\subsection{Perspective}

Based on the user concerns, a lot of advancements are being investigated to improve the prosthetic limbs in areas of comfort, durability, function, control, and appearance with respect to myoelectric prostheses. Those concerning the improvement of the actuation are listed here.

\begin{tabular}{p{7cm}p{6cm}}
Exploration of novel materials and \newline actuators such as shape memory alloys & De Laurentis and Mavroidis 2002 \newline dos Santos et al. 2003\\
Pneumatic actuators & Pylatiuk et al. 2005 \newline Caldwell and Tsagarakis 2002\\
Micromotors & Kyberd et al. 2001 \newline Del Cura et al. 2003\\
Ultrasonic motors & Pons et al. 2002\\
Electroactive polymers & Del Cura et al. 2003\\                   
\end{tabular}

Further groups only loosely coupled with the prosthetics community are working on pneumatic actuators for robotic applications. Their findings together with those listed above are used throughout this thesis and can be found in the references.

\end{document}