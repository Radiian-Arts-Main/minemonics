% Appendix A
\documentclass[main]{subfiles}

\begin{document}
\chapter{How to make pneumatic artificial muscles} % Main appendix title

\label{AppendixA} % For referencing this appendix elsewhere, use \ref{AppendixA}

\lhead{Appendix A. \emph{How to make pneumatic artificial muscles}} % This is for the header on each page - perhaps a shortened title

\section{Introduction}

Pneumatic artificial muscles are actuators operated by pressurized air. Created from an internal pneumatic latex bladder and an outer braided coverage, they are mimicking the natural muscle in function as well as in integration. Normally they are directly attached to the lever arm and are working in an antagonistic manner. Compared to other pneumatic systems, they are easy to create and to repair, the materials are cheap to buy. PAMs are lightweight, compliant and soft and have an especially high force to weight ratio (400:1-3300:1). Pneumatic artificial muscles can also be used with other gases as long as they are inert as well as some light fluids. They can be twisted axially and can be mounted unaligned. They can also be used around bends.

\section{History}

In 1957, R.H. Gaylord patented a device which later was used by J. L. McKibben as the pneumatically actuated muscle (PAM) for a lower-limb therapy system. The also called McKibben Pneumatic Artificial Muscle is used in many biorobotical and biomechanical projects to study biomechanics in existing natural beings as well as the creation of new artificial beings or human support systems. J.L. McKibben was the first scientist who used the PAM for prosthetic applications and robotics. As he says in an interview, he built a system to bring motion to his little daughter's polio-paralyzed hands. Even though normally being an atomic physicist, who notably triggered the first atomic bomb test at Alamogordo near Los Alamos, New Mexico, he worked hard on a new mechanical muscle which would someday help other patients to move, grasp or even write again. The muscle was proposed to him by Dr. Vernon Nickell who was a chief orthopedist at the center, which brought McKibben to a report from German scientists who had designed an ingenious pneumatic gadget operated by carbon dioxide, that could inflate bladders to compress the arm muscle and thereby creating a pinch movement with the paralyzed hand. The "German idea" most likely was the development of pneumatically driven prosthetic hands and arms started in 1948 at the Orthopaedic Hospital in Heidelberg. The "Heidelberg Hand" was invented by Dr O. Haefner \cite{Haefner1958}.

McKibben teamed up with several doctors and technicians from the Rancho Los Amigos center to build a workable device. The muscle lied along the forearm and was fitted to the thumb and first and second finger by a splint and could be activated by releasing $CO_2$ into the tube of the PAM, causing a contraction motion at the fingers.
Dr. Nickell from the center called the device ''a wonderful source of energy...[being]...lightweight, simple and safe''. The application of the muscle initiated a crash research program in the hope to apply the muscle the muscle to move paralyzed elbows and shoulders, further used for artificial limbs or to legs and arms that are weakened. This offered a new variety for the limited lives of disabled people. The gadgets, when built similarly to the original invention of McKibben, would cost less than \$ 100.

Source: The Buckingham Post \cite{BuckinghamPost1958}. 
Further information also on The cybernetic-zoo : http://cyberneticzoo.com/?p=6674

\section{Theory}

The performance of pneumatic artificial muscles can be described in simple words. As soon as the internal bladder is pressurized by the inflowing air, the bladder radially expands and pushes against the inside of the braided mesh sleeve. The mesh is thereby forced to enlarge its diameter and thus shortens in its length. 

\section{Ressources}

Below are the ressources to build a pneumatic artificial muscle. The advantage of pneumatic artificial muscles is that their ressources can be bought very cheaply.

\begin{itemize}
\item Braided-sleeve (Fig. \ref{braided-sleeve} and \ref{braided-sleeve2})
\item Latex Tube (Fig. \ref{latex-tube} and \ref{latex-tube2})
\item Stocking (Fig. \ref{stocking})
\item Metal wire (Fig. \ref{metal-wire})
\item Tube fitting or pressure sensor matching the latex tube (Fig. \ref{pressure-sensor})
\item Polyurethane tube (PUR) (Fig. \ref{PUR-tube})
\item Pliers (Fig. \ref{pliers})
\item Denist picks (Fig. \ref{dentist-picks})
\end{itemize}

\begin{figure}[H]
\centering
\includegraphics[height=0.35\textheight]{Muscles/Making/what-you-need}
\caption[Required materials and tools]{This is an overview of all the materials necessary to build a pneumatic artificial muscle. All the materials will now be examined separately to give a good overview why they are needed.}
\label{what-you-need}
\end{figure}

\begin{figure}[H]
\centering
\includegraphics[height=0.35\textheight]{Muscles/Making/braided-sleeve}
\caption[Braided sleeving]{The braid that is used to cover the the inner bladder to prevent it from inflating and converts the radial expansion into an axial contraction.}
\label{braided-sleeve}
\end{figure}

\begin{figure}[H]
\centering
\includegraphics[height=0.35\textheight]{Muscles/Making/braided-sleeve2}
\caption[Close up of the braid]{A close up of the braid to show the fibres of the braid.}
\label{braided-sleeve2}
\end{figure}

\begin{figure}[H]
\centering
\includegraphics[height=0.35\textheight]{Muscles/Making/latex-tube2}
\caption[Latex tubing]{Latex tube for the inflatable bladder used inside of the muscle.}
\label{latex-tube}
\end{figure}

\begin{figure}[H]
\centering
\includegraphics[height=0.35\textheight]{Muscles/Making/latex-tube}
\caption[Latex tube for PAM]{Latex tube for the pneumatic artificial muscle. It is used as an internal bladder that is inflated.}
\label{latex-tube2}
\end{figure}

\begin{figure}[H]
\centering
\includegraphics[height=0.35\textheight]{Muscles/Making/stocking}
\caption[Women's stocking]{The stocking used in this tutorial is a normal women's stocking. Only the very thin stocking is useful, the thicker version of it can not be used. It can theoretically be left away in this tutorial, but is a very good measure against hysteresis and prevents damage of the inner bladder of the muscle that is caused by the braid and therefore the lifetime of the muscles is increased.}
\label{stocking}
\end{figure}

\begin{figure}[H]
\centering
\includegraphics[height=0.35\textheight]{Muscles/Making/metal-wire}
\caption[Metal wire]{The metal wire is used to fixate the braid at the ends of the inner bladder.}
\label{metal-wire}
\end{figure}

\begin{figure}[H]
\centering
\includegraphics[height=0.35\textheight]{Muscles/Making/pressure-sensor2}
\caption[The pressure sensor MPX5700AS with PUR tube]{The pressure sensor of the type MPX5700AS is already fitted with a piece of polyurethane tube.}
\label{pressure-sensor}
\end{figure}

\begin{figure}[H]
\centering
\includegraphics[height=0.35\textheight]{Muscles/Making/PUR-tube}
\caption[Piece of PUR tube]{A short piece of polyurethane tube is needed to keep one side of the PAM open and to make it connectable with a to pneumatic components such as tubes and valves with a quick fitting.}
\label{PUR-tube}
\end{figure}

\begin{figure}[H]
\centering
\includegraphics[height=0.35\textheight]{Muscles/Making/pliers}
\caption[Necessary pliers for PAM making]{These are the two types of plier that you need to make a pneumatic artificial muscle.}
\label{pliers}
\end{figure}

\begin{figure}[H]
\centering
\includegraphics[height=0.35\textheight]{Muscles/Making/dentist-picks}
\caption[Dentist picks]{The dentist picks are optional for the creation of PAM, but can be very helpful if something must be hooked and pulled without touching other parts.}
\label{dentist-picks}
\end{figure}

\section{Creation}

Before the beginning of the assembly of the muscle, some calculations for the specification of the muscle have to be done. 

%----------------------------------------------------------------------------------------

\subsection{Muscle diameter decision}

There are a number of parameters that significantly affect the performance of the flexible pneumatic
actuator.
The contraction stroke length, force generated, and air volume consumption are all
dependent on the geometry and material of the inner bladder and exterior braided shell. A number of
investigators have attempted to model the actuator, and these models fall into several categories: (a)
empirical models (Chou and Hannaford, 1996; Gavrilovic and Maric, 1969; Medrano-Cerda et al.,
Klute, Czerniecki, and Hannaford - 12Artificial Muscles: Actuators for Biorobotic Systems
1995), (b) models based on geometry (Chou and Hannaford, 1996; Schulte, 1961; Tondu et al., 1994;
Inoue, 1987; Cai and Yamaura, 1996), and (c) models that include material properties (Chou and
Hannaford, 1996; Schulte, 1961).
Our most recent efforts to improve model predictions of
performance have incorporated non-linear material properties; however, further work is necessary to
achieve satisfactory results (Klute and Hannaford, 2000).
\scriptsize
\setlength{\LTleft}{-70pt}%
\setlength{\LTright}{\LTleft}
\begin{longtable}{p{2.5cm}p{10cm}}
	Gaylord 1958 \cite{Gaylord1958}& $F = \frac{P \Pi D^2_{45^\circ}}{2}(3 cos^2(\phi)-1)$
	\newline input pressure P
	\newline the angle between each effective strand of the sheath and a line drawn upon the surface of the sheath parallel to its longitudinal axis $\phi$
	\newline Diameter at $\phi = 45^\circ$ D\\
	Hannaford and Chou \cite{Hannaford1995}& $ F = \frac{\Pi D_0^2P}{4}(3cos^2(\phi)-1) + \Pi P (D_0 t_k(2sin(\phi)-\frac{1}{sin(\phi)} - t_k^2$ 
	\newline input pressure P
	\newline the angle between each effective strand of the sheath and a line drawn upon the surface of the sheath parallel to its longitudinal axis $\phi$
	\newline Diameter at $\phi = 90^\circ$ $D_0$ 
	\newline Wall thickness $t_k$\\
	Chou and Hannaford \cite{Hannaford1995} & $F= \frac{P_g b^2}{4\Pi n^2(\frac{3L^2}{b^2}-1}$
	\newline internal gauge pressure $P_g$
	\newline thread length b
	\newline number of turns of single thread n
	\newline actuator length L\\
	Caldwell 2000 & \[F = \frac{Ph^2cos^2(\chi)}{2B^2\Pi}-\frac{Ph^2sin^2(\chi)}{4B^2\Pi}\]
	\[=\frac{\Pi D_0^2 P}{4} (3cos^2(\phi)-1) if \phi > sin^(-1)(\frac{D_{cap}}{D_0}) \]
	\[=\frac{\Pi P}{4} (2D_0^2 cos^2(\phi)-D_{cap}^2) if \phi < sin^(-1)(\frac{D_{cap}}{D_0}) \]
	\newline Maximum muscle diameter at $\chi = 90^\circ D_0$
	\newline Input pressure P
	\newline Braid angle $\phi$ (increases when contracted)
	\newline Cap diameter $D_{cap}$\\
	Colbrunn 2001& $[ F = \frac{P b^2}{4\Pi n^2}(\frac{3L^2}{b^2}-1)$
	\newline Input pressure P
	\newline the thread length b
	\newline number of turns of single thread n\\
	DeVolder2011 &
	$F = max[(F_{min}, \frac{(P-P_{Corr}) b^2}{4\Pi n^2}(\frac{3(L-L_{Corr})^2}{b^2}-1)] + max[0,k_b (L- L_{b0})]$
	\newline Input pressure P
	\newline Non-functional length $L_{Corr}$
	\newline Non actuating strain $P_{Corr}$
	\newline Strain of the elastic bladder $L_{b0}$
	\newline Stiffness of the elastic bladder $k_b$
	\newline Minimum actuation force $F_{min}$\\
	Surentu (1) & $ F =  \frac{3}{4n^2\Pi}P'(L^2 - 1/3b^2)$
	\newline Relative pressure P'
	\newline Muscle length L
	\newline braid fiber length b
	\newline Number of single fiber coils n\\
	Surentu (2) & $ F = P' \frac{3\Pi D_0^2}{8L_0^2}(L^2-L_0^2)$
	\newline Relative pressure P'
	\newline Initial muscle length $L_0 = \frac{1}{3}sqrt{3}b$
	\newline Initial Diameter $D_0 = \frac{2L_0^2}{n^2\Pi^2}$
	\newline Muscle length L\\
	Tsagarakis 2000 & \[F = frac{\Pi D_0^2 P}{4}(3cos^2(\phi)-1 + P [D_0 t_k (2sin(\chi)-\frac{1}{sin(\chi)}-t_k^2] if \chi > sin^{-1}\frac{D_{cap}}{D_0}
	\]
	\[F = \frac{\Pi P}{4}(2D_0^2cos^2(\chi)-D_{cap}^2)+\Pi P [D_0 t_k(2 sin(\chi)-\frac{1}{sin(\chi)}-t_k] if \chi < sin^{-1}\frac{D_{cap}}{D_0}\]
	\newline Input pressure P
	\newline Diameter $D_0$ at $\chi = 90^\circ$
	\newline Braid interweave angle $\chi$
	\newline End cap diameter $D_{cap}$
	\newline the angle between each effective strand of the sheath and a line drawn upon the surface of the sheath parallel to its longitudinal axis $\phi$ 
	\newline Wall thickness $t_k$\\
\bottomrule
\end{longtable}
\normalsize

Using the formulas above, it is possible to estimate the dimensions of several components of the muscle based on the application and the force output of the muscle. In the thesis, a synthetic approach was chosen instead as the formulas did somehow not apply well and resulted in unrealistic dimensions. The synthetic approach just builds muscles and tests their force output by applying weights to it until it does not contract anymore. By examining the numbers and the force output versus contraction plot, the new dimensions can be chosen. To do that it has to be mentioned that the diameter of the muscle influences the force output of the muscle as in the pneumatic cylinder ($F = P \cdot A$), the length of the muscle's contractile element (length of the effective inner bladder) affects the movement of the muscle. The latter moreover influences the compliance of the muscle, as a bigger volume of air represents a better compliant element. Longer muscle's compliance therefore can be controlled more easily.

\subsection{Ratio between inner and outer diameter}

An important dimension to decide upon is the ratio between the inner and outer diameter. The inner diameter hereby is the diameter of the rubber tube and the outer diameter is the diameter of the braided sleevage. Currently not yet explainable, it is found to be the best range of ratios to take a ratio between 0.666:1 and 0.75:1. These measures showed to produce very successful muscles compared to others.

Since the ratio does not need to be perfectly matched, a suitable composition of available materials can be found by searching for it in a list where the ratio is fixed and one of the dimensions is predefined to an integral number.\\

\begin{table}
\begin{tabular}{lllp{1cm}lll}
\toprule
\multicolumn{7}{l}{\textbf{Example lookup table for dimensions}} \\
\midrule
Rubber & Braid & \multicolumn{2}{l}{Braid}& Braid & Rubber & Rubber\\
{} & Ratio:\textbf{0.666:1} & \multicolumn{2}{l}{Ratio:\textbf{0.75:1}} & {} & Ratio:\textbf{0.666:1} & Ratio:\textbf{0.75:1}\\
\midrule
3 & 5 & 4 & {} & 2 & 1.3332 & 1.5 \\
6 & 9 & 8 & {} & 3 & 1.9998 & 2.25 \\
8 & 12 & 11 & {} & 4 & 2.6664 & 3 \\
9 & 14 & 12 & {}& 6 & 3.9996 & 4.5 \\
10 & 15 & 13 & {} & 8 & 5.3328 & 6 \\
12 & 18 & 16 & {} & 10 & 6.666 & 7.5 \\
14 & 21 & 19 & {} & 12 & 7.9992 & 9 \\
17 & 26 & 23 & {} & 14 & 9.3324 & 10.5 \\
\end{tabular}
\caption[Diameter ratios]{A table to look up successful combinations of diameters for braid and rubber tube. The diameters can be adjusted to whatever lies between the value at the ratio 0.666:1 and the ratio 0.75:1.}
\label{look-up-table}
\end{table}

Based on a table such as the example table \ref{look-up-table}, it is much easier to find suitable dimensions for the necessary resources. Searching for resources with dimensions that match the ratio the best has proven to be a successful strategy.

\subsection{Thickness of the rubber tube's wall}

Something that is often forgotten is the rubber's thickness. The thicker the rubber tube, the higher is the resistance of the tube's inflation. The rubber should not be too thin, otherwise the muscle can burst easily when inflated as it is more sensitive to the friction at the braid and can more easily overinflate and create bubbles through the fibers of the braid, which then are not trapped by the braid and rupture the rubber tube instantly. From the author's experience, it is better to reduce the thickness to its minimum than to stay on the save side and keep it too thick. Since the muscle is hand-made, it is difficult to give a rule of thumb for the thickness and one has to experiment with different thicknesses. A good plan would be to start with a thickness of about 2mm and create a muscle from it and then increase the dimension until the muscle does not burst anymore when inflated with the desired maximum pressure, but still does properly inflate and contract.

A further measure against the rupture of the rubber tube to increase the muscle's lifetime is to add stocking as it will be shown later, by which also the rubber tube's wall can be reduced.

\subsection{Openness of the fiber weaves of the braid}

Coming to the choice of the braid, it is to mention that the contraction ratio of the muscle is related to the 'openness'' of the fiber weave and the corresponding difference in length between fully stretched lengthwise and fully expanded width-wise. The contraction ratio moreover is inversely proportional to the strength of the muscles, meaning that a muscle with a high contraction is weaker than a muscle with a lower contraction. The more open the fiber weave is, the lower the pressure it can be run on should be to prevent a blowout where the tubing will slip between the fibers of the weave. The openness of the muscle can easily be measured by comparing the maximum and minimum diameter of the braid. The ratio between one and the other can be used as a measure for openness.

\begin{figure}[htp]
\centering
\includegraphics[width=0.5\textwidth]{Muscles/Artificial/Loose-braid}
\includegraphics[width=0.5\textwidth]{Muscles/Artificial/Loose-braid2}
\caption[Open fiber braid]{If a braid with very open fibers is used, the inner bladder can disturb the fibers. The muscle does not properly contract and can rupture.}
\label{loose-braid}
\end{figure}

%----------------------------------------------------------------------------------------

\subsection{Muscle length determination}
\label{muscle-length-determination}
Now that all materials are chosen and are in proper diameters, we have just one important dimension to determine left. The length of the whole muscle determines the travel distance if the muscle contracts. The travel distance of the muscle is highly dependent on the application. Using the previous measures, we can create a muscle that travels a short distance with high or low force, or travels a long distance with similar force but with a higher overall diameter. The synthetic approach can be used again here. By fabricating a muscle of eight centimeters in length when fully expanded and it has a travel distance of two centimeters (so that when fully contracted it is six centimeters long), this results in a contraction ratio of 25\%. That means that if we need a muscle that approximately travels about four centimeters, we need a muscle that is 16 centimeter long. For a good initial contraction ratio estimation it is good to start with a ratio of 18\%-25\% at 2-4 bar. The muscle normally contracts further if more pressure is applied then reach their final contraction limit, but are more likely to burst. In general it is better to build a longer muscle than to apply a higher pressure.

\paragraph{Preload}

\label{preload}
The preload of a pneumatic artificial muscle is the amount of force that the muscle exerts if no pressure is applied to the system. A muscle with preload has an internal tension that causes it to contract up to a certain level. Such a muscle has to be hooked to a certain weight or to an antagonistic muscle to keep it expanded. If pressure is applied, a muscle with preload contracts further than a muscle without preload. Preload can be added to the McKibben muscle by making the braid of the muscle longer than the according piece of latex tube. The braid in the muscle will then be slightly or strongly contracted depending on the amount of preload. By attaching a weight to the muscle, the braid contracts and the inner bladder is stretched until the braid touches the bladder. The bladder thereby exerts a certain force that causes a slight contraction of the muscles. Preload can be further used to keep the system under a certain tension to have tensed tendons.

\subsection{The morphology of the muscle}

The muscle that is fabricated in the section 'Fabrication' consists of an inflatable bladder built from a latex tube closed with an tube fitting or a pressure sensor on one side and kept open on the other side. On the open side, a polyurethane (PUR) tube is inserted to the latex tube to facilitate the connection of the muscle to other pneumatic equipment by push-fittings. Since the latex bladder just inflates extremely without doing any contraction, the bladder is covered by a braided sleeve which depending on the openness of the fibers weave contracts axially within a certain ratio to its radial expansion.

Depending on how much preload the application of the muscle can stand, the muscle can be built so that the inner bladder is shorter than the outer braid which makes it contract more than the usual 25\% of its full length. The braid is fixed on both ends of the latex bladder, meaning that metal wire is used to wrap it around the air fitting and the PUR tube side to attach the braid to it. Based on findings that were made during several experiments that are presented in previous chapters of this thesis, another layer can be added to the muscle between the latex and the braid which is made of nylon weave used for stockings. Since it is available in very soft and thin weave types and is available at very low prices, the muscle's price still remains very low. The advantage of this intermediate layer is that the latex tube is protected from scratches carved by the rather static braid. Further it reduces the hysteresis of the muscle and makes it contract further than ordinary pneumatic artificial muscles since it also reduces the friction between the braid and the latex bladder.
 
The muscle has two slopes on either end to hook or attach it to its application which are made from the braid's endings by pushing the PUR tube's end through the braid and then sloping the end of the braid. The slopes created in this manner are not only a lot more lightweight than the other options made from wire or metal such as those in commercial examples of FESTO or the Shadowhand company, but they also withstand very high forces when properly bound to the muscles by the metal wire. How such a muscle is fabricated is described in the following section.

\subsection{Fabrication}

Cut the latex tube to the calculated bladder length plus an additional 3-5 cm for the insertion of the fitting/pressure sensor and the PUR tube.

Then cut the braid to the calculated bladder length plus an additional 6-10 cm for the attachment slopes of the muscle. If you intend to add preload then extend the calculated bladder length up to 125\% of the calculated length.

\begin{figure}[H]
\centering
\includegraphics[height=0.35\textheight]{Muscles/Making/latex-in-stocking}
\caption[Latex tube fits stocking]{Before we start with the fabrication of the muscle, we need a piece of stocking in the appropriate length. Put the latex tube into the stocking, hold the the tube straight to avoid having a stocking piece that is too short and cut off at both ends to completely cover the tube with it. }
\label{latex-in-stocking}
\end{figure}

\begin{figure}[H]
\centering
\includegraphics[height=0.35\textheight]{Muscles/Making/plier-in-latex}
\caption[Plier in latex tube]{Take the bended plier to expand the latex tube and push the air fitting or the pneumatic sensor in it.}
\label{plier-in-latex}
\end{figure}


\begin{figure}[H]
\centering
\includegraphics[height=0.35\textheight]{Muscles/Making/insert-ps-into-latex}
\caption[Pressure sensor in latex tube]{Expand the latex tube with the bended plier and insert the pressure sensor far enough into the latex tube. Then push against the back end of the sensor and pull the bended plier out of it.}
\label{insert-ps-into-latex}
\end{figure}


\begin{figure}[H]
\centering
\includegraphics[height=0.35\textheight]{Muscles/Making/ps-inside-latex}
\caption[Pressure sensor fitted with latex tube]{Now the pressure sensor should be fitted into the latex tube. If the pressure sensor is tilted a bit to the side, pull on the latex tube to adjust this until the pressure sensor is in line with the tube. Pay attention to not damage the latex while doing this.}
\label{ps-inside-latex}
\end{figure}

\begin{figure}[H]
\centering
\includegraphics[height=0.35\textheight]{Muscles/Making/plier-in-latex2}
\caption[Plier with latex tube and pressure sensor]{Put the bended plier into the other end of the latex tube.}
\label{plier-in-latex2}
\end{figure}

\begin{figure}[H]
\centering
\includegraphics[height=0.35\textheight]{Muscles/Making/plier-opens-latex}
\caption[Expanded latex tube]{Expand the latex with the bended plier by opening the plier with both hands and then hold it open with one hand while inserting the piece of PUR tube into it.}
\label{plier-opens-latex}
\end{figure}

\begin{figure}[H]
\centering
\includegraphics[height=0.35\textheight]{Muscles/Making/PUR-inside-latex}
\caption[PUR tube inserted into latex]{Insert the PUR tube into the latex tube up to about 1.5 cm.}
\label{PUR-inside-latex}
\end{figure}

\begin{figure}[H]
\centering
\includegraphics[height=0.35\textheight]{Muscles/Making/both-sides-inserted}
\caption[Closed inner bladder]{The inner bladder is now closed on one side by the pressure sensor and can be inflated on the other side by the PUR tube. In this condition, it would inflate up to 500\% and then it would burst.}
\label{both-sides-inserted}
\end{figure}

\begin{figure}[H]
\centering
\includegraphics[height=0.35\textheight]{Muscles/Making/roll-into-stocking}
\caption[Wrapping the bladder into the stocking]{Put the inner bladder you just built into the piece of stocking and slowly wrap it into the stocking. By that the stocking creates multiple layers that will decrease the friction between the tube and the braid later. This is a good measure to reduce hysteresis in the muscle.}
\label{roll-into-stocking}
\end{figure}

\begin{figure}[H]
\centering
\includegraphics[height=0.35\textheight]{Muscles/Making/roll-into-stocking3}
\caption[Completely wrapped bladder]{The bladder should now be covered by the stocking and be held in place on both ends.}
\label{roll-into-stocking3}
\end{figure}


\begin{figure}[H]
\centering
\includegraphics[height=0.35\textheight]{Muscles/Making/put-into-braid}
\caption[Inserting the inner bladder into the braid]{Now that the inner bladder is complete, it has to be inserted into the braid. This is done by holding it at the PUR tube side of the bladder, compressing the braid so that the fibers are opening at the insertion point for the tube (depending on the additional length chosen at the beginning 3-5cm from one end of the braid) and then pushing the PUR tube inside. If the PUR tube has crossed the braid, contract the braid until it is possible to grab the PUR tube further inside. Then expand the braid again a bit while holding the PUR tube and the stocking in place inside the braid to pull the tube further inside. Then repeat the procedure until the PUR tube reaches the exit point in the braid. }
\label{put-into-braid}
\end{figure}

\begin{figure}[H]
\centering
\includegraphics[height=0.35\textheight]{Muscles/Making/half-into-braid}
\caption[Bladder during insertion into braid]{This is as it looks like during the process of insertion. The PUR tube is visible at the left end.}
\label{half-into-braid}
\end{figure}

\begin{figure}[H]
\centering
\includegraphics[height=0.35\textheight]{Muscles/Making/put-into-braid3}
\caption[Inserting the inner bladder into the braid 2]{At this point, the dentist picks can help by softly pulling the stocking in or just for the expansion the fibers of the braid at the exit point (depending on the additional length chosen at the beginning 3-5cm from the other end of the braid). It also helps if the stocking slips off the latex tube and has to be readjusted.}
\label{put-into-braid3}
\end{figure}

\begin{figure}[H]
\centering
\includegraphics[height=0.35\textheight]{Muscles/Making/put-into-braid4}
\caption[Inserting the inner bladder into the braid 3]{Readjust the stocking by pulling on both ends of it where it covers the endings of the inner bladder. Then expand the braid completely or leave it contracted up to the point you intended to add the preload to the muscle. If you properly cut the braid and chose the insertion and exit point according to the manual, you should be left with a contracted braid between the points if you added preload to the braid.}
\label{put-into-braid4}
\end{figure}

\begin{figure}[H]
\centering
\includegraphics[height=0.35\textheight]{Muscles/Making/tube-in-braid}
\caption[Bladder in braid]{This should be the result up to now. The braid in this case is fully expanded, as we do not need any preload with this muscle. For another application where it could be useful, increase the length of the braid between the insertion and the exit point of the bladder so that the braid can not expand completely. More about the preload is written in the section \ref{preload} above. Further you can see the loose ends that will be used for the slopes.}
\label{tube-in-braid}
\end{figure}

\begin{figure}[H]
\centering
\includegraphics[height=0.35\textheight]{Muscles/Making/wire-around-ends}
\caption[Fixation of the braid and the bladder]{Next we fixate the braid and the inner bladder to the PUR tube by using metal wire. Without this fixation, the braid would move and the latex tube would fall off while inflating. Take one of the loose ends of the braid and fold it to form a slope, which will be used later to attach the muscle to any mechanical lever. The loose end of the braid should now be folded onto the point where the latex tube and the stocking overlap the PUR tube. Now we take the metal wire and wrap it around this point so that we have enough wire to wrap it around three times at least.}
\label{wire-around-ends}
\end{figure}

\begin{figure}[H]
\centering
\includegraphics[height=0.35\textheight]{Muscles/Making/wire-around-ends2}
\caption[Tightening of the fixation]{Use the bigger plier and twist the ends of the wire. The wire can tightened fairly until you will feel that it uses more force to twist further. At this point, the wire is tightened and the PUR tube will not give in more. Further twisting will break the metal wire, therefore this end is finished.}
\label{wire-around-end2}
\end{figure}

\begin{figure}[H]
\centering
\includegraphics[height=0.35\textheight]{Muscles/Making/wire-around-ends3}
\caption[Fixation on the other side]{Repeat the same procedure on the pressure sensor side. Fold the other loose end of the braid onto the overlapping point of latex tube and PUR tube. Wrap the wire around the overlapping part three times.}
\label{wire-around-end3}
\end{figure}

\begin{figure}[H]
\centering
\includegraphics[height=0.35\textheight]{Muscles/Making/finished-muscle}
\caption[Finished muscle]{Now that we fixated both ends of the muscles, the muscle is finished and ready to be used in your application.}
\label{finished-muscle}
\end{figure}

\begin{figure}[H]
\centering
\includegraphics[height=0.35\textheight]{Muscles/Making/muscle-test}
\caption[Inflating the muscle]{The muscle is now connected to the compressor which inflates the muscle with a pressure of 4 bar. The PUR-end of the muscle can easily be connected to other tubes by a quick press fitting.}
\label{muscle-test}
\end{figure}

\begin{figure}[H]
\centering
\includegraphics[height=0.35\textheight]{Muscles/Making/muscle-test2}
\caption[Inflated muscle]{Set under pressure, the muscle quickly inflates and contracts. This muscle has a contraction of around 18\% which is at the lower end of contraction. This is due to a braid to inner bladder diameter ratio that is too low. Increasing it will definitely lead to a higher contraction. On the other hand, a small contraction makes the muscle stronger since its travel distance is lower.}
\label{braided-sleeve2}
\end{figure}

\begin{figure}[H]
\centering
\includegraphics[height=0.26\textheight]{Muscles/Making/finished-muscle-closeup}
\includegraphics[height=0.26\textheight]{Muscles/Making/finished-muscle-closeup2}
\caption[Finished-muscle-closeup]{To prevent yourself from harm it can be helpful shorten the ends of the metal wire and bend it down. Further they can be covered by isolation tape.}
\label{finished-muscle-closeup}
\end{figure}

\begin{figure}[H]
\centering
\includegraphics[height=0.26\textheight]{Muscles/Making/two-muscles}
\caption[Two muscles with and without stocking]{For a direct comparison, two muscles were built based on the same materials and dimensions with and without stocking. The stocking should not influence the contraction of the muscle. The only difference that matters much is the reduction of hysteresis and the lower friction between the braid and the inner bladder the latter increasing the lifetime of the muscle.}
\label{two-muscles}
\end{figure}

\begin{figure}[H]
\centering
\includegraphics[height=0.3\textheight]{Muscles/Making/other-muscle}
\includegraphics[height=0.3\textheight]{Muscles/Making/other-muscle-preload}
\includegraphics[height=0.3\textheight]{Muscles/Making/other-muscle-preload2}
\caption[Muscle with preload]{This smaller muscle has a preload, which can be seen in the middle of the muscle at the spacing between the inner bladder and the braid. Preload in pneumatic muscles increases the contraction and can be useful in some application if the muscle should never hand loosely but should always be under a certain tension. With the preload, the muscle can be stretched out as far as the braid contracts on the inner bladder and the inner bladder will be stretched further than its state without tension and the muscle will try to contract. This tension can be cancelled out with an antagonistic pair of the same type of muscles. The system will be of higher responsiveness as the tendons will always be completely stretched.}
\label{other-muscle}
\end{figure}

\end{document}