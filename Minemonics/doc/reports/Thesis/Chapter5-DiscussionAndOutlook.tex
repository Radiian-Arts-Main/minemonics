\documentclass[main]{subfiles}

\begin{document}
% Chapter Template
\setcounter{chapter}{4}

% Discussion and Outlook
%  Evolutionary limiting of the controller
\chapter{Discussion and Outlook} % Main chapter title

\label{Chapter\thechapter} % Change X to a consecutive number; for referencing this chapter elsewhere, use \ref{ChapterX}

\lhead{Chapter \thechapter. \emph{Discussion and Outlook}} % Change X to a consecutive number; this is for the header on each page - perhaps a shortened title
%
An artificial evolution simulator is developed to evolve virtual three dimensional creatures built from rigid-body primitives. %
%
The simulator features the evaluation of one or multiple populations in different environmental and evolutionary settings. %
%
Different fitness functions can be combined with weights and the fitness landscape can be changed during the evolutionary run. %
%
The simulations can be stopped, saved and continued later with little effort. %
%
The creatures can be manipulated during their evaluation and their internal controller and joint dynamics can be observed in real-time. %
%
The interface features various possibilities to log simulation data to file, so that further processing of the data is simplified. %

Creatures were evolved using three different, distributed controller structures, a non-adaptive, sinusoidal controller and two newly introduced, adaptive chaotic system controllers. %
%
One chaotic controller is only indirectly limited through the morphology of the creature, the other directly limited through the sensory feedback in addition to the limits through the morphology. %
%
In order to develop the chaotic controllers, it was necessary to understand the dynamics to be expected when limiting a chaotic system with the limiter chaos control method. %
%
Therefore, the findings of the simple limiter chaos control method introduced in \cite{bib:Corron2000} with the example model of the Chua circuit \cite{bib:Matsumoto1985} were confirmed and extended by the use of a tunable soft-limiter. %
%
Additionally, a hard and a soft limiter within two different limiter configurations, one dimension limiting another and one dimension limiting itself, and their range of different periodic orbits were analyzed. %
%
It was found that a soft-limiter expands the size of each limiter parameter range of periodic orbit and therefore makes it easier to achieve stable orbits. %
%
This might be explained by the increased flexibility of the system as the soft limiter does not instantly limit the trajectory to the full extent, but still allows some movement, so that the precise limiter position does matter less than in the hard limiter case. %
%
Furthermore, the configuration of a self-limiting dimension additionally expands the limiter ranges. %
%
The developed chaotic controllers internally control an underlying chaotic model system into chaotic and periodic trajectories using variations of the simple limiter chaos control method. %
%
In the directly limited case, the simple limiter hereby is applied through a coupling with the controlled joint degree of freedom by feeding the joint position and velocity output back into the chaotic controller, replacing the original state with the sensor values. %
%
Therefore, the directly limited, chaotic controller faces a limiter if the joint faces an external limiter imposed through the environment. %
%
Limited by this external load, the chaotic system changes its internal dynamics to cope with the new load. %
%
The indirectly limited, chaotic controller does not have this sensory feedback and therefore always exerts a chaotic trajectory, which is then only influenced by the limits of the morphology. %
%
The evolutionary algorithm in this context composes different creatures to find a morphology and appropriate internal control parameters which exploit the internal chaotic dynamics of the controller in such a way that together they show basic locomotion behavior. %
%
For the sinusoidal and the directly limited, chaotic controller, creatures could be evolved showing various ways to locomote. %
%
More generally, it could be shown that using an arbitrarily chosen chaotic system and a simple limiter varying its limitation of the state space depending on the external constraints from the environment, different adaptive locomotion patterns can arise, which are robust to external disturbances. %
%
Using the indirectly limited, chaotic controller, no locomoting creatures could be evolved. %
%
This can be explained through the fact that the limits of the morphology did not influence the controller behavior, and therefore no periodic movement was exerted from the controller. %
%
Furthermore, using the directly limited, chaotic controller, it could be shown that different leg dynamics are emerging depending whether the creature is touching the ground due to gravity or floating in gravity-less space. %
%
This could be shown in a hand-made model organism as well as in several evolved creatures, that showed different leg dynamics when lifted off the ground than when left on ground to locomote. %
%
In experiments with the model leg, featuring one single joint connecting two body parts, different periodic orbits could be stabilized. %
%
The influence of different joint morphological parameters on the behavior of the model leg, namely the amount of joint torque and damping, could be observed. %
%
Applying higher or lower amounts of joint torque increases or decreases the orbit diameter respectively, damping leads to a smoother curvature of the orbit. %
%
However, in all cases the joint converges to a proper periodic orbit.

In the evolved creature case, only some creatures could be found to feature simple limiter control behavior, while others do not feature it very prominently. %
%
Still, given that stabilized orbits can be observed in some creatures makes it a valid hypothesis to be pursued further.
%
A reason for the lack of creatures with said control behavior might be that the torque scaling curves used to improve the robustness of the simulation to unrealistic behavior made it impossible for evolution to put a load on the creatures. %
%
The load from the creature's weight applied as a limiter to the chaotic controller might be a missing variable for the evolutionary algorithm so that the simulation struggles to find a solution. %
%
The problem could be fixed by keep the original torque scaling, but giving evolution a parameter to increase or decrease the actually applied torque. %
%
Another option would be to use a constant torque for all joints, so that the size of the body parts can be varied to apply a load. %
%
The second option might generate an unexpected optimum body part size, so that most other sizes are either a too strong or too weak load to be used. %
%
Also one has to pay attention when varying the weight ratios, as physics simulations are very sensitive to high weight ratios, hence a change in densities of different limbs could destabilize the simulation. %
%
Further studies might be necessary to find appropriate ways for evolution to express and apply simple limiters in order to simplify the emergence of locomotion.%

An issue confronted with during the simulation run was, that no fitness function accounted for gait efficiency, therefore, some creatures considered fit by the fitness functions in use did not perform an efficient locomotion gait. %
%
The simulation additionally has a general tendency to come up with very messy creature morphologies exploiting oscillations of large amplitude, which would require large amounts of energy to keep up the resulting locomotion. %
%
An improvement to address this issue would be that a creature had an amount of energy to spend proportional to its size or weight, so that a creature could not waste an excessive amount of energy on inefficient movements. %

Another issue to address in the future would be the configuration of the limiter. %
%
The current simple limiter was evaluated experimentally to stabilize a periodic orbit, which is an indication that the limiter is applied to the system. %
%
This could be addressed by a less restricting limitation through the sensory feedback, for example the simple limiter could be applied by using a similar definition as that used in the example model of the Chua circuit. %
%
Making the position of the limit value a parameter could enable evolution to choose proper limiters, even though the experiment would be less agnostic of the simple limiter, but would answer the question if simple limiters can simplify locomotion to emerge. %
%
In another example, a higher body part velocity would increase the size of the phase space of the chaotic system, so that dynamics of higher periodicity would arise. %
%
However, it is in question how to design the limiter, because as it is important to not directly build in a desired outcome. %
%
Therefore the choice of an appropriate limiter configuration is a particularly hard task.

A future experiment could be to actually measure the chaoticity of the controller and joint dynamics. %
%
Preliminary results suggest that the controller could be an intermittent system switching back and forth between chaotic and periodic phases. %
%
An experiment to measure the chaoticity would involve a successfully locomoting creature's controllers to be perturbed using a small perturbation. %
%
If the controller is stabilized on a periodic orbit, the trajectory then would be moved away from the orbit by the perturbation, but finally converge back onto the limit cycle. %
%
However, if the controller is not stabilized, the trajectory would diverge exponentially from the original, unperturbed trajectory and never return. %
%
The divergence of behavior could be measured using the Largest Lyapunov exponent, the mean rate of separation of trajectories, and calculated with the form adapted from \cite{bib:Rosenstein1993} as described in Appendix \ref{AppendixA}. %
%
The Largest Lyapunov exponent then expresses if the controller has an attracting, stable limit cycle to which it returns after having been slightly perturbed with uniformly distributed perturbations around zero. % 
%
The perturbation can be automatically performed using the simulator by enabling automatic perturbation. %
%
The preliminary results, suggesting that the chaotic controller is an intermittent system, raise the question at which point during the evaluation the perturbation should be applied. %
%
A perturbation before the collision with the ground might be swallowed by the larger perturbation of the collision, obfuscating the actual results. %
%
Given that the creature starts above ground and falls onto it first, the best guess would be to perturb the creature after it has stabilized on ground and begun to perform its locomotion gait. 

\end{document}