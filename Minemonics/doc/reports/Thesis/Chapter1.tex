% Chapter 1
\documentclass[main]{subfiles}
\setcounter{chapter}{0}

\begin{document}
\chapter{Introduction} % Main chapter title

\label{Chapter1} % For referencing the chapter elsewhere, use \ref{Chapter1} 

\lhead{Chapter 1. \emph{Introduction}} % This is for the header on each page - perhaps a shortened title

%----------------------------------------------------------------------------------------
%Introduction
% Background
% Anatomical description (intrinsic/ extrinsic muscles/types of muscles)
% Motivation
% Approach
% Outline

\section{Background}
The mind meets the world through its senses and the interaction with objects. As long as a person has got normally functioning hands and legs and is able to walk, can use different senses to perceive its surroundings and is able to interact with the world, nobody ever thinks about how complicated our body works and how complex the movement of even a single finger can be. If a person loses a particular part of their body through an accident or a disease, this is most certainly a life changing event. So for instance is the loss of a hand or the whole arm. Most of the people suffer from such an event because the hand is not only a highly dexterous manipulator of our environment, but also connects us with the world and lets us interact with it by giving the human a massive amount of sensory input such as temperature, texture or touch. Since the hand contains as many as 14 different types of nerve fibers \cite{SensoryHand}, it perceives accurate information about its surroundings and can therefore even replace functions such as reading by the aid of braille signs. That the hand is so important can also be seen on the output side such as in gestures. Apart from what has been observed up to a limited capacity in chimpanzees, humans are the only species that is able to communicate with other humans with the hands only \cite{SensoryHand}. All cultures know gestures to communicate non-verbally and several of them include a performance of the hand and its fingers. Known all over the world is for instance the handshake with the right hand. Even though there are many conflicting theories about the origin, the most common theory states that it is a sign of peace showing the lack of a weapon in the right hand. Connecting the hands with a shake guarantees that the hand must be empty \cite{Weapon}. Not being able to participate in these social interactions separates a person from the others and causes a severe psychological and physical drawback. Furthermore if a person is missing the dexterous and precise manipulative abilities given by the hand, the person suffers from problems due to the fact that many systems in our world are made to be controlled or used by hands only such as steering wheels, machines with buttons as well as workshop tools, doors etc. Humans missing one or both hands can therefore not control these systems because they can not use a keyboard to write or steer a car by a wheel because they can not type with their fingers or hold on to the wheel.

Reenabling a person to do several of those tasks and and giving back some sensory input as well as creating robotic hands for space missions or industrial fabrication are fields that are under intensive investigation. Transradial prostheses and human computer interfaces have barely changed in the past 40 years \cite{TheSmartHand2011}. Despite the significant impact of losing a hand, manufacturers do not innovate their products much since the number of amputees needing a prosthesis is too small \cite{DillinghamRecentTrends2002}. Many different systems have been developed in the past decades such as the Sven hand, the Belgrade, the Southampton, the MARCUS, the TBM, the RTR II, the SPRING,
the MANUS, the Ultralight hands in Forschungszentrum Karlsruhe, the Soft hand, the KNU hand \cite{Razic1972,Herberts1973,Light2000,Kyberd1995,Dechev2001,Massa2002,Carrozza2004,Pons2004,Pylatiuk2004,Carrozza2005,Chu2008}, and some of those systems were attempted to be connected to the user via non-invasive feedback systems \cite{Kyberd1993,Pons2005,Pylatiuk2004}, nevertheless a lot of them (probably with the exception of the Southhampton-REMEDI hand, that contains extended sensorization and a number of degrees of freedom to allow prehensile patterns) would not be sufficiently dexterous or perceptive enough if a better control and sensory interface existed \cite{Cipriani2010}. This is the case because the artificial hand is still quite far away from the human hand as it will be shown in chapter \ref{Chapter2}. This is the case since developing a self-contained hand is probably one of the most challenging tasks in the field of prosthetics.\\ The rough requirements for such a system are quite clear and can be described by low power, low weight and still giving the capability to perform a number of prehension patterns useful in activities of daily living \cite{TheSmartHand2011}. Useful mechanisms to achieve such a system have been developed such as the reduction of actuators used by underactuating joints or saving power on a stable grasp by clutching and non-backdrivability. But that is not everything that has to be considered for a prosthetic hand. Dexterity in a hand is only useful if there is a control interface that can exploit the dexterity of the actuated hand to improve the grasping capability of the user. For that purpose, several controlling schemes have been developed such as antagonistic surface electrodes to get electro-myographic signals from residual muscles of an amputee's arm \cite{Nightingale1985}. Still the sensorization of such a hand is very limited and is usedmostly directly inside the prosthesis to improve the stability of the grasp or implement features such as auto-grasp capabilities.\\
While the prostheses mentioned above are all intrinsic hands systems, other devices try to separate the hand from the actuators and build extrinsically actuated hands. Some significant examples are the CyberHand\cite{Carrozza2006}, the Yokoi hand\cite{Ishikawa2000} and the Vanderbilt University prototypes \cite{Fite2008,Dalley2009}. Further revolutionary changes to the prostheses came in 2008-2010 with DARPA's Revolutionizing Prosthetics Program RPP intrinsic hand \cite{Weir2008}, a prototype from the Rehabilitation Institute of Chicago \cite{Mitchell2008}, or the commercial prostheses Ottobock Michelangelo hand, the RSL Steeper Bebionic hand, and the second release of i-Limb named Pulse from Touch Bionics. All of those systems are able to actuate all of their fingers separately and are able to do prehensile movements, even though yet lacking human interface bandwidth to control it delicately. They show how far the dexterity of the artificial hands has been developed up to this point in time. A problem that prostheses like those are still facing even with this high dexterity is prosthesis abandonment. Users have hidden requirements to their hands to accept them in their daily life, otherwise they just abandon their device. The difficulties in acceptance of the prosthesis add up to the complex aim to duplicate the human hand, considering that the task is hard enough provided infinite weight and size, but in this case it needs to be accomplished with a slender morphology, replicating the look and weight of its original. In systems like the Anatomically Correct Testbed (ACT) \cite{Matsuoka2006} and the commercially available Shadow robot hand \cite{shadowrobotcompany} this additional problem is clearly visible. Both of those systems meet a lot of the requirements of a human hand in terms of biological accuracy or dexterity, but do not meet the requirements for a prosthesis such as weight or size. This leads to the only option of a trade-off between the requirements so that only as little as possible of the dexterity is lost and as much as possible of the user's requirements are met. And only a successful integration of commercial components and state of the art techniques with the trade-off described above chosen appropriately can lead to a proper system. Even nowadays, the creation of an anthropomorphic prosthetic hand represents an open problem in upper limb prosthetic research \cite{TheSmartHand2011}.

To understand the complexity of the task of creating an artificial hand, it is necessary to understand several principles of the anatomy of the human hand. It is important to understand osteology, the study of the skeleton of the hand, ortology, the study of the articulations and the ligaments and myo-functionality to understand muscles, tendons and the functionality of prehensile as well as forceful movements.
%----------------------------------------------------------------------------------------

\section{Anatomical description}
The human hand is a versatile organ that is used for grasping heavy or delicate objects and for performing highly complex manipulations on the basis of fine motor control and precise sensory feedback \cite{Kapandji1982}. This can already be seen when looking at the part of the motor cortex of the brain that is responsible for its motor functionality. Compared with other areas, it has about the same size as those of all of the other extremities together \cite{Meier2008}.\\

\begin{figure}[htp]
\centering
\includegraphics[width=0.8\textwidth]{Neurology/motor-cortex-map}
\caption[Map of the motor cortex of the human brain]{Map of the motor cortex of the human brain. To show the results more clearly, the cerebral cortex is increased in size. The neural activity evoked by movements of different body parts is shown in a winner-take-all manner. Image from \cite{Meier2008}.}
\label{motor-cortex-map}
\end{figure}

This could be an indication that its control is a complicated task and its sensory and motor interface is quite big. Already at an early point in time, the complexity of the hand has been recognized by several scientists such as Wood Jones, who wrote about the hand that ''..it is not the hand that is perfect, but the whole nervous mechanism by which the movements of the hand are evoked, coordinated and controlled.'' \cite{SensoryHand} Charles Bell wrote in ''The Hand, its Mechanism and Vital Endowments as Evincing Design'': ''...how happily the hand is constructed: in which we perceive the sensibilities to changes of temperature, to touch and to motion, united to a facility of motion in the joints, for unfolding and tuming the fingers in every possible degree and direction, without abruptness or angularity, and in a manner inimitable by any artifice of joints and levers.'' and thereby honored the design of the natural hand, further he pointed out its inimitability which still holds today \cite{BellHand}.

The human hand consists of 24 muscle groups, controlled by several sensory-motor loops created from sensor and motor nerves and 27 bones, that are the insertion points for the tendons attached to the muscles. Without this complex structure of bones, tendons, muscles, and nerves, the hand would never be able to work in such a delicate manner and perform patterns of action. The bones of the hand are divided into three parts: finger, metacarpal, and carpal. From the fingertip down to the palm we find the distal phalanx, intermediate phalanx, and proximal phalanx. The bones of the wrist form the so called carpus, which consists of eight carpal bones. They allow the hand to rotate with respect to the arm and to perform gliding motions caused by the fingers. Five metacarpal bones form the middle hand and are connected to the carpal bones and present an asymmetry, with a semi-spherical surface to connect to carpal bones, but a spherical surface to connect with the first phalanx. The fingers are formed by five proximal and distal bones but contain only four middle phalangeal bones. All fingers have three joints except for the thumb, which does not have a middle phalangeal bone and therefore is missing the PIP joint. The joints can be compared to the technical types of joints (Figure \ref{joint-types}, but it is important to distinguish them.

\begin{figure}[htp]
\centering
\includegraphics[width=0.8\textwidth]{Anatomy/joint-types}
\caption[Technical joints]{Different technical joint types  which several similar natural joints are used in the hand. Adapted from \cite{Schuenke2005}.}
\label{joint-types}
\end{figure}
Technical joints imply defined axes for the joint rotations which is not completely true for bones in general. Joints between bones are connected with ligaments that hold the joints in place and therefore their rotations differ from the technical axial rotation. The distal and proximal interphalangeal joints (DIP and PIP) located in the fingers can be compared to technical hinge joints which have only one degree of freedom. The carpometacarpal joint (CMC) can be compared to a saddle joint with two degrees of freedom. Unlike the metacarpophalangeal joints (MCP) of the fingers which are also hinge joints, the thumb's MCP directly above the carpal bones is a special joint comparable to a ball joint with three degrees of freedom. This is the joint that gives the thumb its most important ability, namely to rotate around and oppose the other fingers. By flexing the thumb and approaching the other fingers, the hand is able to grab objects.\\
To actuate the hand, the bones serve as attachments for the tendons that connect the bones with the muscles. Tendons from the flexor superficial are incoming from the muscles in the forearm: From this first layer tendons, more tendons go to the middle and ringer, from the second layer tendons enter the index and little finger. After the connection to the first phalanx, the tendon divides into two and attaches to both the left and right side of the intermediate phalanx. This lets the bones move in sequence, first the proximal and intermediate, then the distal phalanx. Furthermore, other tendons originating in the forearm split up into four and are attached to the distal phalanges. They help to flex the distal phalangeal joints after the superficial tendons flexed the other interphalangeal joints. Tendons are held close to the intermediate bones by sheaths, which maintain the position of the tendon relative to the phalanxes, and thus to the line of action of the finger. The tendon sheaths can be seen in figure \ref{Tendon-sheaths}.

\begin{figure}[H]
\centering
\includegraphics[width=0.32\textwidth]{Anatomy/Tendon-sheaths}
\caption[Tendon sheaths of the hand]{The mucous sheaths of the tendons on the front of the wrist and digits. Image from \cite{Gray1918}.}
\label{Tendon-sheaths}
\end{figure}

The sheaths are responsible for the smooth action of the tendon, as the surfaces of joints and sheaths that surround tendons feature very low friction coefficients \cite{Amadio2005,Schweizer2003}. Tendons that cross joint spaces tend to flex that particular joint; a tendon that traverses many joints must therefore be able contract by an appropriate distance. The human hand is optimized for strength in flexion more than for extension. The force of flexion has been shown to be 62\% stronger than the force of extension \cite{Li2001}. That is why extending the hand in any way is not a good way to exert force since the extensors and their tendons are more compact and less powerful than the flexor and their tendons. The table \ref{Do-Table} describes the joints in terms of degrees of freedom (DoF), that are the joints that can be moved independently from the others and the degrees of actuation (DoA), which mean the actuation of the joints that is independent. As indicated, there are also underactuated joints and common actuators in the human hand.
\begin{table}[H]
\begin{adjustwidth}{-1.5em}{-1.5em}
\begin{tabular}{llllll}
\toprule
\multirow{2}{*}{\textbf{Finger}} & \multirow{2}{*}{\textbf{DoFs/DoAs}} & \multicolumn{4}{l}{Articulation}\\
&&CMC & MCP & PIP & DIP\\
\midrule
Thumb & 5/5 & Palmar abd/add / Radial abd/add & flex/ext & - & flex/ext\\
Index & 4/3 & Abd/add & flex/ext & flex/ext & flex/ext\\
Middle & 4/3 & Abd/add & flex/ext & flex/ext & flex/ext\\
Ring & 4/1* & Abd/add & flex/ext & flex/ext & \underline{flex/ext}\\
Little & 4/1* & Abd/add & flex/ext & flex/ext & \underline{flex/ext}\\
\midrule
\textbf{Total} & \textbf{21/12} & \multicolumn{4}{}\\
\end{tabular}
\caption[Table of DoF/DoA in the human hand]{
A description of the degrees of freedom and actuation in the hand and their location and function in the fingers.\\
(*Common actuator/\underline{Underactuated joint})
}
\label{Do-Table}
\end{adjustwidth}
\end{table}

Muscles, actuating the tendons, are natural actuators that are only able to pull on the tendons. That is why in the skeletal structure they are working in an antagonistic way, which means that they are mostly grouped into paired antagonists as a flexor and an extensor, or an abductor and an adductor, or a supinator and a pronator. The flexor decreases the angle between two bones, the extensor increases it. The adductor decreases the angle between the bone and the midplane, the abductor increases said angle. Supination and pronation are movements in the human body and describe for instance the rotation of the wrist and forearm around itself so that the ulna and radius are crossed or parallel.
Several ligaments located in the palm connect the metacarpal to the carpal bones. Blocking some movements of the metacarpals, it is giving the palm a more rigid structure. More ligaments interconnect the phalangeal bones.

The muscles of the hand can be categorized into intrinsic and extrinsic muscles (for a complete list, refer to Appendix \ref{AppendixC}).

\textbf{The intrinsic muscles} are located directly in the hand and mostly give the hand its precise grasps and movements such as holding a pencil and drawing or writing with it. Furthermore they assist the extrinsic muscles in producing a strong finger grip. Several muscle groups are considered as intrinsic groups: The Hypothenar muscles, the Thenar muscles, the musculi interossei (palmar interossei and dorsal interossei) and the lumbricals, which are named after their wormlike shape. 

\begin{figure}[H]
\centering
\subcaptionbox{\label{fig:hypothenar-muscles}}{\includegraphics[width=0.16\linewidth]{Anatomy/Hypothenar-muscles}}\hspace{0.15\textwidth}
\subcaptionbox{\label{fig:thenar-muscles}}{\includegraphics[width=0.16\linewidth]{Anatomy/Thenar-muscles}}\hspace{0.15\textwidth}
\subcaptionbox{\label{fig:palmar-interossei}}{\includegraphics[width=0.16\linewidth]{Anatomy/Palmar-interossei}}\\
\subcaptionbox{\label{fig:dorsal-interossei}}{\includegraphics[width=0.16\linewidth]{Anatomy/Dorsal-interossei}}\hspace{0.15\textwidth}
\subcaptionbox{\label{fig:dorsal-interossei}}{\includegraphics[width=0.16\linewidth]{Anatomy/Lumbrical-muscles}}
\caption[The five intrinsic muscle groups of the human hand]{The five intrinsic muscle groups: \textbf{A} thenar muscles, \textbf{B} hypothenar
muscles, \textbf{C} palmar interossei, \textbf{D} dorsal interossei and \textbf{E} lumbricals. Images from \cite{Reynolds2004}.}
\label{intrinsic-muscles}
\end{figure}

\paragraph{The extrinsic muscles} are muscles that are located in the forearm and can therefore be a lot longer than the intrinsic muscles. They are responsible for the hand's strength to lift and carry heavy objects. Intrinsic and extrinsic muscles combined exploit the bone structure to create levers by pulling on the tendons that insert the bones at a location that is at a short distance to the joint, thus increasing the lever. The tendons of the extrinsic muscles all lead through the wrist and are then held in place by tendon sheaths that lead them along the metacarpal bones to their points of action. They are internally made of collagen and are elastic, so they are able to return the fingers to their original position after a flexion.

\begin{figure}[H]
\centering
\subcaptionbox{\label{fig:anterior-flexor}}{\includegraphics[width=0.22\linewidth]{Anatomy/Anterior-view-of-flexors}}\hspace{0.15\textwidth}
\subcaptionbox{\label{fig:posterior-extensor}}{\includegraphics[width=0.22\linewidth]{Anatomy/Posterior-view-of-extensors}}\\

\caption[The two extrinsic muscle groups of the human hand]{The two extrinsic muscle groups: \textbf{A} Anterior view of the flexors, \textbf{B} Posterior view of the extensors. Images from \cite{Reynolds2004}.}
\label{fig:extrinsic-muscles}
\end{figure}

A muscle is formed from many muscle fibers. The muscle can be seen as an elastic component in parallel to an elastic and a contractile component which are in series. The tendons of the muscles form the elastic component as well as the connective tissue between the muscle fibers. The muscle fibers themselves are the contractile component.

The muscle shows an exponential growth in length when different loads are applied. As soon as the pulse for contraction is sent to the muscle, the fibers start shortening, elongating the tendons thus building up a smooth increase in tension. If the fibers are elongating again, the elastic parts shorten again with a certain time delay. Depending on the forces that are exerted by the muscle, the speed of the contraction changes. If the muscle is not loaded and the force is therefore nearly zero, the speed reaches its maximum and is zero when the load and thereby the force is maximum. In the hand the speed of a complete contraction requires 80 to 200ms with no load. The maximum load in turn varies from 2-5 $\frac{kg}{cm^2}$.

\begin{figure}[H]
\centering
\includegraphics[width=0.3\textwidth]{Anatomy/Muscle-structure}
\caption[Structure of skeletal muscle]{Structure of skeletal muscle at progressively higher magnification, from whole muscle to contractile proteins (A-D, F). E represents the 'sliding filaments' diagrammatically. Image from \cite{thefreedictionary}.}
\label{hand-anthropometric}
\end{figure}

Kapandji divides the human hand into three functional components \cite{Kapandji1982} (cf. numbers in Fig. \ref{human-hand-structure}):
\begin{itemize}
\itemsep0.1em
\item The thumb, which gives the hand most of its functions by being able to oppose the other fingers;
\item The index and the middle finger, which support the thumb to achieve delicate movements and precision grips;
\item The ring and the little fingers, which, combined with the other fingers and the palm, give the hand its capability to grasp objects solidly and support the hand in strong grasps.
\end{itemize}

\begin{figure}[H]
\centering
\includegraphics[width=0.37\textwidth]{Anatomy/human-hand-structure}
\caption[Kapandji et al.'s functional hand division]{Functional division of the hand into the three components as Kapandji et al. proposed. The indication at each joint indicates the DoF of it. Image from \cite{Kapandji1982}}
\label{human-hand-structure}
\end{figure}

The thumb plays a unique role in the function of the hand, being essential for the formation of the pollici-digital pincers and for the development of a powerful grip along with other fingers. In other words, thanks to the thumb, the human hand is able to perform both power grasp and precision grasp, achieving high dexterity and versatility. Thus, without the thumb, the hand loses most of its capabilities. This preeminent role of the thumb is partly due to its location anterior to the palm and the other fingers, which allows the thumb to move towards the fingers individually or together (the movement of opposition) and away from them (the movement of counterposition). Geometrically speaking, opposition of the thumb consists of bringing into contact the pulp of the thumb and the other finger so that they touch. In other words, the tangential planes of the two pulps merge in space at a single point. Five DoFs in three joints are used to achieve opposition and dexterity of the thumb (cf. Fig. \ref{human-hand-structure}): two DoFs are listed for the trapezometacarpal joint or carpometacarpal joint (TM joint or CMC joint), two more DoFs are listed for the metacarpophalangeal (MCP) joint and a last one is allowed by the interphalangeal joint (IP joint).

\paragraph{Control of the hand}
The human hand control system is formed by the brain and the spinal cord which together compose the nervous system. The control is distributed into many centers as it is the case in many vertebrates. The brain's motor centers, as well as the peripheral receptors that are responsible for the reflexes regulate the muscle activity by electrical pulses. The reflex arc, being one of the simplest biological control systems, does not involve encephalon activity at all. A simple reflex response takes between 0.5 and 1.5ms \cite{Folgheraiter2000}. A reflex can be inhibited by other neurons, so for instance the extension of the finger if the finger is flexed. Each reflex of the reflex arc is defined by its response time, the area where the stimulated receptors are located, the threshold for its activation and its ability to summate stimuli in intensity and in space.

Reflexes are proprioceptive as well as esteroceptive, according to the receptors involved. The myotatic reflex, originating from the neuro-muscular fibers, can be seen as the most important proprioceptive reflex. The first phase of this reflex triggers a rapid contraction of the muscle, the second phase stabilizes the muscle to the given length to hold the position. The inverse reflex to the myotatic reflex starts from the Golgi organs, transferred to the spinal center to relax the target muscle by inhibiting its motor neurons to maintain a safe strain on the tendon.

\paragraph{Sensory capabilities}
Esteroceptors as well as proprioceptors present in the hand are organized in parallel as well as in series to the muscle fibers. They continuously send information even though most of it is not perceived but just gives information for the movement of the fingers. The transducers are mechanical, electrical, photo-, and termo-electrical.


\begin{figure}[htp]
\centering
\includegraphics[width=0.7\textwidth]{Anatomy/golgi-muscle-spindle}
\caption[Muscle spindle and golgi apparatus]{Muscle and tendon elongation sensory system. Muscle fibers contain golgi organs that are connected in series and in parallel to the muscle to measure its movement. Image from \cite{Folgheraiter2000}}
\label{golgi-muscle-spindle}
\end{figure}

This knowledge of structure and control of the human hand is of fundamental importance to prosthetic engineers, as it is the model for the mechanically constructed prosthesis that will mimic the natural hand's appearance as well as its functionality. Further it provides a possibility to technically compare the natural human hand with its artificial counterparts. For the moment, aiming for the mechanical and functional complexity of the natural hand would overshoot the current target because of the limited performance offered by current (even by the sophisticated) user-prosthesis interfaces available for control \cite{Tenore2009,Rossini2010}. Nevertheless it is clear that for a next generation of thought-controlled prostheses with multi-sensory feedback, more degrees of freedom and degrees of actuation are needed in order to perform optimized, stable grasps, dexterous manipulation tasks and, in general, mimic the functionality and behavior of the human hand.
%----------------------------------------------------------------------------------------

\section{Motivation}
Even though the research field for robotic prostheses has been existing for a long time already and has made several advancements towards functional hand replacements for amputees, the hand prostheses did not change much during the last decade. Advancements include a higher number of degrees of freedom, which are possible due to technical developments in size, weight and type of motors \cite{Kyberd2001,DelCura2003,Pons2002}. Other commercial advancements in lightweight lithium batteries \cite{Williams2005} include an increase of battery life or improvements in performance of hardware and software. Despite those advancements attained, approximately 20-40\% of users of myoelectric prostheses reject their prosthetic limbs sooner or later \cite{abandonment1}. A study called \citetitle{abandonment2} by Elaine A. Biddis et al. mentions several reasons such as that users are just as or more functional without it, they feel more comfortable without it, that the prosthesis is too difficult or tiring to use, that it is too heavy, the high costs of such a device or that they have more sensory feedback without it. Further mentioned are a dissatisfaction with the prosthetic technology or the appearance of it, medical factors such as skin irritation or that it needs to be removed for several activities such as sleeping or swimming \cite{abandonment2}.

Especially the weight of an hand prosthetic device should be as little as possible. Since most of the prostheses of today are driven by motors, this restriction significantly reduces the number of servo motors on the device and therefore restricts the dexterity of it. Also the output force of the hand is significantly reduced as most methods to circumvent the restriction by using a reduced number of stronger motors and simultaneously actuate the fingers. Another method considers large and/or non back-drivable gear ratios on lighter motors to keep the hand closed when lifting a heavy weight. The second method's drawback is that the lack of back-drivability suggests that the prosthesis is rigid and unnatural. This can be dangerous in certain circumstances if the user can only close the hand on an object but is not able to release it. Additionally the gear ratio prescribes the finger closing speed. Since gears reduce the speed of rotation to create a higher force, this leads to a low finger speed which is not desirable. Pneumatics on the other hand can serve force as well as speed without using energy if the actuators are not moved. 

Therefore we hypothesize that we can use a tendon-driven embedded pneumatic system to improve the prosthetic hand weight by shifting the weight of the actuators to a wearable belt or bag that houses a pneumatic compressor as well as a tank to store pressurized air. Tendon-driven pneumatic systems are favorable in this regard as they have a distinct mass separation between the actuators, the transmission and the compressor. Even though the compressor is in fact the actuator of the complete system, it can be run only at certain times to refill the tank and it can be located remotely to reduce the weight of the prosthesis. As actuators, pneumatic artificial muscles are used which are also favorable to reduce weight and increase the output force of the system. They have a force to weight ratio up to about 1800:1, giving the hand enough force to perform ADLs as long as the hand is powered by pressure. These and other facts can help to reduce the weight of an actuated prosthetic device, while at the same time maintain versatility, force and speed on the device. Towards such an aim, we will design a pneumatic prosthetic hand system that takes into account the constraints discussed above and further discussed in chapter \ref{Chapter3}.
%----------------------------------------------------------------------------------------

\section{Approach}
The present thesis documents the development of a portable pneumatic system for prosthetic hands. Most myoelectric prostheses of today use servo-motors to drive the joints of the hand and wrist. Our approach aims at the creation of a system that is portable, small, low-energy, low-noise and lightweight, but at the same time provides a high dexterity and precision of movement as well as enough output force to perform most common activities of daily living (ADLs). The pneumatic system exploits the fact that with a tendon driven system there is a distinctive separation of mass of the actuators, the transmission and the compressor. Even though strictly speaking the compressor is the actual actuator, it can be moved to a belt and can even be turned off when combined with a small tank used as a reservoir. Furthermore the muscles are similar to biological muscles in the sense that they add some compliance to the grasp of the hand because they are not entirely rigid in position. The air inside is still compressible and therefore the muscle can give the hand a soft or hard grip depending on the stiffness of the muscle. Even though especially the force control of the muscles is really difficult to achieve, the approach is definitely worth a try since the force output of a pneumatic muscle is very high compared to its weight. Force is a crucial component to a pneumatic hand -- force finally gives the user the ability to manipulate object with an actuated hand -- the integration of a pneumatic system could definitely provide a new way to actuate prosthetic devices.

%----------------------------------------------------------------------------------------

\section{Outline}
The portable pneumatic system for the prosthetic hand was created by first identifying previous creations of non-pneumatic as well as pneumatic artificial hands to find their features, advantages and disadvantages that can be used for a comparison with our system as soon as it is of comparable complexity. Furthermore, the causes for prosthesis abandonment were studied to recognize the drawbacks and inconveniences of today's prostheses so that they can be reduced in the proposed system. \textbf{Chapter \ref{Chapter2}} describes the results of this literature review and documents important facts for the upcoming chapters. The main constraints for a prosthetic hand are described in \textbf{Chapter \ref{Chapter3}}. They deserve their own chapter because they greatly influence the creation of such a system in terms of how precise the finger movement will be as well as the strength of the hand, the system's lifetime or the ease of use as well as many more properties of the device. Furthermore, facts about the human hand system were studied to take specifications of speed and force as desired speed and force output for the system. \textbf{Chapter \ref{Chapter4}} documents the evaluation of appropriate components to meet the prosthetic requirements. Before the system was implemented into a first prototype, two muscles were attached to one single finger as antagonistic triarticulate muscles. A pneumatic finger platform was created to tune the finger's position control and to run experiments on it and to test several control strategies. Then several experiments on the pneumatic muscle strength as well as its control with several types of valves were conducted on the said platform. A further experiment run on this platform was a weight experiment on one finger to test the strength of the pneumatic muscle when used on a pneumatic finger. Furthermore the speed of the finger was tested to see if it could compete with a human finger and if it could also be used for a neuroscientific experiment proposed by neuroscientists from the ETH Zürich to use it as a model for the stiffness behavior of natural muscles and to find causes for parkinson disease. Finally several components were built onto an arm-attachment that can be used to fixate it on a human arm. It was made to actuate a human bone hand by controlling pneumatic muscles that pull on tendons to exert forces on the hand's fingers. This platform was not completed but showed a proof-of-concept design that could easily be extended to a full hand actuation. A small lightweight compressor was attached to a tank and run in a closed-loop program that ensures that the tank is always pressurized up to a certain pressure threshold. Then the tank was attached to the pneumatic valves of the system and finally the control of the valves was refined for the usage in a pneumatic prosthetic system. The findings of all these experiments are presented in \textbf{Chapter \ref{Chapter5}} to show the outcome of the assembled system. Since several other prosthetic and non-prosthetic systems were studied in preceding chapters, the results of the experiments were then compared to other known systems to verify its competitiveness.In \textbf{Chapter \ref{Chapter6}}, a conclusion based on the results was drawn, achievements were discussed and other possible projects are outlined.\\ Further the thesis contains a manual on how to make pneumatic artificial muscles with an integrated pressure sensor (\textbf{Appendix \ref{AppendixA}}). It includes pictures and important details on how to calculate the proper dimensions of the components to achieve the required measures such as force and contraction travel distance. Further it contains details on the historical context of McKibben artificial muscles and up to the current development. In \textbf{Appendix \ref{AppendixB}} a listing of robot hands with their measures is presented that are used in the comparison to the specifications in Chapter \ref{Chapter5}. \textbf{Appendix \ref{AppendixC}} contains a listing of the human hand muscles, described with their major function on the hand and the antagonists of the muscle.

%----------------------------------------------------------------------------------------
\end{document}