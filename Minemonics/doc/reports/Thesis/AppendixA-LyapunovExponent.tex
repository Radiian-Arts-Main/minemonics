% Appendix A

\documentclass[main]{subfiles}

\begin{document}
\chapter{} % Main appendix title

\label{AppendixA} % Change X to a consecutive letter; for referencing this appendix elsewhere, use \ref{AppendixX}

\lhead{Appendix A. \emph{Using the Largest Lyapunov Exponent as a measure for gait stability}} % Change X to a consecutive letter; this is for the header on each page - perhaps a shortened title
The divergence of gait performance can be measured using the Largest Lyapunov exponent, the mean rate of separation of trajectories, and calculated with the following form adapted from \cite{bib:Rosenstein1993}. %
%
Given the chaotic system in the form below, so that the iterated function applied \(t\) times to \(x_0\) results in \(x(t)\), the same holds for a perturbance \(\delta x_0\) which after t iterations occurs as \(\delta x(t)\).

\begin{align*}
x(t) = f^t(x_0)\\
x(t) + \delta x(t) = f^t(x_0 + \delta x_0)\\
\end{align*}

To measure the sensitivity to the initial conditions, the evolution of the perturbance \(\delta x_0\) through the system can be calculated:

\[||\delta x(t)|| = \||\delta x_0|| e^{\mu t}\]

The \(\mu\) in that case is also called the Largest Lyapunov exponent (LLE).%
%
If \(\mu > 0\), then the trajectory diverges exponentially. %
%
A \(\mu = 0\) means, that the system is in a critical state, because its trajectories neither diverge nor converge. %
%
However, if \(\mu < 0\), it expresses the convergence of a trajectory to a stable point or orbit. %
%
The quantification would be done by plotting the Log divergence distance ratio:

\[\log\left(\frac{||\delta x(t)||}{||\delta x_0||}\right) = \mu t\]
%
The LLE can then be found by line fitting segments of the time evolution log plot of the trajectory differences and then measuring the slope of the lines. %

\end{document}
