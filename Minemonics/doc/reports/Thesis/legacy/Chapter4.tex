% Chapter 4
\documentclass[main]{subfiles}
\setcounter{chapter}{3}
\begin{document}
\chapter{Hand design} % Main chapter title

\label{Chapter4} % For referencing the chapter elsewhere, use \ref{Chapter1} 

\lhead{Chapter 4. \emph{Hand design}} % This is for the header on each page - perhaps a shortened title

This chapter is structured into three sections. The first section documents the choice of the pneumatic actuator that will be used in the prototype. The features of the actuators are listed briefly and finally compared to the requirements and constraints of chapter \ref{Chapter3}.  The second section describes the six main components of a pneumatic prosthetic hand. Starting from energy storage, the pressure generation and storage will be discussed. Further the controller will be presented which runs a program to ensure the continuous generation of pressurized air and its persistent storage and further to control its distribution. Then the pressure-to-force transformation is lined out and the section ends with the description of how the force is distributed to the hand. The last section describes the prototypes that were built to test our hypothesis and run experiments to extract features for prosthetic use.
%----------------------------------------------------------------------------------------
%Hand design
% Pressure Generation
%  Determination of the battery
%  Determination of the compressor & tank system
%  Calculation of the tank size
% Pressure Distribution
%  Choice of the valve(comparison of valves)
%  Valve labor testing(Lee products)
% Pressure Transformation
%  Creation of pneumatic actuators
%  Ratio between inner and outer tube
%  Muscle diameter decision
%  Muscle length determination
%  Muscle strength diagrams
%  Muscle volume calculation
% Force Distribution
%  Formula for the pulley length
% Energy storage
%  Choice of the battery
% Control system
%  Choice of the micro controller
%  Control using a PID

\section{Why not electric motors?}

Electric motors are probably the most commonly used actuation device in modern day prosthetics. A wide range of motors are already available, their characteristics are well known and they are very easy to control by just connecting a controller board to a voltage amplifier, which then drives the motor. But the electric motor has several disadvantages which are especially unpleasant for prosthetic applications. Most importantly, the force-to-weight ratio is far lower than that of hydraulic and pneumatic devices. Typical electric systems have a power-to-weight ratio of around $50-100 \frac{W}{kg}$ (including a gear reducer), whereas fluidic systems produce $100-200 \frac{W}{kg}$ (including a valve and an actuator) \cite{Binnard1995}. Biological muscles for a comparison produce around $40-250  \frac{W}{kg}$ \cite{Davis2001}. Further, when designing a prosthetic hand, one has to consider that the rotary movement of the electric motor can cause complications compared to the linear movement of a muscle.

\section{Why pneumatics?}

The greatest advantage of the human hand over the mechanical hand is that the biological muscle has a very desirable force-to-weight ratio and requires a low energy level to be activated. The muscle can be tuned to either stiff or compliant in a continuous manner which are favorable to reproduce in the mechanical prosthetic hand. Pneumatics in general can provide a certain tunable compliance, which can be used as an advantage instead of defining it as a lack of precision as it is mostly done in control engineering. Therefore it could be interesting to look at different pneumatic actuators to choose one which meets the requirements and constraints.

\section{Pneumatic actuator types}

One of the most important decision for a pneumatic actuation system for a prosthetic hand has to be made in the initiation of this chapter. This is because the type of pneumatic actuator determines several other components which it depends on. To transform pressure into a mechanical force and perform a movement with it, several types of actuators are existing with their advantages and disadvantages. The chosen actuator has to meet the prosthetic requirements and constrains in terms of force output, weight, size, noise, durability etc. Therefore, a brief description of the pneumatic actuators is presented below to determine the best actuator for the prosthetic actuation system.

\subsection{Pneumatic cylinders}

\begin{figure}[H]
\centering
\includegraphics[width=0.5\textwidth]{Design/pneumatic-cylinder}
\caption[A double-acting pneumatic cylinder]{A double-acting pneumatic cylinder from FESTO. Image from \cite{cylinder}.}
\label{pneumatic-cylinder}
\end{figure}

Pneumatic cylinders eliminate some of the problems associated with electric motors \cite{Song1989}. They are mechanical devices, that produce a force in a reciprocating linear motion by using a compressed gas \cite{Majumdar1995}. A piston inside of the pneumatic cylinder with the area A ideally converts the pressure P into the force F in the manner $F= P \cdot A$ \cite{Majumdar1995}. A pneumatic cylinder can be single or double acted, meaning that it either has just one inlet on one side to push the piston in one direction, while an inner spring pushes it into its initial state or it has two inlets, one pushing the piston in one direction and the second into the opposite direction. The pneumatic cylinder using air as its driving gas provides a certain compliance caused by the compressibility of the air, while this in turn affects the positioning precision. Further they can easily be used to move from one extreme to the other with mechanical stops to halt motion, whereas precise position control is very difficult to achieve.

Pneumatic cylinders have a high force-to-weight ratio which makes them desirable in terms of force output \cite{Hollerbach1992}. They provide an air-compression based compliance, but do not meet the constrains of weight and size because they are made from metal and because they are not built in arbitrary sizes and therefore make it difficult to fit them into a prototype system.


\subsection{Pneumatic bellows}

\begin{figure}[H]
\centering
\includegraphics[width=0.5\textwidth]{Hands/Fluidhand-bellows}
\caption[Fluidic bellows]{The fluidic bellows depicted here are manufactured for the fluidhand, but are not commercially available as such. Image from \cite{Gaiser2009}.}
\label{Fluidic-bellow}
\end{figure}

Pneumatic bellows are an old concept when it comes to blow air into a fire, but are a relatively new invention to actuate machines and especially robots and prostheses. They generate a rotary or linear movement from pressurized air and are used for example by the company FESTO in the learning gripper system \cite{learning-gripper}. Similar ones driven by a fluid are used to actuate the Fluidhand I-III \cite{Gaiser2009}. The so called flexible fluidic actuators transform the pressure directly into a rotary movement. They have a very good power to weight ratio and a high-pressure stability and they provide an inherent compliant actuation. Built from natural rubber, fiber preforms and aluminium, they guarantee a high number of lifecycles. For an indepth study, the interested reader can refer to the paper by De Greef et al. presenting a profound overview over the functionality and the recent advancements \cite{DeGreef2009}.

The difficulty in the fabrication of such bellows is that they must withstand a high pressure. Otherwise they do not exert enough force to move a finger. Further they exert an expansion force if pressure is applied and can not be used to exert a contraction force. The other option that exerts a rotary force could be used for an intrinsic muscle, but not for the force output comparable of an extrinsic muscle meaning that the hand could not be very forceful.

\subsection{McKibben Pneumatic Artificial Muscles (PAM)}

\begin{figure}[H]
\centering
\includegraphics[width=0.5\textwidth]{Design/PAM}
\caption[Pneumatic Artificial Muscles]{McKibben-style Pneumatic Artificial Muscles built from cheap resources. They can be built in any necessary size and diameter. Image from \cite{PAMInstructables}.}
\label{PAM}
\end{figure}

Pneumatic artificial muscles provide several advantages over conventional pneumatic actuators, further they share several characteristics with the biological muscles. They were patented in 1957 by Gaylord and used by McKibben in orthotic devices \cite{Nickel1963}. They consist of an inflatable bladder around which an expandable braided fiber is wrapped. PAMs exert an axial contraction force by expanding radially and can only be used in an antagonistic pair by either combining two muscles or a muscle and spring. The best-known example of an antagonistic setup is the combination of biceps and triceps in the human arm. When the biceps contracts and the triceps is relaxed, the arm flexes. When the triceps contracts and the biceps is relaxed, the arm extends. But this setup has further properties:  when both biceps and triceps contract at the same time with a net force of zero, the elbow joint can be stiffened, if both are relaxed, the arm becomes very compliant. In the first case, the joint will thereby not move, but will be more resistant to perturbations. In the second case it is very adaptive to environmental influences. From a practical standpoint,the biologically inspired concept of an antagonistic setup that the joint can be moved independently to several positions and obtains adaptable compliance \cite{VanHam2009}. The output force depends on the applied pressure and on the ratio between the distance of radial contraction and the length of the muscle. The more the muscle contracts, the less the output force. The biggest disadvantage of PAMs is the displeasing fatigue life. The muscle mostly ruptures after around 10'000 cycles. The Pneumatic Artificial Muscle was commercialized by the company FESTO under the name fluidic muscles \cite{fluidic-muscles} and was improved in its lifetime to around 10'000'000 cycles by integrating the fiber mesh into the expandable bladder. Unfortunately, FESTO provides only very large muscles. 

Their contraction behavior is highly non-linear, but is very similar to biological muscles in terms of force output, contraction speed and compliance. They have a very high force to weight ratio which is even higher than the one of pneumatic cylinders because is significantly lighter than the conventional air cylinder; however, the pneumatic artificial muscle is capable of producing greater forces (and thus possesses a much higher force-to-weight ratio) than its heavier counterpart \cite{Lightner2002}. The contraction of the PAM produces a force-length curve akin to the rising phase of biological muscles \cite{Klute1999}. Moreover, the PAM is not able to exert force if fully contracted, which self-limits the muscle by its nature similar to the biological muscle. The pneumatic actuator running on a non-stable controller is therefore much safer than the electric motor because it can not flex a joint further than its natural threshold and therefore can not destroy the structure it actuates.

The most interesting fact about McKibben Pneumatic Artificial Muscles is that they can be hand-made according to the requirements of a certain application. This fact makes it possible to manufacture muscles that perfectly meet the requirements and constrains. Further, the components to build a PAM are very cheap.
%----------------------------------------------------------------------------------------

\subsection{Conclusion}

Based on the actuator specifications presented above and absence of prosthetic hand prototypes that make use of McKibben Pneumatic Artificial Muscles (PAM), it was chosen to be the best actuator since its weight is significantly lower than the pneumatic cylinder while providing similar output force. The bellows were not chosen because it could not be achieved to build them in a cheap and robust manner. Since the McKibben muscles can be customized very easily by modifying the manufacturing process and also the fact that they can be hand made easily in custom sizes and diameters, they were studied in depth and optimized to be fitting onto a prosthetic limb. The only drawback that will have to be dealt with is the high non-linearity of their contraction, which makes it difficult to control their position and their force output.

\section{The six main components of a pneumatic prosthetic hand}

This section describes the six main components of a pneumatic prosthetic hand. The first subsection describes the accurate power source for the system which matches the constraints and the requirements for a prosthetic hand. The second subsection presents the control system of the prosthesis and describes the features of the controller in detail. The third subsection covers the pressure generation of the system, that is where electric energy is transformed into air pressure and is stored into a tank. The fourth subsection covers the pressure distribution by valves to control the air pressure which will be used later to contract the muscles in an proportional antagonistic manner. The fifth subsection covers the pressure transformation in terms of what is important for the pneumatic artificial muscle(PAM and its required specifications to meet the prosthetic requirements. The sixth subsection section shows how the physical forces of the PAMs are distributed to actuate the  interphalangeal(ICP) and metacarpophalangeal(MCP) joints of the hand.

\subsection{Energy storage}

When looking at today's prostheses, one may realize that all of the actuated ones begin with electrical energy. The main reason for this is that energy storage as electrical energy with the aid of batteries is not only a very old concept providing a secure type of energy retrieval, but also the vast amount of different batteries for different purposes helps to choose this concept over all the others. The disadvantage of batteries is their low energy density compared to inflammable fluids \cite{energydensity} such as diesel or kerosene, materials such as coal, glucose, boron or gases such as methane or hydrogen. Even though several other sources have a higher energy density ($\frac{J}{kg}$) or a higher specific energy ($\frac{J}{L}$), most of the other sources would generate emissions as well as noise and heat which is not really wished for this purpose. Furthermore, the security issues of other sources could be very unpredictable if used in several environments and it could be dangerous for the patient to wear such a prosthesis in case of an electrical malfunction that could enlighten a leaking fluid that could cause explosions and thereby severe injuries. Finally, energy conversion is another factor that has to be considered at which batteries perform very well compared to other sources.  Even in a pneumatic prosthesis, it is reasonable to begin with electrical energy because up to date all of the control components such as the microcontrollers as well as the pneumatic valves use electricity to run and perform their tasks.

%----------------------------------------------------------------------------------------

\subsection{Choice of the battery}
In the prosthetic prototype presented here, the only energy source in the beginning is the battery. Because at first the pressure tank will be empty and no pressure potential can be used, the battery supplies the energy to start the compressor of the system to raise the pressure inside of the tank. The battery is furthermore used to supply energy to the controller and the valves.

Most satisfying would be that the battery lasts at least up to 8 hours or longer so that the patient can go to sleep while the battery is recharging. Otherwise it should at least be possible to replace the battery easily. The battery should be as lightweight and as small as possible to fit to the lower arm as it is done in most servo-driven prostheses. The battery should be at least small enough to fit onto a belt or into a backpack.

In the Aeromanus hand's first prototype, it was not necessary to make this choice since the system is not yet portable enough to require a portable power supply. The system will be powered by a power supply giving the option to control limit the voltage and current flow which is necessary to measure the necessary energy to run the prototype. Based on the measured values, the prototype can be optimized in terms of energy and the decision on the battery can be made at a later stage as soon as the system is working in a more portable manner.

%----------------------------------------------------------------------------------------

\section{Control \& sensory system}

The central component of a prosthesis is the control system. Even though the morphology of the system is very important because it facilitates the manipulation of the environment, the controller together with the sensors and human interface determines the actual capabilities of the system. The controller performs several tasks at once such as running the compressor to refill the tank or calculating the next movement for the fingers. Most actuated prostheses use myoelectric sensors which take up low-voltage energy spikes from the surface of the skin that are generated by the nerves near the muscles. From these spikes it determines the actuation of the prosthesis depending on the position of the detecting sensor. The next task for a single degree of freedom prosthesis such as a simple gripper would be to move the actuators according to the desired grip strength determined by the user generated signal. As described in chapter \ref{Chapter3}, to cover most of the manipulations of active daily living, there are at least three to six different grasps necessary, which in turn require a higher amount of degrees of freedom. Once the desired grasp patterns are defined and implemented, the user can choose from the repertoire of grasp patterns and choose it by an appropriate signal on the EMG electrodes and afterwards can control the finger closing in the chosen grasping pattern with the procedure described for the single degree of freedom. The control of the actuators can also be done by dedicated controller boards (such as PID controllers) to improve the stability of the controlling as well as to outsource tasks from the main controller.

\begin{comment}A big problem that has to be tackled is the big lack of sensory interface of all prosthetic devices up to date. Several amputees feel pain in the stump itself so that a signal can not be safely retrieved. Recent studies \cite{Kuiken2007,targeted-muscle-innervation} propose the reinnervation of nerves into the tissue of the chest of the amputee to make them feel on the chest's skin as well as motor-nerve reinnervation to create artificial residual muscles to extend the prosthesis control capabilities of the user.
\end{comment}

%----------------------------------------------------------------------------------------

\subsection{Choice of the microcontroller}
For the initial experiments to determine the strength of the pneumatic actuator, an arduino Mega board was used \cite{Arduino}. The Arduino Mega is a multipurpose controller based on the ATmega1280 with a big user community and features ease of use with an simplified programming environment to program the on-board Atmega micro-controller. The program that is run on it implements a framework for a PID controlled hand that provides classes to combine and control the joints of the hand with different types of valves such as proportional valves that require direct proportional PID aggregated voltage input or on-off valves with the same PID signal but then further processed to generate a pulse width modulation signal (PWM).

\begin{figure}[htp]
\centering
\includegraphics[width=0.8\textwidth]{Design/Classes}
\caption[Controller classes]{A brief overview over the controller classes and their relations to the others.}
\label{code-classes.}
\end{figure}

Figure \ref{code-classes} shows a brief overview over the controller classes. The diagram shows the composition of the classes on one hand, further the inheritance relations. The aeromanus controller consists of a model of the hardware, therefore containing of the air tank and the hand. The air tank has a linear sensor which as a hardware component is a pressure sensor to measure the current pressure inside of the tank. The hand has 4 fingers and 1 thumb, finger consisting from 3 joints, the thumb consisting of 2. Each joint is driven by two actuators, each having a linear-sensor (either pressure or displacement sensor to measure the contraction of the muscle) and 2 PID classes, which control the position and the force of the muscles. Further the actuator runs an air controller that abstracts the control of the air inside of the muscle. The air controller supports the inflation and deflation of the muscles with a 3-way valve either proportional or on-off. Moreover it is possible to run the muscle with 2-way valves, one for inflation and deflation respectively. Also the 2-way valves can be of proportional or on-off type.
%----------------------------------------------------------------------------------------

\subsection{Low-level control}

For a pneumatic prosthesis, the size of every component plays an important role. The pneumatic muscles of the device is driven by the valves, whose flow and maximum pressure property are of utmost importance to the precise control of the muscle. The size constraint reduces the choice of the valve to particularly small valves, which are generally at most flow controlling valves. That means that the pressure has to be controlled by a closed-loop control program that increases and decreases the flow depending on the deviation from the desired pressure. The flow control of proportional valves is normally not completely linear, therefore the resulting system has several stages of non-linearity, as the pneumatic muscles are highly non-linear as well. Since proportional valves can be quite bulky and definitely more expensive than on-off valves, the latter type is better to meet the price constraint. Even though it remains only one of the stages of non-linearity, controlling such a system is a very difficult task and there are several strategies to address the problem. A static strategy is to try to plot the non-linear behavior of all the components separately by using a linear function on the input of the component and plot the function at the output by a sensor. Then the inverse functions are generated by creating Newton polynomials of the plotted output values.

\begin{figure}[htp]
\centering
\includegraphics[width=0.4\textwidth]{Design/polynomial}
\caption[Concept of an interpolation based on polynomials]{This image shows for four points ((-9, 5), (-4, 2), (-1, -2), (7, 9)), the (cubic) interpolation polynomial $L(x)$ (in black), which is the sum of the scaled basis polynomials $y_0\ell_0(x)$, $y_1\ell_1(x)$, $y_2\ell_2(x)$ and $y_3\ell_3(x)$.The interpolation polynomial passes through all four control points, and each scaled basis polynomial passes through its respective control point and is 0 where x corresponds to the other three control points. Diagram by \cite{Polynomial}}.
\label{polynomial}
\end{figure}

\begin{comment}
\paragraph{Lagrange polynomials}
Using the Lagrange polynomials, the approximation of the control function would be built the following:
Given a set of $k+ 1$ value pairs of points which are measured from the internal non-linear contraction function of the muscle $(x_0, y_0),\ldots,(x_j, y_j),\ldots,(x_k, y_k)$ where no two $x_j$ are the same, we try to approximate the internal function by using Lagrange polynomials. The interpolation is a linear combination:
\[
\begin{aligned}
L(x) &:= \sum_{j=0}^{k} y_j \ell_j(x)
\end{aligned}
\]
of Lagrange basis polynomials:
\[
\begin{aligned}
\ell_j(x) &:= \prod_{\begin{smallmatrix}0\le m\le k\\ m\neq j\end{smallmatrix}} \frac{x-x_m}{x_j-x_m} = \frac{(x-x_0)}{(x_j-x_0)} \cdots \frac{(x-x_{j-1})}{(x_j-x_{j-1})} \frac{(x-x_{j+1})}{(x_j-x_{j+1})} \cdots \frac{(x-x_k)}{(x_j-x_k)},
\end{aligned}
\]
where $0\le j\le k$. Note how, given the initial assumption that no two $x_i$ are the same, $x_j - x_m \neq 0$, so this expression is always well-defined.

For all $j\neq i$, $\ell_j(x)$ includes the term $(x-x_i)$ in the numerator, so the whole product will be zero at $x=x_i$:
\[
\ell_{j\ne i}(x_i) = \prod_{m\neq j} \frac{x_i-x_m}{x_j-x_m} = \frac{(x_i-x_0)}{(x_j-x_0)} \cdots \frac{(x_i-x_i)}{(x_j-x_i)} \cdots \frac{(x_i-x_k)}{(x_j-x_k)} = 0\]

On the other hand:

\[
\ell_i(x_i) := \prod_{m\neq i} \frac{x_i-x_m}{x_i-x_m} = 1
\]

In other words, all basis polynomials are zero at $x=x_i$, except $\ell_i(x)$, for which it holds that $\ell_i(x_i)= 1$, because it lacks the $(x-x_i)$ term.

It follows that $y_i \ell_i(x_i)=y_i$, so at each point $x_i$, $L(x_i)=y_i+0+0+\dots +0=y_i$, showing that $L$ interpolates the function exactly.

\end{comment}
\paragraph{Newton Polynomials}
Given a set of k+1 value pairs of points which are measured from the internal non-linear function of the muscle $(x_0, y_0),\ldots,(x_j, y_j),\ldots,(x_k, y_k)$ where no two $x_j$ are the same, the interpolation polynomial in the Newton form is a linear combination of Newton basis polynomials:
\[N(x) := \sum_{j=0}^{k} a_{j} n_{j}(x)\]

of Newton basis polynomials:
\[
n_j(x) := \prod_{i=0}^{j-1} (x - x_i)
\]

for $j \ge 0$ and $n_0(x) = 1$.

The coefficients are defined as $a_j := [y_0,\ldots,y_j]$ where $[y_0,\ldots,y_j]$ is the notation for divided differences. Thus the Newton polynomial can be written as:

\[
N(x) = [y_0] + [y_0,y_1](x-x_0) + \cdots + [y_0,\ldots,y_k](x-x_0)(x-x_1)\cdots(x-x_{k-1}).
\]

\begin{comment}
The Newton Polynomial above can be expressed in a simplified form when $x_0, x_1, \dots, x_k$ are arranged consecutively with equal space. Introducing the notation $h = x_{i+1}-x_i$ for each $i=0,1,\dots,k-1$ and $x=x_0+sh<$, the difference $x-x_i$ can be written as $(s-i)h$. So the Newton Polynomial above becomes:

\[
\begin{aligned}
N(x) &= [y_0] + [y_0,y_1]sh + \cdots + [y_0,\ldots,y_k] s (s-1) \cdots (s-k+1){h}^{k} \\
&= \sum_{i=0}^{k}s(s-1) \cdots (s-i+1){h}^{i}[y_0,\ldots,y_i] \\
&= \sum_{i=0}^{k}{s \choose i}i!{h}^{i}[y_0,\ldots,y_i]
\end{aligned}
\]

is called the Newton Forward Divided Difference Formula.

If the nodes are reordered as ${x}_{k},{x}_{k-1},\dots,{x}_{0}$, the Newton Polynomial becomes:

\[
N(x)=[y_k]+[{y}_{k}, {y}_{k-1}](x-{x}_{k})+\cdots+[{y}_{k},\ldots,{y}_{0}](x-{x}_{k})(x-{x}_{k-1})\cdots(x-{x}_{1})
\]

If ${x}_{k},\;{x}_{k-1},\;\dots,\;{x}_{0}$ are equally spaced with $x={x}_{k}+sh$ and ${x}_{i}={x}_{k}-(k-i)h$ for $i = 0, 1, \ldots, k$, then:

\[
\begin{aligned}
N(x) &= [{y}_{k}]+ [{y}_{k}, {y}_{k-1}]sh+\cdots+[{y}_{k},\ldots,{y}_{0}]s(s+1)\cdots(s+k-1){h}^{k} \\
&=\sum_{i=0}^{k}{(-1)}^{i}{-s \choose i}i!{h}^{i}[{y}_{k},\ldots,{y}_{k-i}]
\end{aligned}
\]

is called the Newton Backward Divided Difference Formula.
\end{comment}

The inverse functions can be generated by exchanging the x and y values of the measured internal non-linear function of the muscle. The function can then be used to calculate the input to the desired output.

\paragraph{Conclusion}
Measuring the non-linear internal functions of each component and precalculating the coefficients of the polynomial, then nesting the functions according to the direction of component usage in the device. For example if we have a non-linear valve and muscle, we nest the valve function into the muscle function. The resulting function gives the controller the ability to calculate the exact control value for a desired position of the muscle. In case of an ideal non-linear system that would not change during the usage of the muscle, the approximated polynomials would be sufficient to linearize the muscle to a further extent. The main disadvantage of this method is that it can not react to disturbances of the system such as the touching of an object with the muscle driven finger, which changes the real position of the finger. It is also the case that a pneumatic muscle can change its contraction behavior over time due to hysteresis or when contracted and expanded quickly because of friction-related thermal dissipation inside of the muscle. In a prosthesis, it is of high importance that the hand does not react statically, because it could lead to locking the hand to a certain grip that can not be opened or closed anymore because the controller assumes that the hand is open already. Furthermore it could not react to a touch sensor input that for example indicates that an object is touched and the hand will apply force onto it when closing further. The Newton polynomial can be extended by more values as new information is available. Further it would be necessary to be able to delete old information as soon as it is clear that it is no longer accurate. In other words, if the state of the muscles changes due to the amputee trying to grasp an object, the position at which the object is touched should be detected and the old information be deleted at this position. Further, a method would be needed in the case that the entire polynomial seems to be inaccurate and the system must try to reach the desired position based on the information with another strategy.

\paragraph{Concept of PID}
To actuate the hand in a non-static manner, a basic control strategy can be used that applies in engineering to several non-linear problems. The strategy is very dynamic and can thereby react on disturbances and can even change the controlling behavior while controlling the system. The PID strategy, so called because it includes the three terms called proportional (P), integral (I), derivative (D):

\begin{figure}[htp]
\centering
\includegraphics[width=0.8\textwidth]{Prosthesis/Control/PID}
\caption[Diagram of general functionality of PID]{A block diagram of the general functionality of a PID controller. In this case the derivative element is being driven only from plant feedback. The plant feedback is subtracted from the command signal to generate an error. This error signal drives the proportional and integral elements. The resulting signals are added together and used to drive the plant. Diagram by \cite{PID}.}
\label{PID}
\end{figure}

A common control strategy widely used in industrial control systems is a closed-loop feedback mechanism. It is highly dependent on a sensory input measuring the current state of the system and can calculate the error between the current and desired state by comparing the current state to a reference state given to the PID controller. The controller then adjusts the control inputs of the system and attempts to minimize the error. The main concept of the PID controller is that it consists of three terms that influence the control in different ways. The first term is the proportional term that only depends on the current error. The second term is the integral term that depends on the accumulation of past errors. The third term is the derivative error that tries to predict the future error by using the derivation of the error curves, namely the error change rate. The user tunable part of this concept is that the three terms are combined by a weighted sum, whose coefficients are defined by the user. The performance of the controller is highly dependent on those coefficients and the performance can then be classified by its responsiveness to an error, its degree of overshoot to the reference point and the degree of oscillatory behavior.

\begin{comment}By setting one of the three tuning factors to zero, it is possible to make a PI or PD controller (which have shown to be particularly useful). The controller highly depends on a proper input sampling rate which should at least not vary more than 20\% over a sample of 10 and based on Wescott in ''PID without a PhD''\cite{PID} each sample should only vary $\pm$1-5\% of the correct sample time. A rule of thumb for digital control systems is that the sample time should be between $\frac{1}{10}$th and $\frac{1}{100}$th of the desired system settling time. System settling time is the amount of time from the moment the drive comes out of saturation until the control system has effectively settled out.\\
\end{comment}
The impressive fact about a PID controller is that it is not necessary to have a good understanding of formal control theory to do a fairly good job of it. Unless the project needs very critical performance parameters, it is often enough to use control gains that are within a factor of two which is near the optimal value \cite{PID}. This means that PID controllers do not need to be perfectly tuned to do their task and further it is possible to optimize the processor performance on slow processors by performing the multiplication by using bit shift. Wescott states that about 90\% of the closed-loop controller applications in the world are only tuned fairly well and still work properly. In absence of knowledge of the underlying process, a PID controller has historically been considered to be the best controller \cite{Control-Engineering}.

First of all, it is important to characterize the system to be controlled in terms of desired speed of actuation, required stability to noise in the feedback signal, needed precision of reached position or force and so on. After this characterization, it is easier to choose the P,I and D gains so that only some fine tuning is necessary to achieve good results. In general the proportional component for instance passes noise through unmolested, the differential component suffers very strongly from noise, while the integral components flattens out noise as it is only based on long-term tendency.

To characterize pneumatic actuators in general, then it must be mentioned that they are non-linear and their force output and position are not easy to control. They offer some compliance which can be used to cancel out some imprecisions of control. The prosthetic hand will offer further compliance, therefore we can accept less precision to raise the speed of the system. All in all, a PD controller could serve as an optimal controller for the prosthetic device, as the integrative component is not needed here.

%----------------------------------------------------------------------------------------

\subsection{Myoelectric sensors}
As for most of the traditionally controlled prostheses, the Aeromanus hand is intended to be controlled by myoelectric sensors. A myoelectric prosthesis uses electromyographic signals or potentials detected on the surface of the skin. These signals are generated by the voluntarily contracted residual muscles within an amputees residual limb and can be used to control the prosthesis. Opposed to electric-switch prostheses that use cables or straps to operate switches on the prosthesis itself, they give the user a much more intuitive way of controlling their hand with their former flexor and extensor muscles of the hand. Nearly all myoelectric prostheses use two myographic sensors to sense two antagonistic muscles. Being a non-invasive method of control, this is what is currently most suitable to make the fitting of the prosthesis easy and comfortable for the amputee . Other methods such as the targeted muscle reinnervation(TMR) include a special surgery in which the motor nerves of the hand are rerouted into muscles on the chest. After a successful reinnervation taking several months of letting the nerves grow into the muscle tissue, the patient is able to contract the muscles in the chest by thinking of actuating his hand. Sensors placed on those chest muscles can then control parts of the artificial hand in a better way than the non-invasive method is able to \cite{Kuiken2007,targeted-muscle-reinnervation}.

\subsection{Pressure sensors}
The controller further can be used with several sensors to measure the displacement of the McKibben artificial muscles. 

\begin{figure}[htp]
\centering
\includegraphics[width=0.6\textwidth]{Muscles/Making/pressure-sensor}
\caption[Pressure sensor]{Pressure sensor used in the prototype to measure the pressure inside of the pneumatic artificial muscle.}
\label{pressure-sensor}
\end{figure}

\begin{figure}[htp]
\centering
\includegraphics[width=0.6\textwidth]{Muscles/Artificial/muscle-with-ps2}
\caption[Pneumatic Artificial Muscle with integrated pressure sensor]{The pressure sensor integrated into the pneumatic artificial muscle.}
\label{pressure-sensor}
\end{figure}

The pneumatic muscle's displacement is roughly proportional to the applied pressure and therefore it is possible to estimate the position from a certain pressure. In the current setting, the muscles are PD controlled by using the pressure sensor input as a feedback signal. In a future system that has different loads applied to the hand, the proportionality can not be assumed anymore because the muscle displacement also depends on the load applied to the tendon of the muscle. At this point the pressure sensor also gives the controller an indication of the compliance of the muscle depending on its stiffness. This means that if two muscles are attached to a finger and actuated in an antagonistic way, the pressure sensors give the controller an indication of stiffness and compliance of the finger as well as an indication of position. The indication of position can be replaced by a muscle displacement sensor that directly attaches to the muscle to measure its length or a joint sensor to measure the angle of it.

In contrast to the pressure sensor feedback controlled muscle, it is possible to get rid of the undesired deadband of the muscle which is caused by the inner bladder not touching the outer braid of the muscle which results in a deadband as long as the inner bladder has to inflate to touch the outer braid and to initiate the contraction.

\begin{figure}[htp]
\centering
\includegraphics[width=0.6\textwidth]{Muscles/Making/other-muscle-preload2}
\caption[Preloaded pneumatic artificial muscle]{The pneumatic artificial muscle contains a gap also defined as the ''preload'' between the inner bladder (covered in black stocking) and the outer braid (in blue) to increase the contraction ratio. From no contraction to full contraction there exists a deadband while the inner bladder is inflating, but the muscle is not contracting yet.}
\label{pressure-sensor}
\end{figure}

\subsection{Displacement sensor}
In the human muscle, it is not possible to contract to a special length by giving a special input to it. Muscles contract if a myosin filament attaches to the actin fiber and thereby shortens the muscle fiber. The motor neurons initiate this shortening and and the sensor neurons give feedback to the grey matter. With this closed-loop mechanism, the muscle contracts to its desired length. In an artificial hand with pneumatic artificial muscles, this is currently one of the only mechanisms that work quite precisely and stably. For position control, it is necessary to measure the position/angle of the joint that is to control. This can be done with a hall-sensor integrated into the joint of the hand as it is mostly done for robot hands. Human hands do this in a different way. The muscles contain muscle spindles, a type of sensory receptions within the belly of the muscle that detect changes in the length of the muscle \cite{Hulliger1984}. This information is then sent to the central nervous system via sensory neurons to determine the position of muscle parts. With an artificial muscle spindle made from a material changing its resistance depending on how much it is stretched, it is possible to mimic this mechanism. 

In the presented system, mainly pressure sensors were used to estimate the displacement. But it has also been shown that the control is possible with the aid of a potentiometer attached to the cable between the muscle and the attached weight.

\section{Pressure Generation and Storage}

Now that energy is provided and the control components are running, pressure is needed to be able to contract the muscles. Air pressure is commonly generated by air compressors and stored in air pressure tanks. Air has a high compression factor (up to 860 bar until it turns into a liquid \footnote{The density of liquid nitrogen is $0.8 \frac{g}{cm^3}$; the density of the atmosphere is $1.2 \frac{kg}{m^3}$.  The ratio is about 667, therefore it is possible to compress it to an equal amount of atmospheric pressure ($\approx 667 bar$) before it turns liquid. Liquid oxygen 's density is $1.1 \frac{g}{cm^3}$, which results in around 900 atmospheres before it turns liquid. At room temperature, the expansion ration is closer to 860.}), therefore most air tanks are made for high pressures to exploit this fact. Outlining multiple pressure supply strategies and choosing an appropriate strategy for a pneumatic prosthesis is a good choice.

\paragraph{High-pressure tank and compressor}

One setup would be possible with a strong high-pressure compressor that should provide air pressure of at least 100-200 bar to fill a high-pressure tank with it. The tank, having a volume of around 500ml to 3L, could thereby provide compressed air regulated down by a pressure regulator to 4 bar for an extended time. 3L of pressurized air at 200 bar would equal to 150L of pressurized air at 4 bar. The problem is that there exists no portable compressor that is able to generate pressures higher than 10 bar, therefore this option is quite unpleasant. Furthermore it would make more sense to run the system without a compressor if possible.

\paragraph{High-pressure tank without compressor}
\label{pressure-tank}

This setup contains no compressor since it would not be portable in any way. Therefore it is necessary if the tank would at least be portable and would last long enough to be useful for a patient. Using a pneumatic muscle of 12mL volume and a hand that uses at least 2 muscles per finger this would result in 120mL for a full hand grasp. From the above mentioned tank, this would enable the user to perform 1250 full hand grasps. Keeping the hand in a certain position does not require air since the air is trapped inside of the muscles.

 From the ideal gas law described by $\acute{E}mile$ Clapeyron in 1834 as a combination of Boyle's law and Charles's law, we can derive that doubling the pressure results in half the volume:
\[P \cdot V = nRT = mR_{spec}T\]

where P is the pressure in Pascal, V is the volume in $m^3$, n is the amount of substance of a specific gas, R is the gas constant ($R_{spec}$ is the special gas constant for a defined gas), m is the mass in kg and T is the temperature in Kelvin.\\

In another form: $V = \frac{mR_{spec}T}{P}$

The law shows that doubling the pressure P results in half the volume. So it would be an option to store pressurized air in a tank at a higher pressure to provide a lower pressure for a longer time since pressure is $P = \frac{F}{A}$.

Depending on the shape of the vessel there are the following formulas to calculate the mass of the needed vessel.

\paragraph{Spherical vessel}

$m = \frac{3}{2}PV\frac{\rho}{\sigma}$

where m is mass, P is the pressure difference from ambient (the gauge pressure), V is volume, $\rho$ is the density of the pressure vessel material, $\sigma$ is the maximum working stress that the material can tolerate.\footnote{For a sphere the thickness $d = \frac{rP}{2\sigma}$, where r is the radius of the tank. The volume of the spherical surface then is $4\pi r^2d = \frac{4\pi r^3P}{2\sigma}$. The mass is determined by multiplying by the density of the material that makes up the walls of the spherical vessel. Further the volume of the gas is $\frac{4\pi r^3}{3}$. Combining these equations give the above results. The equations for the other geometries are derived in a similar manner.}

Other shapes than the sphere have constants larger than $\frac{3}{2}$ and therefore require a higher weight to withstand the same pressure.
\begin{comment}
\paragraph{Cylindrical vessel with hemispherical ends}

The shape called capsule or a cylinder with hemispherical ends the formula is the following:

$m = 2 \pi R^2 (R + W) P \frac{\rho}{\sigma}$

where R is the radius W is the middle cylinder width only, and the overall width is W + 2R.

\paragraph{Cylindrical vessel with semi-elliptical ends}

In a vessel with an aspect ratio of middle cylinder width to radius of 2:1, we have a formula:

$m = 6 \pi R^3 P \frac{\rho}{\sigma}$. 
\end{comment}
\paragraph{Gas storage efficiency}

Using the ideal gas law, we can derive that there is no theoretical efficiency of scale in terms of $\frac{pressure vessel mass}{stored gas mass}$. For the same temperature, the efficiency is independent of pressure.

So to to use a tank for storing air, using a sphere for a minimum shape constant, carbon fiber for the best possible $\frac{\rho}{\sigma}$ and very cold air for the best possible $\frac{M}{pV}$ would be the best option.

The optimal tank derived from the formulas above weighting 1kg fully pressurized with air would be able to hold a pressure of:

\begin{equation}
\label{eq:vesselstrength}
\begin{aligned}
M &= \frac{3}{2}PV\frac{\rho}{\sigma} + \frac{PV}{R_{spec}T}\\
&= PV(\frac{3}{2}\frac{\rho}{\sigma} + \frac{1}{R_{spec}T})\\
P &= \frac{M}{V(\frac{3}{2}\frac{\rho}{\sigma} + \frac{1}{R_{spec}T})}\\
&= \frac{M}{\frac{4}{3}\pi*r^3(\frac{3}{2}\frac{\rho}{\sigma} + \frac{1}{R_{spec}T})}\\
\end{aligned}
\end{equation}

Derived from the mean molar mass for dry air $M_{air} = 28.9645 \frac{g}{mol}$, $R_{spec} = 287.058 \frac{J}{kgK}$, using carbon fiber as the material with a $\rho =1700\frac{kg}{m^3}$ \cite{cfrhosigma} and a $\sigma = 4.8 GPa$ \cite{cfrhosigma}, a Temperature of $293^\circ$ Kelvin and using a radius of $r = 0.2$m results in:

\begin{equation}
\label{eq:final-vessel-strength}
\begin{aligned}
P &= \frac{10N}{\frac{4}{3}\pi*(0.2m)^3(\frac{3}{2}\frac{1700\frac{kg}{m^3}}{4.8 \cdot 10^9Pa} + \frac{1}{287.058 \frac{J}{kgK}\cdot 293^\circ K})}\\
&= 2.40 *10^7 Pa = 240 bar\\
\end{aligned}
\end{equation}

$0.2^3 m^3$ is a volume of 8L which means that the tank could provide 480 L of air pressurized at 4 bar. With the muscles of 12mL volume and 10 muscles for the full hand the prosthetic device could perform 4000 full hand grasps. If a typical safety measure of the factor 2 is included, we get a maximum pressure of 120 bar which results in 2000 full hand grasps.

The problem with a tank that can withstand a pressure this high is that on one hand the tank is a security problem if it is bursting, moreover the patient needs a possibility to refill it if it is empty. If the hand can run for at least 8 hours a day, the problem's impact gets smaller, but still it could affect the ease of use of the device.

\paragraph{Low-pressure tank and compressor}

When looking at the actual pressure the device needs another option could become handy. The device does not need a high pressure to provide a high force output, a pressure of between two and six bar is highly sufficient in most cases. Having a low-pressure portable compressor could therefore be an option for a pneumatic hand. The tank in this case would be more of a reservoir to reduce the time the compressor has to run. Furthermore the system would not be dependent on a high-pressure compressor.

All that is necessary is a tank that provides about 4 bar of pressure for at least a 10 minutes before the compressor has to run again and refill the tank. The compressor should not affect the patient much in mobility and should also fit into the requirements for a prosthesis, therefore should not be of heavy weight, large size or be generating much noise that disturbs the user.

%----------------------------------------------------------------------------------------

\subsection{Determination of the compressor \& tank system}

Since in our world it is much easier to find an electric energy source than to find a air pressure source, our approach chooses the strategy described in the last paragraph and therefore contains a low-pressure tank and a small low-pressure compressor. Compressors normally use vast amounts of energy, are very noisy and mostly come only in bigger sizes and weights. Even if their pressure outputs are high, they would not be portable enough to be integrated into a pneumatic hand actuation system. Further the security of the patient with a filled high-pressure air tank can not be guaranteed. Therefore our approach chose the described setup.

Based on the chosen strategy as it is described above, the tank only needs to provide a pressure of about 4 bar directly. Tanks especially made for such pressures are quite rare because tanks such as those for paint ball or scuba diving systems are made to withstand 200-300 bar of air pressure. Therefore, a low-pressure general purpose tank for gardening was chosen providing 5 L of air pressurized to about 4 bar. The tank is added to the system mainly as a reservoir of the pressurized air generated by the compressor.

\begin{figure}[htp]
\centering
\subcaptionbox{The general purpose tank of 5L that is filled with pressurized air to source the prosthesis from a reservoir which allows to turn off the air compressor from time to time.\label{fig:tank}}{\includegraphics[width=0.25\textwidth]{Prosthesis/tank}}\hspace{0.1\textwidth}
\subcaptionbox{A lightweight, low power compressor (180g), supplying a maximal pressure of 4 bar, unfortunately at a relatively low flow (0.5L/min at 2 bar) \cite{SQUSE}. \label{fig:compressor}}{\includegraphics[width=0.35\textwidth]{Prosthesis/compressor}}
\label{tank-compressor}
\end{figure}

%----------------------------------------------------------------------------------------

\section{Pressure Distribution \& Flow Control}

The air is now compressed and stored in a tank. The next thing to do is to distribute the air and control the flow to fill and empty the muscles at the right moment in time. To do this in a fully automatic way, electrically controlled valves are used to control the flow of air or the pressure.

%----------------------------------------------------------------------------------------

\subsection{Choice of the valve}
\label{valve}
The control of the air should be possible to be done by the micro controller so that the wearer of the prosthesis does not need to take care of it. Solenoid valves can achieve this task perfectly well. The controller therefore computes the next change of air pressure for the desired movement and then sends the amplified control signals to the appropriate valves. As for all the other components, the constraints for a prosthesis ask for a reduction of the size and the weight. Several types of valves exist which are either controlling the flow of air or the pressure at the valve. A pressure controlling valve is basically a flow controlling valve with an additional pressure sensing component that measures the pressure, compares it to the desired pressure and adjusts the opening of the valve accordingly, assuming that by opening the valve accordingly it either can increase or reduce the pressure at the measuring point. A flow-control valve is not able to control the pressure by itself, but can open and close to let air flow in or out and thus the pressure is increased or reduced, but the valve is not able to react to the pressure. The most important properties of a flow-control valve are basically its maximum pressure (in bar), its maximum flow (in L/min) and the type of control, being either on-off flow-control or proportional flow-control. On-Off valves can only either be fully open, thus the flow is at its full capacity or closed to reduce it to its lowest, which is ideally no flow. Proportional valves on the other hand are able change the flow throughput proportionally, which means that the opening and closing is not properly linear, still that it can be opened and closed in a continuous manner. To control it linearly, the PID control described above is needed. A further diversity of valves exists in the number of combined inlets and outlets, meaning that air can be either just be flowing out at one outlet or not at all or it can be switched resp. proportionally distributed among two or more outlets. All in all, an in-depth comparison of different types of pneumatic valves will definitely help to find the appropriate valve on one hand and could show a scaling law for pneumatic valves itself.

\begin{table}[H]
\caption{Comparison of valves}
\begin{adjustwidth}{-1.1in}{-1.1in}
\scriptsize
\begin{tabular}{@{}p{2.5cm}lllllllll@{}}
\toprule
Name & Type & Dim.($mm^3$) & m(g) & Price(\$) & Flow($\frac{L}{min}$) & P(bar) & Energy(W) & Speed & Noise\\
\midrule


SMC S070C-6DG-32 \cite{s070} & On/0ff & 32.5 x 11.5 x 7 & 5g & 30\$ & 9.6 $\frac{L}{min}$ & 0-5 bar & 0.5W & 5ms & 38dB \\

SMC S070C-6CG-32 \cite{s070} & On/0ff & 32.5 x 11.5 x 7 & 5g & 30\$ & 15.1 $\frac{L}{min}$ & 0-3 bar & 0.5W & 5ms & 38dB \\

Norgren Flatprop EQP \cite{NorgrenFP}& Proport. & - & 30g & 200\$ & 2-250 $\frac{L}{min}$ & 0-7 bar & 2.5W & 10ms & - \\

Norgren FAS Flatprop \newline 12-216P-00220++BED \cite{NorgrenFP}& Proport. & 38.1 x 14.8 x 14.8 & 30g & 120\$-200\$ & 0-2 $\frac{L}{min}$ & 0-12 bar & 0.5W & 10ms & - \\

Norgren FAS Flatprop \newline 12-216P-01-20++BED \cite{NorgrenFP}& Proport. & 38.1 x 14.8 x 14.8 & 30g & 120\$-200\$ & 0-30$\frac{L}{min}$ & 0-10 bar & 2.5W & 10ms & - \\

Norgren FAS Flatprop \newline 12-216P-03-20++BED \cite{NorgrenFP}& Proport. & 38.1 x 14.8 x 14.8 & 30g & 120\$-200\$ & 0-70$\frac{L}{min}$ & 0-5 bar & 2.5W & 10ms & - \\

Norgren FAS Flatprop \newline 2-216P-04520+EQIFIL+BED \cite{NorgrenFP} & Proport. & 38.1 x 14.8 x 14.8 & 30g & 120\$-200\$ & 0-250$\frac{L}{min}$ & 0-7 bar & 2.5W & 10ms & - \\

Lee HDI \cite{LeeHDI}& On/0ff & 31.75x7.37x5.84 & 4g & 50\$ & 7.8$\frac{L}{min}$ & 0-2 bar & 0.75W & 2ms & - \\

Lee Prop Prototype & Proport. & 31.75x7.37x5.84 & 4g & 50\$ & 7.8$\frac{L}{min}$ & 2 Bar & 0.2 & 2ms & - \\

FESTO MPYE-5-1/8-HF-010 B Flow Valve \cite{FESTO-MPYE}& Proport. & 129.9 x 26 x 20.1 & 290g -740g & 1000\$ & 100-1000$\frac{L}{min}$ & 10 Bar & 2.4W & 10ms & 60dB \\

\hline
\end{tabular}
\end{adjustwidth}
\label{valves}
\end{table}

\begin{figure}[H]
\centering
\subcaptionbox{\label{fig:smc070}}{\includegraphics[width=0.2\textwidth]{Valves/smcs070}}\hspace{0.1\textwidth}
\subcaptionbox{\label{fig:MPYE}}{\includegraphics[width=0.2\textwidth]{Valves/MPYE}}
\hspace{0.1\textwidth}
\subcaptionbox{\label{fig:norgren}}{\includegraphics[width=0.2\textwidth]{Valves/norgren}}\\
\caption[Valve types]{The different types of valves in the comparison:
A) SMC s070 B) Festo MPYE C) Norgren Flatprop}
\end{figure}

%----------------------------------------------------------------------------------------
\subsection{Valve Labor Testing}

Since it would be very interesting to use proportional flow control valves instead of on-off flow control valves, a lot of proportional valves were examined for their suitability for a prosthetic hand. Unfortunately, nearly all of them are not ready for this use except for the Norgren Flatprop EQP and a valve from the Lee company, the later being not yet a product but only announced in the media as being a new miniature proportional flow-control valve. Since the Norgren Flatprop EQP did not meet the requirements because they are very expensive, the only option was to ask the Lee company if their new valve was available for testing. Fortunately, Juergen Prochno and Guenther Pfeiffer from the Lee Company Germany were interested in the integration of a testing prototype valve into the prosthesis. Two valves were provided for testing, which could be used to control one muscle, one valve for the in-flow, the other for the out-flow of the muscle. 

\begin{figure}[H]
\centering
\includegraphics[width=0.5\textwidth]{Valves/Prop-valve-lee}
\caption[Lee company prototype]{The proportional valve prototype of the Lee company.}
\label{prop-valve-lee}
\end{figure}

Overall, the valves featured a strong reduction of the operation noise of the device and their control was in a precise manner. Their only drawback was their quite low flow, which slowed the operation of the muscles down so that the prosthesis' open and closing time of the hand would increase to an undesirable time of 5 seconds or more.

Unfortunately, the valve specification was not precise enough in terms of the voltage to control the valve and the maximum current, which burned the valves after a certain time before they could have been used for more intensive experiments. Later it was mentioned that the valve prototype's coil responsible for the flow control is very sensitive to high voltage spikes which probably caused them to burn so early. Since it is a prototype of an unfinished product, this is not very problematic and it was a great experience to be an early product tester. The author hereby thanks the Lee Company again for the trust and the interest in this integration test and hopes to have contributed to making the product better. Both the maximum flow and the sensitivity are features that are work-in-progress at the product-development of the Lee company and the valve will definitely perform better in the end.  


As none of the proportional valves was suitable for the pneumatic prosthesis, the best on-off valve was used, which was found to be the SMC 070 type, which featured a high flow and pressure compared to its weight, size and price, whereas others failed on the price constraint.
%----------------------------------------------------------------------------------------

\section{Pressure Transformation}

Now that the pressurized air is distributed to the actuators, the device comes to its main task. The actuation of the fingers is not only the task that shows if the whole system is working well or not, but it is also the most complicated, whereas being the most underestimated by humans since it is one of our daily activities that we do without thinking about it. As we think of moving the hand in a certain trajectory while changing its shape from one to another, the hand just follows our high-level commands such as ''move towards the cup of coffee'' or ''grab the cup and lift it''. We do not think about how much force it needs to do a certain task, our complex motor cortex just does it for us. To perform this task in a proper manner, we came this far to work out the necessary specifications for the pneumatic artificial muscle actuator to move the fingers in the most delicate way, even though we can expect that it is not yet similar to the human movement, but the same as the prosthesis, the human hand was not built in a day as well.

%----------------------------------------------------------------------------------------



For the following calculations, we refer to the manual on how to make pneumatic artificial muscles which is in Appendix A.


%----------------------------------------------------------------------------------------

\subsection{Muscle diameter decision}

First off, it has to be defined how much force is needed per muscle. Every application has its own requirement on the force of the used muscle. In the case of a pneumatic prosthesis, the force requirement from chapter \ref{Chapter3} can be taken and can be divided by the number of muscles that are involved in the force output into one direction of the hand. Beginning with a simple prototype, this would mean that each finger is driven by one pair of muscles and that half of the muscles is involved in closing the hand and the other half is involved in opening it. One muscle per flexion per finger, this results in 5 muscles that need to fulfill the force output requirement. Based on the requirement to do most ADLs from chapter \ref{Chapter3}, the prosthesis needs a force of 585 N, the next higher output force would be 760 N and to try to compete with the Bebionic v.2 hand it should have an output force of 900 N. To find the force of a pneumatic artificial muscle, several formulas have been proposed to calculate it. Based on the materials that were bought, we get the following force output results by the formulas in Appendix A.

\begin{table}[H]
\scriptsize
\begin{tabular}{p{2.5cm}p{10cm}|l}
\toprule
	Gaylord 1958 \cite{Gaylord1958}& $F = \frac{400'000 Pa \pi 0.025m^2}{2}(3 cos^2(\pi)-1)$ & 392 N\\

	Hannaford \newline and Chou \cite{Hannaford1995}& $ F = \frac{\pi (0.025m)^2 400'000 Pa}{4}(3cos^2(\pi/2)-1) + \pi 400'000 Pa (0.025m 0.002m(2sin(\pi/2)-\frac{1}{sin(\pi/2)} - (0.002m)^2$ & 638 N\\
	Chou and \newline Hannaford \cite{Hannaford1995} & $F= \frac{4000000 Pa (0.195m)^2}{4\pi 2^2(\frac{3(0.112m)^2}{(0.195m)^2}-1}$ & 312 N\\
	Colbrunn 2001& $F = \frac{400'000 Pa (0.195m)^2}{4\pi 2^2}(\frac{3(0.112m)^2}{(0.195m)^2}-1)$ & 3 N\\
	DeVolder 2011 &
	$F = max[0, \frac{(400'000 Pa-50'000 Pa) (0.195m)^2}{4\Pi 2^2}(\frac{3(0.112m-0)^2}{(0.195m)^2}-1)] + max[0,k_b (0.112m- 0)]$ & 0 N\\
	Surentu(1) & $ F =  \frac{3}{4 2^2\pi}400'000 Pa ((0.112m)^2 - 1/3(0.195m)^2)$ & 3N \\
	Surentu(2) & $ F = 400'000 Pa \frac{3\pi 0.01m^2}{8 (0.112m)^2}((0.08m)^2-(0.112m)^2)$ & 23 N\\
	\bottomrule
\end{tabular}
\normalsize
\end{table}

Whereas the formula by Gaylord and the one by Chou and Hannaford gave reasonable results, the other results were not reasonable. Therefore a synthetic method was chosen. This was done by choosing a certain maximum braid diameter and the according dimensions according to the ratios as described below. For the first muscle that was built, a braid diameter of 0.8 cm was chosen. For the second muscle that was built later on, a braid diameter of 1.2 cm was chosen. 

%----------------------------------------------------------------------------------------

\subsection{Ratio between inner and outer tube}

Following the synthetic method, we can simply look up the outer diameter of the tube in the look up table below. The look up table is based on the experience of many fabricated muscles during the development of the actuation system and also on the successful fabrication of pneumatic muscles in several papers such as \cite{Klute1999}. From these references, a ratio of between 0.666:1 and 0.75:1 has been found. The table just shows several common braid and tube diameters and the diameter of the tube / braid respectively.

\begin{table}[H]
\begin{tabular}{lllp{1cm}lll}
\toprule
\multicolumn{7}{l}{\textbf{Example lookup table for dimensions}} \\
\midrule
Rubber & Braid & \multicolumn{2}{l}{Braid}& Braid & Rubber & Rubber\\
{} & Ratio:\textbf{0.666:1} & \multicolumn{2}{l}{Ratio:\textbf{0.75:1}} & {} & Ratio:\textbf{0.666:1} & Ratio:\textbf{0.75:1}\\
\midrule
3 & 5 & 4 & {} & 2 & 1.3332 & 1.5 \\
6 & 9 & 8 & {} & 3 & 1.9998 & 2.25 \\
8 & 12 & 11 & {} & 4 & 2.6664 & 3 \\
9 & 14 & 12 & {}& 6 & 3.9996 & 4.5 \\
10 & 15 & 13 & {} & 8 & 5.3328 & 6 \\
12 & 18 & 16 & {} & 10 & 6.666 & 7.5 \\
14 & 21 & 19 & {} & 12 & 7.9992 & 9 \\
17 & 26 & 23 & {} & 14 & 9.3324 & 10.5 \\
\end{tabular}
\caption[Diameter ratios]{A table to look up successful combinations of diameters for braid and rubber tube. The diameters can be adjusted to whatever lies between the value at the ratio 0.666:1 and the ratio 0.75:1.}
\label{look-up-table}
\end{table}

Using the lookup table above, the first muscle contains a rubber tube of 6mm diameter and the second muscle contains a rubber tube of diameter 9 mm.

%----------------------------------------------------------------------------------------


\subsection{Formula for the muscle travel distance}

The next dimension to define is the muscle travel distance. The muscle travel distance is the distance that the muscle has to travel from the initial fully relaxed state to the fully contracted state. The muscle travel distance is dependent of the application. In our case the muscle actuates a model of a human finger in which three joints including the metacarpophalangeal joint can be flexed and extended the same as in the human counterpart. The travel distance therefore is dependent on the flexion angle that each joint rotates around when flexing/extending. For every joint, a circle of rotation can be defined to that the tendon cable of the finger is a tangent. As the finger joint flexes, the circle rotates by the flexion angle $\phi$ resulting a travel distance of $2r\Pi*\frac{\phi}{2\Pi} = r\phi$. From the finger model that will be actuated, the flexion angles and the radius of the joint circles are as follows: The metacarpophanageal (MCP) joint has a flexion angle of $45^\circ$ (0.7854 rad) and a radius of 4.3mm. The Proximal interphalangeal (PIP) joint has a flexion angle of $47^\circ$ (0.8203 rad) and a radius of 3.3mm. The last joint, the distal interphalangeal (DIP) joint has a flexion angle of $40^\circ$ (0.6981 rad) and a radius of 3mm. Further every joint can be overbended by about $\chi = 8^\circ$ (0.1396 rad).


\[
\begin{aligned}
 travel_{proximal} &= 2 \cdot ((\phi_{MCP} + \chi) \cdot r_{CMP} +(\phi_{PIP} + \chi) \cdot r_{PIP} \\
 &= 2 \cdot ((0.7854 + 0.1396) \cdot 4.3mm +(0.8203 + 0.1396) \cdot 3.3mm\\
 &= 14.29 mm
 \end{aligned}
\]

\[
\begin{aligned}
 travel_{distal} &= 2 \cdot ((\phi_{CMP} + \chi) \cdot r_{CMP} +(\phi_{PIP} + \chi) \cdot r_{PIP} + (\phi_{DIP} + \chi) \cdot r_{DIP} \\
 &= 2 \cdot ((0.7854 + 0.1396) \cdot 4.3mm +(0.8203 + 0.1396) \cdot 3.3mm + (0.6981 + 0.1396) \cdot 3mm\\
 &= 19.32 mm
 \end{aligned}
\]

A muscle attaching to the proximal phalanx and therefore actuating the PIP joint and the MCP joint will therefore need a travel distance of 1.429cm.

The travel distance for a muscle attaching to the distal phalanx and therefore additionally actuating the DIP joint of the finger will be 1.932cm.

%----------------------------------------------------------------------------------------

\subsection{Muscle length determination}

Having determined the distance of muscle travel, we can calculate the length of the muscle from it. Several papers mention a maximum contraction of 18\%-25\% when referring to pneumatic muscles \cite{Klute1999,Hannaford1995}. This has been validated by a pre-experiment when beginning with the synthetic method. As it is better to have too much travel instead of not enough, we use $ratio_{cont} = 18\% = 0.18$.

To get the total muscle length, we divide the travel distance by the contraction percentage.
\[
\begin{aligned}
length_{proximal muscle} &= \frac{travel_{proximal}}{ratio_{cont}}
&= 79.393 mm
\end{aligned}
\]
The length can be obtained in the same manner for the distal muscle.
\[
\begin{aligned}
length_{distal muscle} &= \frac{travel_{proximal}}{ratio_{cont}}
&= 107.318 mm
\end{aligned}
\]

It has to be stated that the determined lengths are the length of the muscle in the contractory area of the muscle, meaning that the muscle will be longer as elements such as the attachment slopes will increase the muscle in length.

\subsection{Fabrication}

Now that the important dimensions for a pneumatic artificial muscle are defined and determined, the muscles can be built as described in the Appendix A. The Appendix helps to build the muscle step by step and further gives good advice on how to avoid future issues with the muscle. To simplify the experiments on force, we just build the distal muscles so that the behavior of one muscle can be studied separately. After the muscles had been built, they were attached to the valves by push-fittings to facilitate the reconfiguration of the device and simplify the maintenance of it. The muscles are moreover attached to the tendons of the the appropriate actuation point.

%----------------------------------------------------------------------------------------

\section{Force Distribution}

The force exerted by the muscles in the forearm and hand is transmitted to the fingers by tendons, which are led through the wrist to their insertion points at the different phalanges of the finger. The tendon mechanism is a key feature of the human hand that guarantees a small and lightweight hand that still is strong enough to perform activities of daily living, being prehensile or forceful in character. The same tendon mechanism is implemented in the prototype of the Aeromanus hand to separate the actuators and valves from the very lightweight fingers. For this purpose, the arm attachment of the prototype features small wholes at the front side with Teflon tubes to reduce the friction and little cable holders along the finger model to redirect the force to the tip of the finger. By this separation of the source of the force and its actuation point, the finger keeps lightweight and thin, but has an enormous force as long as all the redirections stay in place.
%----------------------------------------------------------------------------------------

\section{Dexterity}

When looking at a human hand, one definitely expects high dexterity and believes that this can only be reached by a higher amount of actuators and a precise control of the fingers. But even in the hand there are some underactuated joints such as the distal interphalangeal joints In a prosthesis, given the technical limits of today's technology, the designer has reduce the dexterity significantly. Mason established that a minimum of 3 DoF are necessary in a hand with non-compliant, non-rolling and non-sliding contacts to achieve basic prehension and 9 DoF to perform dexterous manipulation \cite{Mason1985}. But since it is also important to consider energy consumption as well as weight of the prosthesis,a trade-off between complexity and dexterity of the design is necessary \cite{Controzzi2010}. The current prototype of the Aeromanus hand contains its actuators in the forearm because pneumatic actuators need a certain length and a certain diameter which makes the the forearm a more suitable place to fit them. In a later version, one could think of using muscles inside of the hand to do more prehensile movements. Combining muscles inside and outside of the hand as seen in the human model has its advantages, by involving the intrinsic muscle mostly for dexterous, fast, precise movements, and the extrinsic muscles for strong grasps.

\section{Representation of different muscles in the prosthesis}

As seen in the anatomical description of chapter \ref{Chapter1}, the hand contains intrinsic muscles in the hand itself and extrinsic muscles in the forearm. In the case of natural function, several muscles groups work together to exert a weaker force in a precise way or a stronger force in a coarse way. In the case of a prosthesis, the case is different. Prostheses contain normally just one type of actuator to achieve the same specified goal. This raises the issue of having a system that is either very precise and weak or coarse and strong. But the same system is then not able to switch between these two options. In this early prototype of the Aeromanus hand, only muscles with a large diameter and length were chosen because of the advantages mentioned above. In the case of pneumatic muscles, it is a good choice to choose strength over precision since PAMs lack of very precise control but have an extraordinary force-to-weight ratio. Furthermore it is still possible to improve the control of the system so that it can achieve the desired precision. Very desirable about large diameter pneumatic artificial muscles is also that they can hold the exerted force without consuming energy as long as there is no leakage in the system. For larger muscles this means that they can provide a strong consistent grip whose compliance can be controlled at a very low device weight.

\subsection{Pneumatic finger - A prototype showing potential}

The first prototype built to test the PD control of the pneumatic muscles was the following setup. The idea of it was to build two muscles and a model of a finger and its corresponding metacarpal bone into a isolated system to observe the effect of the muscle movement on the finger joint flexion and extension induced by the cables attached to the muscles and to the distal phalanx of the finger. The biological model is shown in Fig. \ref{antagonistic-muscles}.

\begin{figure}[H]
\centering
\includegraphics[width=0.35\textwidth]{Anatomy/Antagonistic-muscles}
\caption[Antagonistic muscles]{Antagonistic muscles in the arm. Two muscles that are antagonistic in the arm are for instance the flexor carpi radialis and the flexor carpi ulnaris.}
\label{antagonistic-muscles}
\end{figure}

The cable is further led along the finger by small Teflon tubes to lower the friction. The tubes are held by small cable holders that keep the tubes in place near the middle of each phalanx. Thereby the cables, acting similar to tendons in natural hands, pull at the fingertip with an underactuated mechanism giving the finger the ability to adapt to an object which is grasped by exploiting the mechanism.

\begin{figure}[H]
\centering
\subcaptionbox{\label{fig:First-platform}}{\includegraphics[width=0.37\textwidth]{Pneumatic-finger-platform/First-platform}}\hspace{0.1\textwidth}
\subcaptionbox{\label{fig:First-platform2}}{\includegraphics[width=0.29\textwidth]{Pneumatic-finger-platform/First-platform2}}\\
\caption[First platform]{A picture of the first platform. The mounted finger on it is driven by the tendons connected to the muscles.}
\label{antagonistic-muscles}
\end{figure}

The finger is mounted on top of an aluminium bar. Fixed to the bar is a special piece similar to the head of a hammer but movable on the aluminium bar that holds the two muscles at a custom distance to the finger. The aluminium bar features several screw holes along the axial direction to mount the piece at a custom distance. The hammer-head piece itself can be moved along the bar if not fixed by a screw, and the muscles can be mounted at three radial distances with reference to the bar.

The latter is designed to support multiple muscle diameters so that they have enough space to expand radially. By reconfiguring the muscles, the finger can be created to be symmetrically flexing and extending as well as having either actuation with some offset. The hammer-head piece holds the rear-side of the muscles at the rear side of the aluminium bar and on the other side of the muscles the tendons are attached to them. Further hammer-head pieces can be used to mount multiple muscle pairs to the finger to actuate it in a less underactuated manner. 
 
\begin{figure}[H]
\centering
\includegraphics[width=0.5\textwidth]{Pneumatic-finger-platform/PF2-1}
\caption[Hammer-head piece]{The hammer-head piece shown above with muscles mounted to it, which move the finger. In this setting, it was used to hook weights to it to investigate the force output of the muscle.}
\label{hammer-piece}
\end{figure}

\begin{figure}[H]
\centering
\subcaptionbox*{\label{PF2-1}}{\includegraphics[width=0.15\textwidth]{Pneumatic-finger-platform/PF2-4}}
\subcaptionbox*{\label{PF2-2}}{\includegraphics[width=0.15\textwidth]{Pneumatic-finger-platform/PF2-5}}
\subcaptionbox*{\label{PF2-3}}{\includegraphics[width=0.15\textwidth]{Pneumatic-finger-platform/PF2-6}}
%\subcaptionbox*{\label{PF2-4}}{\includegraphics[width=0.15\textwidth]{Pneumatic-finger-platform/PF2-7}}
\subcaptionbox*{\label{PF2-5}}{\includegraphics[width=0.15\textwidth]{Pneumatic-finger-platform/PF2-8}}
\subcaptionbox*{\label{PF2-6}}{\includegraphics[width=0.15\textwidth]{Pneumatic-finger-platform/PF2-9}}
%\subcaptionbox*{\label{PF2-7}}{\includegraphics[width=0.15\textwidth]{Pneumatic-finger-platform/PF2-10}}
\subcaptionbox*{\label{PF2-8}}{\includegraphics[width=0.15\textwidth]{Pneumatic-finger-platform/PF2-11}}
\subcaptionbox*{\label{PF2-9}}{\includegraphics[width=0.15\textwidth]{Pneumatic-finger-platform/PF2-12}}
\subcaptionbox*{\label{PF2-10}}{\includegraphics[width=0.15\textwidth]{Pneumatic-finger-platform/PF2-13}}
\subcaptionbox*{\label{PF2-11}}{\includegraphics[width=0.15\textwidth]{Pneumatic-finger-platform/PF2-14}}
\subcaptionbox*{\label{PF2-12}}{\includegraphics[width=0.15\textwidth]{Pneumatic-finger-platform/PF2-15}}
\subcaptionbox*{\label{PF2-13}}{\includegraphics[width=0.15\textwidth]{Pneumatic-finger-platform/PF2-16}}
\subcaptionbox*{\label{PF2-14}}{\includegraphics[width=0.15\textwidth]{Pneumatic-finger-platform/PF2-17}}\\
\caption[Underactuated flexion]{Underactuated flexion of the finger induced by pressurizing the pneumatic artificial muscles antagonistically.}
\end{figure}

\subsection{Design of the prosthesis prototype}

The architecture of the hand was designed based on the constraints previously described. The complete Aeromanus hand consists of an arm attachment to fixate it to the user's arm, a lightweight hand, that is inspired by the bones of the human hand, pneumatic valves, a controller with an amplifier board, a compressor, a tank, and a power supply.

\begin{figure}[htp]
\centering
\subcaptionbox{\label{hand-socket}}{\includegraphics[width=0.48\textwidth]{Prosthesis/hand_socket_pneumatic2}}\hspace{0.03\textwidth}
\subcaptionbox{\label{hand-socket2}}{\includegraphics[width=0.48\textwidth]{Prosthesis/hand_socket_pneumatic3}}\\
\caption[Hand prototype]{Pictures of the hand prototype model including socket (arm attachment) that fits on the lower arm.}
\end{figure}

The attachment consists of two bows that are surrounding the arm's ulna and radix, one a bit below the elbow and the other 12-15cm below the first one just above the wrist. The bows are made of ABS printable plastic and are hollowed to be as lightweight as possible. The one around the wrist is smaller since also the wrist is of smaller diameter than the location of the other bow below the elbow.
 
\begin{figure}[htp]
\centering
\subcaptionbox{The wrist bow which attaches to the lower arm near to the wrist. It features several holes along the front edge to redirect the tendons to the hand.
\label{wrist-bow}}{\includegraphics[width=0.4\textwidth]{Prosthesis/wrist-bow}}\hspace{0.1\textwidth}
\subcaptionbox{The elbow bow which attaches to the lower arm near to the elbow. It features several holes on the front side to add hooks to it to attach muscles. The upper sides provide screw holes for the pneumatic valve holders.\label{elbow-piece}}{\includegraphics[width=0.4\textwidth]{Prosthesis/elbow-piece}}\\
\end{figure}
 
On the inner side of the bows, two straps are attached to attach the bows properly and stably to the arm. Between the two bows, three pillars are holding them apart. The pillars are made from carbon fiber tubes and plates and are adjusted in a manner that always one carbon fiber plate is placed between two parallel carbon fiber tubes. This setting improves the stability since they are all holding the two bows apart alongside themselves. The tubes protect each other from breaking along the plane that is formed by them. The plate is inserted perpendicular to the plane described to protect the tubes from breaking into the direction of the plane described by the plate. The pillars between the ends of the bows are made from two tubes and one plate, the one connecting the middle of the bows is made from three tubes an two plates.
  
\begin{figure}[htp]
\centering
\includegraphics[width=0.6\textwidth]{Prosthesis/prosthesis-with-muscles}
\caption[Two bows of attachment]{The two bows of the prosthesis are connected with the carbon pillars. Further visible are the muscles that are attached by the hooks.}
\label{pillars}
\end{figure}
  
The bigger bow placed near the elbow is fitted with hooks to hook pneumatic artificial muscles to it. While the inner surface of the bow is rounded to fit the arms shape, the outer surface of the bow is shaped like a nut of a screw to have flat surfaces to mount the valves to it. 
   
\begin{figure}[H]
\centering
\includegraphics[width=0.6\textwidth]{Prosthesis/valve-closeup2}
\caption[Smaller bow]{On the bow near the elbow, the valves are mounted by using the valve holders. The valve holders keep the tubes in place and prevent the tubes from breaking off the nozzles of the valves.}
\label{pillars}
\end{figure}
   
For each valve pair, there is a little case so that the Polyurethane tubes connecting the pneumatic components do not break off the barbs of the valves as well as the cases protect the valves from damage.

The pneumatic artificial muscles(PAMs) are located between the pillars and end near the front bow when uncontracted. Both endings of the PAM are formed to a loop, one to attach it to the hook at the rear bow and one to attach a fishing line tendon to it.


\begin{figure}[H]
\centering
\includegraphics[width=0.8\textwidth]{Prosthesis/muscle-contraction1}
\caption[Antagonistic pneumatic artificial muscles]{Pneumatic artificial muscles antagonistically controlled. Therefore one of the muscles is contracted, the other expanded.}
\label{pneumatic muscles}
\end{figure}

\begin{figure}[H]
\centering
\subcaptionbox*{\label{PM-1}}{\includegraphics[width=0.3\textwidth]{Prosthesis/muscle-contraction1}}
\subcaptionbox*{\label{PM-2}}{\includegraphics[width=0.3\textwidth]{Prosthesis/muscle-contraction2}}
\subcaptionbox*{\label{PM-3}}{\includegraphics[width=0.3\textwidth]{Prosthesis/muscle-contraction3}}
\subcaptionbox*{\label{PM-4}}{\includegraphics[width=0.3\textwidth]{Prosthesis/muscle-contraction4}}
\subcaptionbox*{\label{PM-5}}{\includegraphics[width=0.3\textwidth]{Prosthesis/muscle-contraction5}}
\subcaptionbox*{\label{PM-6}}{\includegraphics[width=0.3\textwidth]{Prosthesis/muscle-contraction6}}
%\subcaptionbox*{\label{PM-7}}{\includegraphics[width=0.3\textwidth]{Prosthesis/muscle-contraction7}}
\subcaptionbox*{\label{PM-8}}{\includegraphics[width=0.3\textwidth]{Prosthesis/muscle-contraction8}}
\subcaptionbox*{\label{PM-9}}{\includegraphics[width=0.3\textwidth]{Prosthesis/muscle-contraction9}}
\subcaptionbox*{\label{PM-10}}{\includegraphics[width=0.3\textwidth]{Prosthesis/muscle-contraction10}}
\\
\caption[Contraction of an antagonistic pneumatic muscle pair]{Contraction of an antagonistic pneumatic muscle pair which is mounted to the muscle. The position of the muscle can be changed by a potentiometer.}
\end{figure}

 The smaller bow that is placed near to the wrist has several holes along the hand-directed edge fitted with Teflon-tubes to hold and lead the tendons from the muscles as frictionless as possible. Furthermore it is possible to redirect the tendons to the right angle to actuate the joints of the hand. The tendons are then lead to the pulleys of the joints by more Teflon-tubes to hold them in place and to redirect them further.

\begin{figure}[htp]
\centering
\subcaptionbox*{On the bow near the wrist, the tendons are redirected by teflon tubes, that are inserted into the holes of the wrist bow. The muscles tendons are stretched by small weights to keep the muscles under minimal tension.\label{prosthesis-with-weight}}{\includegraphics[width=0.45\textwidth]{Prosthesis/prosthesis-with-weight3}}
\hspace{0.07\textwidth}
\subcaptionbox*{An overview of the electronics of the controller. Lower half: the Arduino Mega board. Upper left: Signal amplifier board, which amplifies the 5V signal of the Arduino board up to a 24 V signal to drive the valves. Further the tank of the device. Upper right: the compressor is visible.\label{electronics}}{\includegraphics[width=0.45\textwidth]{Prosthesis/electronics-overview}}
\\
\end{figure}

The controller is not mounted onto the system, but could be mounted onto the pillars to be near to the valves as well as the pressure sensors, which are attached to the PAM's end pointing towards the hand. The currently used amplifier-board could be made smaller and added to the controller board as well. The current setting is not portable and is fixed onto the surface of a table to keep the components and wires in place. The control is done by an Arduino mega board that is running the control code and thus gets input signals from the sensors and provides the control signals on its GPIO pins. Running the controller's code initiates the calibration of the sensors, which measures the minimum and maximum pressure on the pneumatic muscles to fully expand and contract them. The device could easily be extended to use angle sensors in the finger joints and displacement sensors on the muscles which was already tested in the experiments with a potentiometer as a displacement sensor.

The tank as well as the compressor can be worn in a backpack which could also carry the accumulator for the system. The tubes of the system are led along the arm by a sleeve to assure the non-constrained movement of the shoulder and elbow.

A requirement of the prosthesis is the weight of it. Therefore the weight of its components were summed up.

\begin{table}[H]
\scriptsize
\centering
\begin{tabular}{lll}
\toprule
Component & Weight & Quantity\\
\midrule
Prosthetic weight & &\\
\hline
ABS plastic and carbon socket & 150g (estimated) & 1\\
Valve & 5g & 4\\
PAM & 9.52g & 2 \\
Pressure sensor & 3.18 g & 2\\
Hand & 100g (estimated) & 1\\
Controller & 100g & 1 \\
\midrule
\textbf{Total} & \textbf{395g} & \\
\hline
Backpack/Belt weight & &\\
\hline
Tank & 300g & 1\\
Compressor & 180g & 1\\
\midrule
\textbf{Total} & \textbf{480g} & \\
\end{tabular}
\label{weight-table}
\normalsize
\end{table}

The weight of the prototype is therefore around 395g, with a backpack or belt of 480g. For a complete prosthesis, the weight only increases slightly as only more actuators are needed which equals to 35.4 g per actuator (2 valves, 2 PAMs and 2 pressure sensors). To upgrade the system to a similar functionality as other prostheses, two additional actuators are necessary to actuate the thumb, the index and the middle finger and the ring and little finger together. This would result in a total weight of 465.8g.

\end{document}