%%%%%%%%%%%%%%%%%%%%%%%%%%%%%%%%%%%%%%%%%
% Thesis 
% LaTeX Template
% Version 1.3 (21/12/12)
%
% This template has been downloaded from:
% http://www.latextemplates.com
%
% Original authors:
% Steven Gunn 
% http://users.ecs.soton.ac.uk/srg/softwaretools/document/templates/
% and
% Sunil Patel
% http://www.sunilpatel.co.uk/thesis-template/
%
% License:
% CC BY-NC-SA 3.0 (http://creativecommons.org/licenses/by-nc-sa/3.0/)
%
% Note:
% Make sure to edit document variables in the Thesis.cls file
%
%%%%%%%%%%%%%%%%%%%%%%%%%%%%%%%%%%%%%%%%%

%----------------------------------------------------------------------------------------
%	PACKAGES AND OTHER DOCUMENT CONFIGURATIONS
%----------------------------------------------------------------------------------------
\documentclass[11pt, a4paper, oneside]{Thesis} % Paper size, default font size and one-sided paper

\graphicspath{{./Pictures/}} % Specifies the directory where pictures are stored

%% Packages
%% ========

%% LaTeX Font encoding -- DO NOT CHANGE
\usepackage[OT1]{fontenc}

%% Babel provides support for languages.  'english' uses British
%% English hyphenation and text snippets like "Figure" and
%% "Theorem". Use the option 'ngerman' if your document is in German.
%% Use 'american' for American English.  Note that if you change this,
%% the next LaTeX run may show spurious errors.  Simply run it again.
%% If they persist, remove the .aux file and try again.
\usepackage[english]{babel}

%% Input encoding 'utf8'. In some cases you might need 'utf8x' for
%% extra symbols. Not all editors, especially on Windrows, are UTF-8
%% capable, so you may want to use 'latin1' instead.
\usepackage[utf8]{inputenc}

%% This changes default fonts for both text and math mode to use Herman Zapfs
%% excellent Palatino font.  Do not change this.
\usepackage[sc]{mathpazo}


%% We unfortunately need this for the Rules chapter.  Remove it
%% afterwards; or at least NEVER use its underlining features.
\usepackage{soul}
\usepackage{bm}
\usepackage{datetime}
%% To use alternating colors in the glossary
\usepackage[table]{xcolor}
%% We use subfiles to separate the file into subfiles which are compilable on their own by copying the preamble from this file.
\usepackage{subfiles}

%common
\usepackage{comment}
\usepackage{ifthen}
\usepackage{todonotes}
% to disable the todo notes in the final version in case we still have to do notes
\presetkeys{todonotes}{disable}{}

\usepackage{titlesec}

%lists
\usepackage{listings}
\usepackage{enumerate}


%plain text
\usepackage{verbatim} %% many common packages

%% See the TeXed file for more explanations

%% [OPT] Multi-rowed cells in tabulars
\usepackage{multirow}

%% [REC] Intelligent cross reference package. This allows for nice
%% combined references that include the reference and a hint to where
%% to look for it.
\usepackage{varioref}

%% [OPT] Easily changeable quotes with \enquote{Text}
%\usepackage[german=swiss]{csquotes}

%% [REC] Format dates and time depending on locale
\usepackage{datetime}

%% [OPT] Provides a \cancel{} command to stroke through mathematics.
\usepackage{cancel}

%% [NEED] This allows for additional typesetting tools in mathmode.
%% See its excellent documentation.
\usepackage{mathtools}

%% [ADV] Conditional commands
%\usepackage{ifthen}

%% [OPT] Manual large braces or other delimiters.
%\usepackage{bigdelim, bigstrut}

%% [REC] Alternate vector arrows. Use the command \vv{} to get scaled
%% vector arrows. (package texlive-fonts-extra)
\usepackage[h]{esvect}

%% [NEED] Some extensions to tabulars and array environments.
\usepackage{array}

%% [OPT] Postscript support via pstricks graphics package. Very
%% diverse applications.
%\usepackage{pstricks,pst-all}

%% [?] This seems to allow us to define some additional counters.
%\usepackage{etex}

%% [ADV] XY-Pic to typeset some matrix-style graphics
%\usepackage[all]{xy}

%% [OPT] This is needed to generate an index at the end of the
%% document.
%\usepackage{makeidx}

%% [OPT] Fancy package for source code listings.  The template text
%% needs it for some LaTeX snippets; remove/adapt the \lstset when you
%% remove the template content.
\usepackage{listings}
\lstset{language=TeX,basicstyle={\normalfont\ttfamily}}

%% [REC] Fancy character protrusion.  Must be loaded after all fonts.
\usepackage[activate]{pdfcprot}

%% [REC] Nicer tables.  Read the excellent documentation.
\usepackage{booktabs}

%% International System measurement units (package texlive-science)
\usepackage{siunitx}

%% Subfigures
%\let\subcaption\undefined
%\let\subfloat\undefined
%\usepackage{subcaption}

%section customisation
\usepackage{titlesec}

%% Advanced figures
\usepackage{tikz}

%% Electronics circuits
%\usepackage[arrowmos]{circuitikz}

%%Image position
\usepackage{float}

%%Long tables
\usepackage{longtable}
\usepackage{tabu}

%% LaTeX' own graphics handling
\usepackage{graphicx}

%% The AMS-LaTeX extensions for mathematical typesetting.  Do not
%% remove.
\usepackage{amsmath,amssymb,amsfonts,mathrsfs,amscd,xspace}

%% NTheorem is a reimplementation of the AMS Theorem package. This
%% will allow us to typeset theorems like examples, proofs and
%% similar.  Do not remove.
%% NOTE: Must be loaded AFTER amsmath, or the \qed placement will
%% break
\usepackage[amsmath,thmmarks]{ntheorem}

%different enumerations
\usepackage{enumitem}

%% Make document internal hyperlinks wherever possible. (TOC, references)
%% This MUST be loaded after varioref, which is loaded in 'extrapackages'
%% above.  We just load it last to be safe.
\usepackage[linkcolor=black,colorlinks=true,urlcolor=black,citecolor=black,filecolor=black]{hyperref}


\input{glyphtounicode}
  \pdfgentounicode=1
\usepackage{cmap}

\usepackage{accsupp}
\usepackage{calc}
\usepackage{layouts}
\usepackage{layout}

 
\mathtoolsset{showonlyrefs}  

% Lorem ipsum
%\usepackage[]{blindtext}
\usepackage{lipsum}% dummy text
% include pdfs into the latex document
\usepackage{pdfpages}
%for landscape cheatsheet
\usepackage{pdflscape}


% Units
\usepackage{units}

% tables
\usepackage{array}
\usepackage{rotating}
\usepackage{longtable}

%layout
\usepackage{multicol}
\setlength{\columnseprule}{0.4pt}
\usepackage{chngpage}
 %% Some more packages that you may want to use.  Have a look at the
%% file, and consult the package docs for each.

%% Theorem-like environments

%% This can be changed according to language. You can comment out the ones you
%% don't need.

\numberwithin{equation}{chapter}

%% German theorems
%\newtheorem{satz}{Satz}[chapter]
%\newtheorem{beispiel}[satz]{Beispiel}
%\newtheorem{bemerkung}[satz]{Bemerkung}
%\newtheorem{korrolar}[satz]{Korrolar}
%\newtheorem{definition}[satz]{Definition}
%\newtheorem{lemma}[satz]{Lemma}
%\newtheorem{proposition}[satz]{Proposition}

%% English variants
\newtheorem{theorem}{Theorem}[chapter]
\newtheorem{example}[theorem]{Example}
\newtheorem{remark}[theorem]{Remark}
\newtheorem{corollary}[theorem]{Corollary}
\newtheorem{definition}[theorem]{Definition}
\newtheorem{lemma}[theorem]{Lemma}
\newtheorem{proposition}[theorem]{Proposition}
\newtheorem{axiom}[theorem]{Axiom}
%\theoremstyle{definition}
%\newtheorem{definition}[theorem]{Definition}
%\theoremstyle{remark}
%\newtheorem{remark}[theorem]{Remark}

%% Proof environment with a small square as a "qed" symbol
\theoremstyle{nonumberplain}
\theorembodyfont{\normalfont}
\theoremsymbol{\ensuremath{\square}}
\newtheorem{proof}{Proof}
%\newtheorem{beweis}{Beweis}
 %% Theorem environments.  You will have to adapt this for a German thesis.

%%% Custom commands
%% ===============

%% Special characters for number sets, e.g. real or complex numbers.
\newcommand{\C}{\mathbb{C}}
\newcommand{\K}{\mathbb{K}}
\newcommand{\N}{\mathbb{N}}
\newcommand{\Q}{\mathbb{Q}}
\newcommand{\R}{\mathbb{R}}
\newcommand{\Z}{\mathbb{Z}}
\newcommand{\X}{\mathbb{X}}

% surrounding every content with the math environment does make the content copyable from the pdf document back into latex form. In some cases for example in captions or section titles, you will need to add \protect before the printlatex command, otherwise you get a strange error about a } too many.
%Usage:: \(\printlatex{2^i}\) or \(\pl{2^i}\) as shorthand
\newcommand*{\printlatex}[1]{%
  \BeginAccSupp{%
    ActualText=\detokenize{#1},%
    method=escape,
  }%
  #1%
  \EndAccSupp{}%
}
\newcommand{\pl}[1]{\printlatex{#1}}

\newcommand{\mc}[1]{\mathcal{#1}}

%% Special characters for Expected value |E , Variance \V, \I, Prediction error \predR
\newcommand{\E}{\mathbb{E}}
\newcommand{\V}{\mathbb{V}}
\newcommand{\I}{\mathbb{I}}
\newcommand{\predR}{\mathcal{R}}

%% Fixed/scaling delimiter examples (see mathtools documentation)
\DeclarePairedDelimiter\abs{\lvert}{\rvert}
\DeclarePairedDelimiter\norm{\lVert}{\rVert}

%% Use the alternative epsilon per default and define the old one as \oldepsilon
\let\oldepsilon\epsilon
\renewcommand{\epsilon}{\ensuremath\varepsilon}

%% Also set the alternate phi as default.
%\let\oldphi\phi
%\renewcommand{\phi}{\ensuremath{\varphi}}

%% create the signum function for mathematical formulas
\newcommand{\sgn}{\operatorname{sgn}}

\DeclareMathOperator*{\xpt}{\textit{E}}
\newcommand{\argmin}{\arg\!\min}
\newcommand{\argmax}{\arg\!\max}
 %% Helpful macros.

% lslisting style for python
\lstset{
	basicstyle=\ttfamily,
	breaklines=true,
	commentstyle=\color{green},
	keepspaces=true,
	keywordstyle=\color{blue},
	language=Python,
	morekeywords={off},
	showstringspaces=false,
	stringstyle=\color{purple},
	title=\lstname
}

% lslisting style for matlab
\lstset{
	basicstyle=\ttfamily,
	breaklines=true,
	commentstyle=\color{green},
	keepspaces=true,
	keywordstyle=\color{blue},
	language=Matlab,
	morekeywords={off},
	showstringspaces=false,
	stringstyle=\color{purple},
	title=\lstname
}

\pdfpagewidth=\paperwidth
\pdfpageheight=\paperheight

\expandafter\def\expandafter\normalsize\expandafter{%
    \normalsize
    \setlength\abovedisplayskip{4pt}
    \setlength\belowdisplayskip{4pt}
    \setlength\abovedisplayshortskip{4pt}
    \setlength\belowdisplayshortskip{4pt}
}

% define colors for cheatsheet
%https://en.wikibooks.org/wiki/LaTeX/Colors#Predefined_colors
\definecolor{sectionColor}{HTML}{FF7F00}
\definecolor{subsectionColor}{HTML}{EE0000}
\definecolor{subsubsectionColor}{HTML}{EE6600}


\renewcommand{\familydefault}{\sfdefault}

\usepackage[hyperref=true,
            url=true,
            isbn=false,
            backend=biber,
            backref=true,
            bibencoding=utf8,
            style=custom-numeric-comp,
            citereset=chapter,
            maxcitenames=3,
            maxbibnames=100,
            block=none]{biblatex}


% the followings activate 'custom-english-ordinal-sscript.lbx'
% in order to print ordinal 'edition' suffixes as superscripts,
% and adjusts (reduces) spacing between suffix and following "ed."
\DeclareLanguageMapping{english}{custom-english-ordinal-sscript}
\DeclareFieldFormat{edition}%
                   {\ifinteger{#1}%
                    {\mkbibordedition{#1}\addthinspace{}ed.}%
                    {#1\isdot}}


% back reference text preceding the page number ("see p.")
\DefineBibliographyStrings{english}{%
    backrefpage  = {see p.}, % for single page number
    backrefpages = {see pp.} % for multiple page numbers
}

% removes period at the very end of bibliographic record
\renewcommand{\finentrypunct}{}

% removes period after DOI and suppresses capitalization
% of the word following DOI ("See p. xx" -> "see p. xx")
\renewcommand{\newunitpunct}{\addspace\midsentence}

\DeclareFieldFormat{journaltitle}{\mkbibemph{#1},} % italic journal title with comma
\DeclareFieldFormat[inbook,thesis]{title}{\mkbibemph{#1}\addperiod} % italic title with period
\DeclareFieldFormat[article]{title}{#1} % title of journal article is printed as normal text
\DeclareFieldFormat[article]{volume}{\textbf{#1}\addcolon\space} % makes volume of journal bold and adds colon
\DeclareFieldFormat{pages}{#1} % removes pagination (p./pp.) before page numbers

%%%%%%%%%
% the command \sjcitep defined below prints footnote citation above punctuation
\newlength{\spc} % declare a variable to save spacing value
\newcommand{\sjcitep}[2][]{% new command with two arguments: optional (#1) and mandatory (#2)
        \settowidth{\spc}{#1}% set value of \spc variable to the width of #1 argument
        \addtolength{\spc}{-1.8\spc}% subtract from \spc about two (1.8) of its values making its magnitude negative
        #1% print the optional argument
        \hspace*{\spc}% print an additional negative spacing stored in \spc after #1
        \supershortnotecite{#2}}% print (cite) the mandatory argument
%%%%%%%%%


\usepackage[justification=raggedright,singlelinecheck=off]{caption} %%page layout settings and listing templates etc.

\addbibresource{Bibliography.bib} % The references (bibliography) information are stored in the file named "Bibliography.bib"

\title{\ttitle} % Defines the thesis title - don't touch this

\begin{document}

\frontmatter % Use roman page numbering style (i, ii, iii, iv...) for the pre-content pages

\setstretch{1.3} % Line spacing of 1.3

% Define the page headers using the FancyHdr package and set up for one-sided printing
\fancyhead{} % Clears all page headers and footers
\rhead{\thepage} % Sets the right side header to show the page number
\lhead{} % Clears the left side page header

\pagestyle{fancy} % Finally, use the "fancy" page style to implement the FancyHdr headers

\newcommand{\HRule}{\rule{\linewidth}{0.5mm}} % New command to make the lines in the title page

% PDF meta-data
\hypersetup{pdftitle={\ttitle}}
\hypersetup{pdfsubject=\subjectname}
\hypersetup{pdfauthor=\authornames}
\hypersetup{pdfkeywords=\keywordnames}

%----------------------------------------------------------------------------------------
%	TITLE PAGE
%----------------------------------------------------------------------------------------

\begin{titlepage}
\begin{center}

\textsc{\LARGE \univname}\\[1.5cm] % University name
\textsc{\Large Master's Thesis}\\[0.5cm] % Thesis type

\HRule \\[0.4cm] % Horizontal line
{\huge \bfseries \ttitle}\\[0.4cm] % Thesis title
\HRule \\[1.5cm] % Horizontal line
 
\begin{minipage}{0.45\textwidth}
\begin{flushleft} \large
\emph{Author:}\\
\href{http://www.sleepy-robots.org}{\authornames \\ 09--919--622} % Author name - remove the \href bracket to remove the link
\end{flushleft}
\end{minipage}
\begin{minipage}{0.45\textwidth}
\begin{flushright} \large
\emph{Supervisor:} \\
\href{http://stoop.ini.uzh.ch}{\supname\linebreak} % Supervisor name - remove the \href bracket to remove the link
\end{flushright}
\end{minipage}\\[3cm]
 
\large \textit{A thesis submitted in fulfilment of the requirements\\ for the degree of \degreename} % University requirement text
\textit{in the}\\[0.4cm]
\groupname/\facname\\\deptname\\[2cm] % Research group name and department name
 
{\large Feb 29, 2015}\\[2.5cm] % Date
%\includegraphics{uzh_logo_e_pos} % University/department logo - uncomment to place it
 
\vfill
\end{center}

\end{titlepage}
%----------------------------------------------------------------------------------------
%	DECLARATION PAGE
%	Your institution may give you a different text to place here
%----------------------------------------------------------------------------------------

\Declaration{

\addtocontents{toc}{\vspace{1em}} % Add a gap in the Contents, for aesthetics

I, \authornames, declare that this thesis titled, '\ttitle' and the work presented in it are my own. I confirm that:

\begin{itemize} 
\item[\tiny{$\blacksquare$}] This work was done wholly or mainly while in candidature for a research degree at this University.
\item[\tiny{$\blacksquare$}] Where any part of this thesis has previously been submitted for a degree or any other qualification at this University or any other institution, this has been clearly stated.
\item[\tiny{$\blacksquare$}] Where I have consulted the published work of others, this is always clearly attributed.
\item[\tiny{$\blacksquare$}] Where I have quoted from the work of others, the source is always given. With the exception of such quotations, this thesis is entirely my own work.
\item[\tiny{$\blacksquare$}] I have acknowledged all main sources of help.
\item[\tiny{$\blacksquare$}] Where the thesis is based on work done by myself jointly with others, I have made clear exactly what was done by others and what I have contributed myself.\\
\end{itemize}
 
Signed:\\
\rule[1em]{25em}{0.5pt} % This prints a line for the signature
 
Date:\\
\rule[1em]{25em}{0.5pt} % This prints a line to write the date
}

\clearpage % Start a new page

%----------------------------------------------------------------------------------------
%	QUOTATION PAGE
%----------------------------------------------------------------------------------------
\begin{comment}
\pagestyle{empty} % No headers or footers for the following pages

\null\vfill % Add some space to move the quote down the page a bit

\textit{''It is not the strongest of the species that survive, nor the most intelligent, but the one most responsive to change.''}

\begin{flushright}
Charles Darwin
\end{flushright}

\vfill\vfill\vfill\vfill\vfill\vfill\null % Add some space at the bottom to position the quote just right

\clearpage % Start a new page
\end{comment}
%----------------------------------------------------------------------------------------
%	ABSTRACT PAGE
%----------------------------------------------------------------------------------------

\addtotoc{Abstract} % Add the "Abstract" page entry to the Contents

\abstract{\addtocontents{toc}{\vspace{1em}} % Add a gap in the Contents, for aesthetics

% rev. 1

\textbf{Motivation.} The evolution of locomotion was of central importance for the spread of the first complex life in the ocean and the transition from the ocean to land. %
%
Locomotion is omnipresent in simple and complex organisms of micro and macroscopic level, thus there must exist an underlying natural process that simplifies its emergence. %
%
When observing a variety of locomotion gait patterns in nature, it becomes obvious how robust animal gaits are against perturbations. %
%
Therefore the underlying system has to be adaptive to the quickly changing environment locomotion takes place in. %
%
Unfortunately, our understanding of the fundamental process of adaptive gaits is still limited. %
%
In robotic applications, locomotion implemented using computationally expensive planning methods often fails to copy the delicate behavior of animals, especially when it comes to keeping locomotion robust and adaptive in unpredictable situations. %
%
The difference might be that robot locomotion often relies on a large sensory interface to sense the environment and an internal model to plan the next actuation of the limbs based on the current state of movement. %
%
Natural walking however relies on central pattern generators (CPG), which emit oscillatory signals to control and coordinate limb movement. %
%
It has been shown that the pyloric central pattern generators of the spiny lobster are closed neural systems that undergo chaotic oscillations \cite{bib:Rabinovich1997}. %
%
We hypothesize that the underlying natural process to simplify the emergence of adaptive, oscillatory movement in locomotion patterns are chaotic systems, which are controlled into different periodicities and adapt their oscillation frequency based on feedback from the environment. %
%
The adaptive system consists not only of the central pattern generator, though; the whole complex of pattern generators, the limbs and the connecting joints all work together to self-adapt to the external constraints. \\
%
%
%
\textbf{Methods.} This work focusses on finding examples of locomoting, simulated creatures that use chaotic systems to control periodic leg movement. %
%
Therefore, an example chaotic system was chosen, the Chua circuit, to be controlled by a chaos control method called simple limiter control \cite{bib:Corron2000}. %
%
The limiting component, the simple limiter, can be set appropriately so that the chaotic system presents chaotic and periodic trajectories. %
%
Furthermore, a simulator was implemented to evolve simple creatures in a rigid-body physics engine using an evolutionary process, a terrestrial flat plane environment, and a genome and phenotype structure to encode a morphology out of 3D primitives (such as boxes and capsules) connected by joints and a controller structure using the chaotic system. %
%
To compare the results of evolution using chaotic controllers, reference simulations were performed using a sinusoidal controller. %
%
Some preliminary experiments were performed to replicate the simple limiter control method and explore the influence of soft limiters and different limiter configurations on the behavior of the Chua circuit. %
%
Then a chaotic controller was configured with different limiters by the movement of the joint of a simple model leg. %
%
The model leg was set up in different settings: gravity-less space and interacting with ground, additionally with different joint torques and damping settings. %
%
Finally, the evolution simulator was run to evolve locomoting creatures on land using the chaotic controller. \\
%
%
%
\textbf{Results.} The simple limiter control method benefits from soft limiters in the sense that the simple limiter parameter ranges of different stabilized limit cycles are increased. %
%
Furthermore, under which conditions the simple limiter is applied to which part of the system plays an important role in the success of the control method and must be chosen wisely. %
%
In the simulation of the model leg, only limit cycles of periodicity 1 could be found. In the absence of gravitation, the model leg always converged to the limit cycle. %
%
In the interaction with ground, smaller limit cycles were stabilized, but the system tried to converge back to the uninfluenced larger limit cycle. %
%
It was found that damping leads to a smoother curvature of the limit cycle within the original undamped limit cycle. %
%
The increased joint torque leads to a larger limit cycle, however, contrary to expectations, does not make the system switch to a higher periodicity in either the controller or joint phase space. %
%
The evolution simulator successfully evolved locomoting creatures using both the sinusoidal and the feedback limited, chaotic controller. %
%
In the case of the creatures using the chaotic controllers, a more in-depth analysis of the trajectories showed that higher periodicity can be observed both in the controller and joint phase space. %
%
Furthermore it could be shown that the gaits generated by a chaotic controller are of high robustness towards perturbations. %

}

\clearpage % Start a new page

%----------------------------------------------------------------------------------------
%	ACKNOWLEDGEMENTS
%----------------------------------------------------------------------------------------

\setstretch{1.3} % Reset the line-spacing to 1.3 for body text (if it has changed)

\acknowledgements{\addtocontents{toc}{\vspace{1em}} % Add a gap in the Contents, for aesthetics

I would like to thank Professor Ruedi Stoop for his patience, numerous advices and kind critique. I am very grateful for all the support, discussions and material I was provided by the Stoop group, especially I want to thank Thomas Lorimer and Karlis Kanders for their help with the implementation of the chaotic model and limiter control, the interpretation of the data and the helpful discussions leading to different improvements of this thesis. In addition I want to thank my girlfriend Perrine Dubuis for her support during my studies and work on my thesis, without which I would not have had the same patience for the countless failed experiments and evolutionary runs without usable results. Furthermore, I want to thank Randolph Busch and Karlis Kanders for proof-reading this thesis and for finding inconsistencies in reasoning and explanations.


}
\clearpage % Start a new page

%----------------------------------------------------------------------------------------
%	LIST OF CONTENTS/FIGURES/TABLES PAGES
%----------------------------------------------------------------------------------------

\pagestyle{fancy} % The page style headers have been "empty" all this time, now use the "fancy" headers as defined before to bring them back

\lhead{\emph{Contents}} % Set the left side page header to "Contents"
\tableofcontents % Write out the Table of Contents

\lhead{\emph{List of Figures}} % Set the left side page header to "List of Figures"
\listoffigures % Write out the List of Figures

\lhead{\emph{List of Tables}} % Set the left side page header to "List of Tables"
\listoftables % Write out the List of Tables

%----------------------------------------------------------------------------------------
%	ABBREVIATIONS
%----------------------------------------------------------------------------------------
\begin{comment}
\clearpage % Start a new page

\setstretch{1.5} % Set the line spacing to 1.5, this makes the following tables easier to read

\lhead{\emph{Abbreviations}} % Set the left side page header to "Abbreviations"
\listofsymbols{ll} % Include a list of Abbreviations (a table of two columns)
{
\textbf{CPG} & \textbf{C}entral \textbf{P}attern \textbf{G}enerators \\ \\

& \\

Chaos Theory: \\

\textbf{UPO} & \textbf{U}nstable \textbf{P}eriodic \textbf{O}rbit\\

\textbf{CoR} & \textbf{C}enter \textbf{o}f \textbf{R}otation

%\textbf{Acronym} & \textbf{W}hat (it) \textbf{S}tands \textbf{F}or \\
}
\end{comment}
\begin{comment}
%----------------------------------------------------------------------------------------
%	PHYSICAL CONSTANTS/OTHER DEFINITIONS
%----------------------------------------------------------------------------------------

\clearpage % Start a new page

\lhead{\emph{Physical Constants}} % Set the left side page header to "Physical Constants"

\listofconstants{lrcl} % Include a list of Physical Constants (a four column table)
{
%Speed of Light & $c$ & $=$ & $2.997\ 924\ 58\times10^{8}\ \mbox{ms}^{-\mbox{s}}$ (exact)\\
% Constant Name & Symbol & = & Constant Value (with units) \\
}

%----------------------------------------------------------------------------------------
%	SYMBOLS
%----------------------------------------------------------------------------------------

\clearpage % Start a new page

\lhead{\emph{Symbols}} % Set the left side page header to "Symbols"

\listofnomenclature{lll} % Include a list of Symbols (a three column table)
{
%$a$ & distance & m \\
%$P$ & power & W (Js$^{-1}$) \\
% Symbol & Name & Unit \\

& & \\ % Gap to separate the Roman symbols from the Greek

%$\omega$ & angular frequency & rads$^{-1}$ \\
% Symbol & Name & Unit \\
}

%----------------------------------------------------------------------------------------
%	DEDICATION
%----------------------------------------------------------------------------------------

\setstretch{1.3} % Return the line spacing back to 1.3

\pagestyle{empty} % Page style needs to be empty for this page

\dedicatory{Dedicated to science and great minds helping to lead this world to a better tomorrow.} % Dedication text

\addtocontents{toc}{\vspace{2em}} % Add a gap in the Contents, for aesthetics
\end{comment}
%----------------------------------------------------------------------------------------
%	THESIS CONTENT - CHAPTERS
%----------------------------------------------------------------------------------------

\mainmatter % Begin numeric (1,2,3...) page numbering

\pagestyle{fancy} % Return the page headers back to the "fancy" style

% Introduction
%  Locomotion in robots
%  A more natural approach
%   Evolutionary Algorithms
%   Chaotic systems as a source of high variability
%   Simple Limiter Control
\subfile{./Chapter1-Introduction}

% Evolutionary Optimization
% Basic components
%  Universe, Planets and Environments
%   Planet Physics
%  Epochs
%   Fitness functions
%  Populations
%  Creatures
%   Genotypes
%   Phenotypes
%    Constrained Rigid body model
%    Featherstone Multi-rigidbody model
%  Model organisms
% Reaper
%  Crossover
%  Mutations
% Evaluation Step
% Mutation Step
\subfile{./Chapter2-EvolutionaryOptimization}

% Controllers
%  Uncoupled sinusoidal oscillators
%  Chaotic systems
%  Chua Circuit
%   Simple limiter control in Mathematica
\subfile{./Chapter3-Controllers}

% Results
%  Creatures with uncoupled sinusoidal oscillators
%  Simple limiter control in the simulator
%   Indirect limiter control through the morphology
%   Direct limiter control through the sensor feedback
%    Limiting the Model Leg in the Minemonics simulator
%    Evolved examples 
\subfile{./Chapter4-Results}


% Discussion and Outlook
%  Evolutionary limiting of the controller
\subfile{./Chapter5-DiscussionAndOutlook}


%----------------------------------------------------------------------------------------
%	THESIS CONTENT - APPENDICES
%----------------------------------------------------------------------------------------

\addtocontents{toc}{\vspace{2em}} % Add a gap in the Contents, for aesthetics

\appendix % Cue to tell LaTeX that the following 'chapters' are Appendices

% Include the appendices of the thesis as separate files from the Appendices folder
% Uncomment the lines as you write the Appendices

% Using the Largest Lyapunov Exponent as a measure for gait stability
\subfile{./AppendixA-LyapunovExponent}


\addtocontents{toc}{\vspace{2em}} % Add a gap in the Contents, for aesthetics

\backmatter

%----------------------------------------------------------------------------------------
%	BIBLIOGRAPHY
%----------------------------------------------------------------------------------------

\label{Bibliography}

\lhead{\emph{Bibliography}} % Change the page header to say "Bibliography"

%\bibliographystyle{unsrtnat} % Use the "unsrtnat" BibTeX style for formatting the Bibliography

%\bibliography{Bibliography} % The references (bibliography) information are stored in the file named "Bibliography.bib"

\printbibliography

%\subfile{TODO.tex}

\end{document}
