% lslisting style for python
\lstset{
	basicstyle=\ttfamily,
	breaklines=true,
	commentstyle=\color{green},
	keepspaces=true,
	keywordstyle=\color{blue},
	language=Python,
	morekeywords={off},
	showstringspaces=false,
	stringstyle=\color{purple},
	title=\lstname
}

% lslisting style for matlab
\lstset{
	basicstyle=\ttfamily,
	breaklines=true,
	commentstyle=\color{green},
	keepspaces=true,
	keywordstyle=\color{blue},
	language=Matlab,
	morekeywords={off},
	showstringspaces=false,
	stringstyle=\color{purple},
	title=\lstname
}

\pdfpagewidth=\paperwidth
\pdfpageheight=\paperheight

\expandafter\def\expandafter\normalsize\expandafter{%
    \normalsize
    \setlength\abovedisplayskip{4pt}
    \setlength\belowdisplayskip{4pt}
    \setlength\abovedisplayshortskip{4pt}
    \setlength\belowdisplayshortskip{4pt}
}

% define colors for cheatsheet
%https://en.wikibooks.org/wiki/LaTeX/Colors#Predefined_colors
\definecolor{sectionColor}{HTML}{FF7F00}
\definecolor{subsectionColor}{HTML}{EE0000}
\definecolor{subsubsectionColor}{HTML}{EE6600}


\renewcommand{\familydefault}{\sfdefault}

\usepackage[hyperref=true,
            url=true,
            isbn=false,
            backend=biber,
            backref=true,
            bibencoding=utf8,
            style=custom-numeric-comp,
            citereset=chapter,
            maxcitenames=3,
            maxbibnames=100,
            block=none]{biblatex}


% the followings activate 'custom-english-ordinal-sscript.lbx'
% in order to print ordinal 'edition' suffixes as superscripts,
% and adjusts (reduces) spacing between suffix and following "ed."
\DeclareLanguageMapping{english}{custom-english-ordinal-sscript}
\DeclareFieldFormat{edition}%
                   {\ifinteger{#1}%
                    {\mkbibordedition{#1}\addthinspace{}ed.}%
                    {#1\isdot}}


% back reference text preceding the page number ("see p.")
\DefineBibliographyStrings{english}{%
    backrefpage  = {see p.}, % for single page number
    backrefpages = {see pp.} % for multiple page numbers
}

% removes period at the very end of bibliographic record
\renewcommand{\finentrypunct}{}

% removes period after DOI and suppresses capitalization
% of the word following DOI ("See p. xx" -> "see p. xx")
\renewcommand{\newunitpunct}{\addspace\midsentence}

\DeclareFieldFormat{journaltitle}{\mkbibemph{#1},} % italic journal title with comma
\DeclareFieldFormat[inbook,thesis]{title}{\mkbibemph{#1}\addperiod} % italic title with period
\DeclareFieldFormat[article]{title}{#1} % title of journal article is printed as normal text
\DeclareFieldFormat[article]{volume}{\textbf{#1}\addcolon\space} % makes volume of journal bold and adds colon
\DeclareFieldFormat{pages}{#1} % removes pagination (p./pp.) before page numbers

%%%%%%%%%
% the command \sjcitep defined below prints footnote citation above punctuation
\newlength{\spc} % declare a variable to save spacing value
\newcommand{\sjcitep}[2][]{% new command with two arguments: optional (#1) and mandatory (#2)
        \settowidth{\spc}{#1}% set value of \spc variable to the width of #1 argument
        \addtolength{\spc}{-1.8\spc}% subtract from \spc about two (1.8) of its values making its magnitude negative
        #1% print the optional argument
        \hspace*{\spc}% print an additional negative spacing stored in \spc after #1
        \supershortnotecite{#2}}% print (cite) the mandatory argument
%%%%%%%%%


\usepackage[justification=raggedright,singlelinecheck=off]{caption}